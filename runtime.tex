\subsection{Performance Analysis}

To get an overview of the capabilities of the monitor implementations, each of them were run on at least 100 traces with 1.000 events, which were randomly generated by following specific parameters to show, which of these parameters result in faster or slower run times. For this evaluation, the TeSSLa interpreter version 1.0.12 were used and it was run on a computer with a i5-6600k processor running on 4.3 GHz. The operating system was Windows 10.\\ %TODO excact windows version
The run times were measured as time between per event. For that, a program\footnote{This program and the complete measured run times can be found at \href{https://github.com/HendrikStreichhahn/TeSSLa-Autosar-Timing-Extensions/tree/master/traceGenerator}{https://github.com/HendrikStreichhahn/TeSSLa-Autosar-Timing-Extensions/tree/master/traceGenerator}} was written, which generates traces for each constraint and then measures the time between the input of the events of one timestamp and the output of the TeSSLa interpreter. The communication between the test program and the TeSSLa interpreter is done via the \textit{standard input} and \textit{standard output stream} of the interpreter. The time is measured by the java function \textit{System.nanoTime()} immediately before the events are written into the input stream and immediately after an reaction was received on the output stream. It must be noted, that this time measurement is not completely accurate, because neither the used java runtime environment, nor the operating system were build to fulfill real time requirements. Therefore, unpredictable delays may occur in the test program, in the java interpreter or between them, but the results show, what the monitors are capable of and where it limits are.\\
For every Trace, the minimum, the maximum and the average time per event was stored, additionally the overall minimum, overall maximum and overall average of the run times were determined.

\subsubsection{DelayConstraint}
	The \textit{DelayConstraint} was evaluated with 100 Traces of 1.000 events. The traces fulfilled the constraint with the attributes $lower\in\{100, 200, 300, 400, 500, 600, 700, 800, 900, 1000\}$ and $upper=lower$.  The distance of $source$ subsequent were $distance\in\{1, 2, 4, 16, 32, 64, 128, 256, 512, 1024\}$, while the distance was smaller than $2*lower$ in.  The run time per event was between 0.23ms and 55.3ms, averaging at 1.2ms per event. As expected, the highest average run times were by the traces with low event distances, because these traces require to store more timestamps. In table~\ref{tab:runtimeDelay} are the average run times for this constraint with the parameters $lower=upper=1000$ in dependency of the time between subsequent $stimulus$ events. It can be seen, that these differences are fairly small, because the trace generator does not create worst case traces for this constraint, where all stored $stimulus$ events must be removed in one one timestamp.
	\begin{table}
		\begin{tabular}{|c|c|c|c|c|c|c|c|c|c|c|c|}
			\hline 
			\makecell{Distance\\$stimulus$\\events}  & 	1	 & 2 	  & 4 	   & 8 	   & 16 	& 32 	 & 64 	  & 128    & 256 	& 512 	& 1024\\
			\hline
			\makecell{avg. run time\\per event(ms)}	& 1.92 & 1.95 & 1.72 & 1.4 & 1.21 & 1.12 & 1.08 & 0.94 & 0.95 & 1.02 & 1.08\\
			\hline
		\end{tabular}
		\centering
		\caption{Run times of the \textit{DelayConstraint}(1000, 1000)}
		\label{tab:runtimeDelay}
	\end{table}

\subsubsection{StrongDelayConstraint}
	The traces for the \textit{StrongDelayConstraint} were created by using the same parameters as the \textit{DelayConstraint}, of course the traces fulfill the \textit{StrongDelayConstraint} this time. The runtime per event was between 0.14ms and 66.9ms, averaging around 1 ms. In table~\ref{tab:runtimeStrongDelay} the average run times for this constraint with the parameters $lower=upper=1000$ can be seen. The run times are fairly constant, beside some measurement errors.

	\begin{table}	
		\begin{tabular}{|c|c|c|c|c|c|c|c|c|c|c|c|}
			\hline 
			\makecell{Distance\\$stimulus$\\events}  & 	1	 & 2 	  & 4 	   & 8 	   & 16 	& 32 	 & 64 	  & 128    & 256 	& 512 	& 1024\\
			\hline
			\makecell{avg. run time\\per event(ms)}	& 1.03 & 1.03 & 1.35 & 1.19 & 0.97 & 1.04 & 0.87 & 0.94 & 1.12 & 0.93 & 1.14\\
			\hline
		\end{tabular}
		\centering
		\caption{Run times of the \textit{StrongDelayConstraint}(1000, 1000)}
		\label{tab:runtimeStrongDelay}
	\end{table}

\subsubsection{RepeatConstraint}
	%TODO nicht plausibel
	The \textit{RepeatConstraint} was evaluated with 100 Traces of 1.000 events. The traces were created with the attributes $span\in\{1,2,3,4\}$, $lower=\{500,600,700,800,900\}$ and $upper=lower+x$, $x\in{100,200,300,400}$. The  runtime over the traces were between 0.29ms and 41.43ms with an average of 0.72ms. A correlation between the trace parameters and the run time could not be observed.
	
	\begin{table}
		\begin{tabular}{|c|c|c|c|}
			\hline
							& Minimum &  Maximum &  Average \\
			\hline
			overall Minimum & 0.29ms & 31.24ms & 0.62ms\\
			\hline
			overall Maximum & 0.48ms & 41.43ms & 0.82ms\\
			\hline
			overall Average & 0.38ms & 33.68ms & 0.72ms\\
			\hline
		\end{tabular}
		\centering
		\caption{Run times of the \textit{RepeatConstraint}}
		\label{tab:runtimeRepeat}
	\end{table}

\subsubsection{RepetitionConstraint}
	%TODO nicht plausibel
	The traces for this constraint was created with the same parameters as for the \textit{RepeatConstraint}, additionally the $jitter$ was set $\frac{lower}{10}$. The run time per event was between 0.14ms and 41.92ms, averaging around 0.94ms. Similar to the previous constraint, no correlation between 
	\begin{table}
		\begin{tabular}{|c|c|c|c|}
			\hline
							& Minimum &  Maximum &  Average \\
			\hline
			overall Minimum & 0.014ms & 29.92ms & 0.8ms\\
			\hline
			overall Maximum & 0.46ms & 41.92ms & 1.16ms\\
			\hline
			overall Average & 0.024ms & 32.18ms & 0.94ms\\
			\hline
		\end{tabular}
		\centering
		\caption{Run times of the \textit{RepetitionConstraint}}
		\label{tab:runtimeRepetition}
	\end{table}
	
	
	
	
\subsubsection{SynchronizationConstraint}
%TODO
\begin{table}
	\begin{tabular}{|c|c|c|c|}
		\hline
		& Minimum &  Maximum &  Average \\
		\hline
		overall Minimum & ms & ms & ms\\
		\hline
		overall Maximum & ms & ms & ms\\
		\hline
		overall Average & ms & ms & ms\\
		\hline
	\end{tabular}
	\centering
	\caption{Run times of the \textit{SynchronizationConstraint}}
	\label{tab:runtimeSynchronization}
\end{table}

\subsubsection{StrongSynchronizationConstraint}
% TODO nochmal laufen lassen
The Traces for the \textit{StrongSynchronizationConstraint} were generated with 2 to 5 event streams and the parameter $tolerance\in\{10, 13, ..., 25\}$. The starting of subsequent synchronization clusters were placed 1, 2, 4, 8 or 16 timestamps apart. The run times are between 0.015ms and 76.81ms, with an average of 2.12ms. 
\begin{table}
	\begin{tabular}{|c|c|c|c|}
		\hline
		& Minimum &  Maximum &  Average \\
		\hline
		overall Minimum & 0.015ms & 33.57ms & 1.46ms\\
		\hline
		overall Maximum & 0.43ms & 76.81ms & 4.36ms\\
		\hline
		overall Average & 0.022ms & 36.13ms & 2.12ms\\
		\hline
	\end{tabular}

	\centering
	\caption{Run times of the \textit{StrongSynchronizationConstraint}, $|event|=5$, $clusterDistance=1$}
	\label{tab:runtimeStrongSynchronization}
\end{table}

\subsubsection{ExecutionTimeConstraint}
	The run time evaluation of the \textit{ExecutionTimeConstraint} monitor was done by traces, which fulfill the constraint the parameters $lower\in\{100,300,500,700,900\}$ and $upper=lower+x$, $x\in\{100,600,1100,1600,2100\}$. For each of these parameters, one trace with 1, 11, 21 and 31 preemptions between the each $start$ and $end$ event were created. The run times were between 0.013ms and 45.9ms with an average of 0.62ms. In table~\ref{tab:runtimeExecutionTimeConstraint}, the run times of the monitor with the parameters $lower = 300$ and $upper = 400...2400$ and 1 to 31 preemptions can be seen. A correlation between the input parameters and the run times can not be observed, which was expected, because the run time is independent from the parameters or the placement of events, which can be seen in the runtime, like stated in chapter~\ref{chapter-implementation}.
\begin{table}
	\begin{tabular}{|c|c|c|c|c|}
		\hline
					& \makecell{preemptions\\ = 1} & \makecell{preemptions\\ = 11} & \makecell{preemptions\\ = 21} & \makecell{preemptions\\ = 31}\\
		\hline
		upper = 400	& 	0.64ms		  & 0.61ms			 &	0.57ms			& 0.57ms\\
		\hline
		upper = 900	& 	0.6ms		  & 0.64ms			 &	0.71ms			& 0.63ms\\
		\hline
		upper = 1400& 	0.58ms		  & 0.61ms			 &	0.6ms			& 0.62ms\\
		\hline
		upper = 1900& 	0.69ms		  & 0.66ms			 &	0.58ms			& 0.68ms\\
		\hline
		upper = 2400& 	0.6ms		  & 0.58ms			 &	0.6ms			& 0.62ms\\
		\hline
	\end{tabular}
	\centering
	\caption{Run times of the \textit{ExecutionTimeConstraint} with $lower = 300$}
	\label{tab:runtimeExecutionTimeConstraint}
\end{table}

\subsubsection{OrderConstraint}
% TODO nicht plausibel, nochmal laufen lassen mit distances > 1
\begin{table}
	\begin{tabular}{|c|c|c|c|}
		\hline
		& Minimum &  Maximum &  Average \\
		\hline
		overall Minimum & ms & ms & ms\\
		\hline
		overall Maximum & ms & ms & ms\\
		\hline
		overall Average & ms & ms & ms\\
		\hline
	\end{tabular}
	\centering
	\caption{Run times of the \textit{OrderConstraint}}
	\label{tab:runtimeOrderConstraint}
\end{table}


\subsubsection{SporadicConstraint}
The traces, that were used for the fulfill the constraint with the parameters $jitter\in\{1,11,21,31\}$, $lower\in\{500,600,...,900\}$ and $upper=lower+x$ $x\in\{100, 200, ..., 500\}$. The run times were between 0.014ms and 40.09ms, with an average of 1.14ms. The average run time per event of the monitor with the parameters $lower=500$, $upper\in\{600,700,...,1000\}$ and $jitter\in\{1,11,21,31\}$ can be seen in table~\ref{tab:runtimeSporadicConstraint}. Like expected by the analysis in the previous section, the parameters had no influence on the run time.

\begin{table}
	\begin{tabular}{|c|c|c|c|c|}
		\hline
		& jitter = 1 & jitter = 11 & jitter = 21 & jitter = 31\\
		\hline
		upper = 600 & 1.05ms & 1.07ms& 1.07ms& 1.18ms\\
		\hline
		upper = 700 & 1.15ms & 1.1ms& 1.24ms& 1.2ms\\
		\hline
		upper = 800 & 1.12ms& 1.12ms& 1.18ms & 1.08ms\\
		\hline
		upper = 900 & 1.07ms& 1.25ms& 1.19ms& 1.17ms\\
		\hline
		upper = 1000 & 1.08ms& 1.13ms& 1.21ms& 1.18ms\\
		\hline
	\end{tabular}
	\centering
	\caption{Run times of the \textit{SporadicConstraint} with $lower = 500$}
	\label{tab:runtimeSporadicConstraint}
\end{table}

\subsubsection{PeriodicConstraint}
The run time evaluation was done on traces, which fulfill the \textit{PeriodicConstraint} with the parameters $period\in\{10,20,30,..,100\}$ and $jitter\in\{0,1,..,9\}$. The run time per event was between 0.013ms and 49ms, with an average of 1.15ms. In table~\ref{tab:runtimePeriodicConstraint} the run time of the monitor with $period = 10$ can be seen. No dependency between the parameters and the run time per event can be observed, which was expected by the complexity analysis.
\begin{table}
	\begin{tabular}{|c|c|c|c|c|c|c|c|c|c|c|}
		\hline
		jitter & 0& 1& 2& 3& 4& 5& 6& 7& 8& 9\\
		\hline
		avg. run time(ms) &1.1& 1.16& 1.1&1.24&1.08&1.08&1.12&1.07&1.11&1.19\\
		\hline
	\end{tabular}
	\centering
	\caption{Run times of the \textit{PeriodicConstraint} with $period = 10$}
	\label{tab:runtimePeriodicConstraint}
\end{table}

\subsubsection{PatternConstraint}
The run time of the \textit{PatterConstraint} monitor was evaluated three times with different lengths of the $offset$ parameter. In the first evaluation, the $offset$ had the length 1 and contained values from 0 to 4. The other parameters were $period\in\{10,20,..., 90\}$ and $jitter\in\{0,2,...,8\}$.
\begin{table}
	\begin{tabular}{|c|c|c|c|c|c|}
		\hline
					& jitter=0 &  jitter=2 &  jitter=4 & jitter=6 & jitter=8 \\
		\hline
		offset = [0] & 1.45ms & 1.66ms & 1.5ms & 1.63ms & 1.43ms\\
		\hline
		offset = [1] & 1.64ms & 1.52ms & 1.45ms & 1.4ms & 1.43ms\\
		\hline
		offset = [2] & 1.59ms & 1.55ms & 1.54ms & 1.47ms & 1.52ms\\
		\hline
		offset = [3] & 1.58ms & 1.48ms & 1.59ms & 1.49ms & 1.46ms\\
		\hline
		offset = [4] & 1.43ms & 1.58ms & 1.39ms & 1.67ms & 1.54ms\\
		\hline
	\end{tabular}
	\centering
	\caption{Run times of the \textit{PatternConstraint} with $period=10$ and $|offset| = 1$}
	\label{tab:runtimePatternConstraint}
\end{table}
The second evaluation was done with $|offset| = 2$, $offset_1\in\{0, 1, ..., 4\}$, $offset_2\in\{1, 2, ..., 9\}$, $period=offset_2+x$, $x\in\{10, 30, ..., 90\}$ and $jitter = min(\frac{period}{10}, offset_2-offset_1)$.
\begin{table}
	\begin{tabular}{|c|c|c|c|c|c|}
		\hline
					& x = 11 & x = 31 & x = 51 & x = 71 & x = 91\\
		\hline
		\makecell{\textit{offset}$=[0,1]$\\$jitter=0$\\$period=x$}& 1.47ms & 1.53ms & 1.42ms & 1.46ms & 1.31ms\\
		\hline
		\makecell{\textit{offset}$=[1,4]$\\$jitter=2$\\$period=x+3$}& 1.57ms & 1.32ms & 1.65ms & 1.35ms & 1.54ms\\
		\hline
		\makecell{\textit{offset}$=[1,4]$\\$jitter=1$\\$period=x+3$}& 1.49ms & 1.71ms & 1.57ms & 1.62ms & 1.43ms\\
		\hline
	\end{tabular}
	\centering
	\caption{average run times of the \textit{PatternConstraint}}
	\label{tab:runtimePatternConstraint2}
\end{table}

The third evaluation was done with $|offset| = 3$, $offset_1\in\{1, 2, 3\}$, $offset_2\in\{2, 3, ..., 6\}$, $offset_3\in\{3, 4, ..., 9\}$, $period=offset_3+x$, $x\in\{10, 30, ..., 90\}$ and $jitter = min(\frac{period}{10}, offset_3-offset_2)$.\\
Some average run times with different parameters can be seen in table~\ref{tab:runtimePatternConstraint}, \ref{tab:runtimePatternConstraint2} and \ref{tab:runtimePatternConstraint3}. The run times were pretty similar and varying from 0.49ms to 96ms with an average of 1.49ms. Similar to the length of the $|offset|$ parameter, the other parameters also did not an observable influence on the performance of the implementation, which matches with the previous analysis of the implementation.
\begin{table}
	\begin{tabular}{|c|c|c|c|c|c|}
		\hline
		& x = 13 & x = 33 & x = 53 & x = 73 & x = 93\\
		\hline
		\makecell{\textit{offset}$=[1,2,3]$\\$jitter=0$\\$period=x$}& 1.52ms & 1.37ms & 1.45ms & 1.42ms & 1.37ms\\
		\hline
		\makecell{\textit{offset}$=[1,3,6]$\\$jitter=1$\\$period=x+3$}& 1.5ms & 1.52ms & 1.46ms & 1.61ms & 1.52ms\\
		\hline
		\makecell{\textit{offset}$=[3,4,6]$\\$jitter=0$\\$period=x+3$}& 1.64ms & 1.47ms & 1.4ms & 1.42ms & 1.41ms\\
		\hline
	\end{tabular}
	\centering
	\caption{average run times of the \textit{PatternConstraint}}
	\label{tab:runtimePatternConstraint3}
\end{table}


\subsubsection{ArbitraryConstraint}
The \textit{ArbitraryConstraint} was evaluated three times with different lengths of the $minimum$ and $maximum$ parameters. In the first run, the length of these parameters were 1 they had the values $minimum_1\in\{10, 20, ..., 100\}$ and $maximum_1=minimum_1+x$, $x\in\{0, 10, ..., 90\}$.
\begin{table}
	\begin{tabular}{|c|c|c|c|c|c|}
		\hline
		&\makecell{maximum\\=[60]} & \makecell{maximum\\=[70]} &  \makecell{maximum\\=[80]} & \makecell{maximum\\=[90]} & \makecell{maximum\\=[100]}\\
		\hline
		minimum=[10] & 0.67ms & 0.73ms & 0.6ms & 0.66ms & 0.61ms\\
		\hline
		minimum=[30] & 0.61ms & 0.57ms & 0.64ms & 0.64ms & 0.74ms\\
		\hline
		minimum=[50] & 0.72ms & 0.79ms & 0.62ms & 0.66ms & 0.64ms\\
		\hline
	\end{tabular}
	\centering
	\caption{Run times of the \textit{ArbitraryConstraint} with $|minimum|=|maximum|=1$}
	\label{tab:runtimeArbitraryConstraint1}
\end{table}
In the second evaluation, the $minimum$ and $maximum$ had the length 2 and had the values $minimum_1\in\{10, 20, ..., 100\}$, $minimum_2=2*minimum_1$, $maximum_1=minimum_1+x$, $x\in\{0, 10, ..., 90\}$ and $maximum_2=2*maximum_1$.
\begin{table}
	\begin{tabular}{|c|c|c|c|c|c|}
		\hline
						&\makecell{maximum\\=[60,120]} & \makecell{maximum\\=[70,140]} &  \makecell{maximum\\=[80,160]} & \makecell{maximum\\=[90,120]} &  \makecell{maximum\\=[100,200]}\\
		\hline
		minimum=[10,20] & 0.99ms & 0.92ms & 0.97ms & 0.96ms & 0.87ms\\
		\hline
		minimum=[30,60] & 0.78ms & 0.98ms & 0.88ms & 0.97ms & 0.84ms\\
		\hline
		minimum=[50,100] & 0.93ms & 0.86ms & 0.78ms & 0.86ms & 1.12ms\\
		\hline
	\end{tabular}
	\centering
	\caption{Run times of the \textit{ArbitraryConstraint} with $|minimum|=|maximum|=2$}
	\label{tab:runtimeArbitraryConstraint2}
\end{table}
In the third evaluation, the $minimum$ and $maximum$ had the length 3 and had the values $minimum_1\in\{10, 20, ..., 100\}$, $minimum_2=2*minimum_1$, $minimum_3=3*minimum_1$, $maximum_1=minimum_1+x$, $x\in\{0, 10, ..., 90\}$, $maximum_2=2*maximum_1$ and $maximum_3=3*maximum_1$. Some average run times with different parameters can be seen in table~\ref{tab:runtimeArbitraryConstraint1}, \ref{tab:runtimeArbitraryConstraint2} and \ref{tab:runtimeArbitraryConstraint3}. Inside the individual evaluation, no relation between the parameters and the run times could be observed. In between the three evaluations, it can be seen that the run time increases the longer the $minimum$ and $maximum$ parameters get.
\begin{table}
	\begin{tabular}{|c|c|c|c|c|c|}
		\hline
					&\makecell{maximum\\=[60,120,180]} & \makecell{maximum\\=[70,140,210]} &  \makecell{maximum\\=[80,160,210]} &  \makecell{maximum\\=[90,180,270]} &  \makecell{maximum\\=[100,200,300]}\\
		\hline
		\makecell{minimum\\=[10,20,30]} & 1.01ms & 1.01ms & 1.12ms & 1.06ms & 1.13ms\\
		\hline
		\makecell{minimum\\=[30,60,90]} & 0.99ms & 1.16ms & 0.91ms & 0.99ms & 1.07ms\\
		\hline
		\makecell{minimum\\=[50,100,150]} & 1.02ms & 1.02ms & 1.17ms & 1.06ms & 1.14ms\\
		\hline
	\end{tabular}
	\centering
	\caption{Run times of the \textit{ArbitraryConstraint} with $|minimum|=|maximum|=3$}
	\label{tab:runtimeArbitraryConstraint3}
\end{table}

\subsubsection{BurstConstraint}
%TODO nicht plausibel-> repeatConstraint
\begin{table}
	\begin{tabular}{|c|c|c|c|}
		\hline
		& Minimum &  Maximum &  Average \\
		\hline
		overall Minimum & ms & ms & ms\\
		\hline
		overall Maximum & ms & ms & ms\\
		\hline
		overall Average & ms & ms & ms\\
		\hline
	\end{tabular}
	\centering
	\caption{Run times of the \textit{BurstConstraint}}
	\label{tab:runtimeBurstConstraint}
\end{table}

\subsubsection{ReactionConstraint}
The runtime evaluation of the \textit{ReactionConstraint} was done on traces with the parameters $minimum\in\{100,200,...,1000\}$ and $maximum = minimum$, while the distances between subsequent $stimulus$ event were in $\{1, 2, 4, 8, ..., 1024\}$, so that $minimum$, $\lceil\frac{minimum}{2}\rceil$,  $\lceil\frac{minimum}{4}\rceil$, ...,  $\lceil\frac{minimum}{1024}\rceil$ events must be stored an considered at every event in the monitor. The runtime per event was between 0.22ms and 51.74ms, with an average of 2.27ms. In table~\ref{tab:runtimeReactionConstraint}, some average run times can be seen.
The runtimes for traces with short $stimulus$ event distances and a larger $maximum$ value were greater than for traces with longer $stimulus$ event distances. This was expected, because with long $stimulus$ and a short $maximum$ value, less events must be stored and considered.
\begin{table}
	\begin{tabular}{|c|c|c|c|c|}
		\hline
					 & $dist=2^0$ & $dist=2^1$ & $dist=2^2$ & $dist=2^3$  \\
	 	\hline
	 	maximum=100 & 2.93ms & 2.5ms & 2.16ms & 1.26ms\\
	 	\hline
	 	maximum=200 & 4.41ms & 3.21ms & 2.41ms & 1.92ms\\
	 	\hline
		maximum=300 & 4.58ms & 4.18ms & 2.67ms & 1.69ms\\
		\hline
		\hline
					 & $dist=2^4$ & $dist=2^5$ &$dist=2^6$ & $dist=2^7$\\
		\hline
		maximum=100 & 1.22ms & 1.05ms & 1.01ms & 0.96ms\\
		\hline
		maximum=200 & 1.28ms & 1.16ms & 1.05ms & 1.1ms\\
		\hline
		maximum=300 & 1.47ms & 1.24ms & 1.08ms & 1.05ms\\
		\hline
	\end{tabular}
	\centering
	\caption{Run times of the \textit{ReactionConstraint}}
	\label{tab:runtimeReactionConstraint}
\end{table}

\subsubsection{AgeConstraint}
The run time of the \textit{AgeConstraint} monitor were measured on traces with the parameters $minimum = maximum\in\{100,200,...,1000\}$. The distance between subsequent \textit{stimulus} events were between 1 and $2*maximum$ timestamps, therefore between $maximum$ and one event were stored and considered in the monitoring steps. The run time per event were significantly higher, as smaller the distances between the $stimulus$ events and the greater the $maximum$ parameter were. This matches with the expectations based on the analysis in the previous section, because the smaller the distances and the greater $maximum$, the more events are stored and processed in each monitoring step.
\begin{table}
		\begin{tabular}{|c|c|c|c|c|}
		\hline
		& $dist=1$ & $dist=3$ & $dist=6$ & $dist=12$  \\
		\hline
		maximum=100 & 39.55ms & 11.52ms & 5.18ms & 3.15ms\\
		\hline
		maximum=200 & 79.61ms & 25.32ms & 11.02ms & 4.96ms\\
		\hline
		maximum=400 & 159.71ms & 55.32ms & 24.71ms & 11ms\\
		\hline
		\hline
		& $dist=25$ & $dist=50$ &$dist=100$ & $dist=200$\\
		\hline
		maximum=100 & 2.76ms & 2.34ms & 2.37ms & 1.42ms\\
		\hline
		maximum=200 & 3.95ms & 2.94ms & 2.48ms & 2.37ms\\
		\hline
		maximum=400 & 5.8ms & 3.8ms & 3.11ms & 2.48ms\\
		\hline
	\end{tabular}
	\centering
	\caption{Run times of the \textit{AgeConstraint}}
	\label{tab:runtimeAgeConstraint}
\end{table}

\subsubsection{OutputSynchronizationConstraint}
%TODO nochmal laufen lassen, clusterDistance > 1
The runtime evaluation of the \textit{OutputSynchronizationConstraint} was done on traces with 2 to 5 stimulus streams, which fulfilled the constraint with the parameter $tolerance\in\{10,13,.., 25\}$. The distances between synchronization clusters were in $\{1,2,4,8, 16\}$. 
\begin{table}
	\begin{tabular}{|c|c|c|c|}
		\hline
						& Minimum &  Maximum &  Average \\
		\hline
		overall Minimum & 0.019ms & 180.29ms & 60.28ms\\
		\hline
		overall Maximum & 9.14ms & 869.14ms & 291.78ms\\
		\hline
		overall Average & 3.45ms & 399.31ms & 134.76ms\\
		\hline
	\end{tabular}
	\centering
	\caption{Run times of the \textit{OutputSynchronizationConstraint}}
	\label{tab:runtimeOutputSynchronizationConstraint}
\end{table}

\subsubsection{InputSynchronizationConstraint}
%TODO nochmal laufen lassen, clusterDistance > 1
The traces for the evaluation of the \textit{InputSynchronizationConstraint} were generated with the same parameters as the ones for the \textit{OutputSynchronizationConstraint}. The runtime per event was between 0.016ms and 72.3ms, with an average of 5ms. Monitoring traces with smaller cluster distances were slower, %TODO why?
but not as significant as in the monitor of the \textit{OutputSynchronizationConstraint}.%TODO stimmt?
\begin{table}
	\begin{tabular}{|c|c|c|c|}
		\hline
		& Minimum &  Maximum &  Average \\
		\hline
		overall Minimum & 0.016ms & 38.25ms & 2.32ms\\
		\hline
		overall Maximum & 3.04ms & 72.3ms & 8.41ms\\
		\hline
		overall Average & 0.65ms & 53.67ms & 5ms\\
		\hline
	\end{tabular}
	\centering
	\caption{Run times of the \textit{InputSynchronizationConstraint}}
	\label{tab:runtimeInputSynchronizationConstraint}
\end{table}