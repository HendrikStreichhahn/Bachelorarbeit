%!TEX root = thesis.tex

\chapter{Grundlagen}
\label{chapter-basics}

\section{TeSSLa}

TeSSLa (\textbf{Te}mporal \textbf{S}tream-based \textbf{S}pecification \textbf{La}nguage) ist eine funktionale Programmiersprache, die für die Laufzeitverifikation von Datenströmen konzipiert wurde. In TeSSLa sind Ströme als Folgen von Events definiert, wobei ein Event aus einem Element der jeweiligen Datentypmenge $\mathbb D$ sowie aus einem Zeitwert aus der diskreten Zeitdomäne $\mathbb T$ besteht. In dieser muss es eine totale Ordnung geben und ein Event $b$, welches zeitlich nach einem Event $a$ auftritt, muss einen höheren Zeitwert aufweisen. Innerhalb einer Spezifikation können mehrere Ströme aus unterschiedlichen Datentypmengen $\mathbb D_1, \dots, \mathbb D_n$ verwendet werden, wobei die Zeitdomäne $\mathbb T$ innerhalb einer Spezifikation auf allen Strömen dieselbe sein muss.\\
In TeSSLa wird zwischen synchronen Strömen, in denen alle Ströme einer Spezifikation Events in gemeinsamen Zeitpunkten haben, und asynchronen Strömen unterschieden, bei denen die Zeitpunkte der Events zwar einer globalen Ordnung folgen, die Zeitpunkte aber sonst unanhängig von einander sind.Die Spezifikationen mit synchronen Strömen sind eine echte Teilmenge der Spezifikationen mit asynchronen Strömen, da Ströme mit geordneten, aber ansonsten unabhängigen Zeitpunkten Events mit gleichzeitigen Zeitpunkten auf allen Strömen zulassen. Andersherum gilt diese Relation offensichtlich nicht. Aufgrund dieser Teilmengenrelation werden im Folgenden nur asynchrone Ströme behandelt, wenn von Strömen die Rede ist, sind immer asynchrone gemeint.\\
Die Berechnungen erfolgen, nachdem sie von eintreffenden Events gestartet wurden, wodurch ein Ausgabestrom mit den gleichen Zeitwerten wie Eingabeströme, allerdings kann der \emph{delay}-Operator verwendet werden, um neue Zeitpunkte zu erzeugen, dazu später mehr. Ohne neue Zeitpunkte heißt die Spezifikation \emph{timestamp conservative}. Innerhalb dieser Berechnungen sind Direktzugriffe nur auf die aktuellen Datenwerte der Ströme möglich. Diese Werte bleiben solange bestehen, bis ein neues Event auf diesem Strom eintrifft, der Zeitwert des Events ändert sich hierbei nicht. Mit dem \emph{last}-Operator, welcher auch rekursiv angewendet werden kann, sind Zugriffe auf das jeweils letzte Element möglich. Der \emph{lift}-Operator wendet eine Funktion über Datenwerten auf die Datenwerte jedes eintreffenden Events an. Der \emph{slift}-Operator agiert ähnlich dem \emph{lift}-Operator, allerdings wird die Funktion erst dann angewendet, wenn auf jedem Strom, der dem \emph{slift}-Operator übergeben wurde, bereits ein Event übertragen wurde. (TODO$\rightarrow$ weiter ausführen)\\
In ~\cite{TeSSLa} werden verschiedene Fragmente von TeSSLa beschrieben, die unterschiedliche Mächtigkeiten haben und äquivalent zu verschiedenen Transduktormodellen sind. Im Fragment \emph{TeSSLa$_{bool}$} sind die Datentypmengen der Ströme auf boolesche Werte beschränkt, als Operatoren sind nur der oben genannte \emph{last}-Operator, der \emph{lift}-Operator