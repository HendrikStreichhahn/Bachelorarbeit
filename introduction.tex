%!TEX root = thesis.tex

\chapter{Introduction}

	Timing behavior is one of the most important properties of computer systems. Especially in safety-critical applications, wrong timed actions or reactions of the system can have disastrous consequences, for example, in the Electronic Stability Control of a vehicle. Almost all car manufacturers use the \emph{AUTOSAR} (\textbf{AUT}omotive \textbf{O}pen \textbf{S}ystem \textbf{AR}chitecture) standards \cite{AUTOSARpartner}. With AUTOSAR, development processes and components are standardized, which increases productivity, interoperability and exchangeability of these components.\\
	To describe the timing behavior of soft- and hardware components of cars, the \emph{AUTOSAR Timing Extensions} were developed. The goal of this thesis is to implement a monitoring tool for the timing constraints defined in this standard.\\
	Some of the constraints defined in the \emph{AUTOSAR} standard are written informally and can be misunderstood, which will be described as part of this thesis. This is problematic for monitoring because the implementation of a monitor must not be based on ambiguous definitions. To solve this problem, the timing constraints defined in the \textbf{T}iming \textbf{A}ugmented \textbf{D}escription \textbf{L}anguage Version \textbf{2} (\textit{TADL2})\cite{TIMMO2USE} are used as the basis for the monitoring tool. \emph{TADL2} was created as part of the TIMMO project, which had similar goals to AUTOSAR, but the definitions are written more formally. The AUTOSAR Timing Extensions are comparable and partly compatible with the TADL2 timing constraints. Most of the constraints defined in the AUTOSAR standard can be described as an equivalent combination of TADL2 timing constraints and vice versa.\\
	The monitoring tool is written in \emph{TeSSLa} (\textbf{T}emporal \textbf{S}tream-based \textbf{S}pecification \textbf{L}anguage), which is made for stream runtime verification and is capable of non-intrusive observation and can be run as a Java program or on specialized embedded hardware, like FPGAs.

	In the first part of this thesis, an overview of the AUTOSAR Timing Extensions and an example of the informal and ambiguous definitions  will be given. Next, the TADL2 timing constraints will be listed and the relations between these constraints and the AUTOSAR Timing Extensions will be described.
	In the next chapter, TeSSLa, its fundamental functionality and other prerequisites needed to understand the theoretical part of this thesis will be explained.
	The term \emph{simple monitorability} is introduced, which ensures that a property on infinite streams can always be monitored with finite time and memory resources.
	Then, each of the TADL2 timing constraints is checked, if it \textit{simple monitorable} or not. After that, the TeSSLa implementations of these constraints are described and evaluated in a theoretical and practical way.\\
	In the end, an overview of the accomplished work is given and ideas for further work will be discussed.