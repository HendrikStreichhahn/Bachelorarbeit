%!TEX root = thesis.tex

\chapter{Introduction}

	Timing behavior is one of the most important properties of computer systems. Especially in safety-critical applications, a wrong timed reaction of the system can have disastrous consequences, for example in the Electronic Stability Control of a vehicle. The \emph{AUTOSAR} (\textbf{AUT}omotive \textbf{O}pen \textbf{S}ystem \textbf{AR}chitecture) standards are used by almost all car manufacturers in their software development processes to standardize components and therefore increase the interoperability and exchangeability.\\
	To describe the timing behavior of soft- and hardware components of cars, the \emph{AUTOSAR Timing Extensions} were developed. The goal of this thesis is to implement a monitoring tool for the timing constraints defined in this standard.\\
	Some of the constraints defined in the \emph{AUTOSAR} standard are written in an informal way and can be misunderstood, which will be describe as part of this thesis. This is problematic for monitoring, because the implementation of a monitor should be based on unambiguous definitions. To solve this problem, the timing constraints defined in \emph{TADL2} (\textbf{T}iming \textbf{A}ugmented \textbf{D}escription \textbf{L}anguage Version \textbf{2}) are used as basis for the monitoring tool. The TADL2 timing constraints are comparable and partly compatible to the AUTOSAR Timing Extensions, as most of the constraints defined in the AUTOSAR standard can be described as equivalent combination of TADL2 timing constraints.\\
	The monitoring tool is written in \emph{TeSSLa} (\textbf{T}emporal \textbf{S}tream-based \textbf{S}pecification \textbf{L}anguage), which is made for stream runtime verification and is capable of non-intrusive observation and can be run as Java program or on specialized embedded hardware, like FPGAs.

	In the first part of this thesis, an overview over the AUTOSAR Timing Extensions and an example about the informal and ambiguous definitions  will be given. Next, the TADL2 timing constraints will be listed and the relations between the these constraints and the AUTOSAR Timing Extensions will be described.
	In the next chapter, TeSSLa, its fundamental functionality and other prerequisites, which are needed for understanding the theoretical part of this thesis, will be explained.
	The term of \emph{finite monitorability} is introduced, which insures, that a property on infinite streams can always be monitored with finite resources.
	Then, each of the TADL2 timing constraint is checked, if it finite monitorable or not. After that, the TeSSLa implementations of these constraints is described and evaluated in a theoretical and practical way.\\
	In the end an overview of the accomplished is given and ideas for further work will be discussed.