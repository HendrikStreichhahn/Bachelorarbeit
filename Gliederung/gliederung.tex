\documentclass[]{article}
\begin{document}
\begin{itemize}
	\item
		Abstract
	\item
		Introduction
		\begin{itemize}
			\item
				problem
			\item
				solution approach
			\item
				related work
			\item
				structure of thesis
		\end{itemize}
	\item
		Timing Constraints
		\begin{itemize}
			\item
				AUTOSAR Timing Extensions
				\begin{itemize}
					\item
						goals and basics
					\item
						problems with definitions AUTOSAR Timing Extensions
						\begin{itemize}
							\item
								not formal enough
							\item
								not reasonable monitorable
						\end{itemize}
				\end{itemize}
			\item
				TIMMO-2-USE
				\begin{itemize}
					\item
						goals and basics
					\item
						explanatory parenthesis
						\begin{itemize}
							\item	
								A Simple and Flexible Timing Constraint Logic
						\end{itemize}
					\item
						overview of constraints
				\end{itemize}
			\item
				comparison TIMMO-2-USE Constraints $\leftrightarrow$ AUTOSAR Timing Extensions
				\begin{itemize}
					\item
						which are missing
					\item
						subsets, supersets
				\end{itemize}
		\end{itemize}
	\item
		monitorability models
		\begin{itemize}
			\item
				TeSSLa basics
				\begin{itemize}
					\item
						definitions of streams
					\item
						possibility of non-intrusive online monitoring
				\end{itemize}
			\item
				finite transducers
			\item
				timed moore automata
			\item
				finite timed transducers
				\begin{itemize}
					\item
						$\epsilon$-extension
				\end{itemize}
		\end{itemize}
	\item
		monitorability
		\begin{itemize}
			\item
				problem of indiscrete time stamps
			\item
				finite monitorability
			\item
				finite monitorability with single delay
			\item
				non-finite monitorability
				\begin{itemize}
					\item
						infinite memory requirements in worst case scenario
					\item
						continous growth of memory requirements in ''all'' cases
				\end{itemize}
		\end{itemize}
	\item
		TIMMO-2-USE II
		\begin{itemize}
			\item
				memory usage
			\item
				monitorability of each constraint
		\end{itemize}
	\item
		implementation
		\begin{itemize}
			\item
				evaluation
			\begin{itemize}
				\item
					theoretical
					\begin{itemize}
						 \item
						 	O-notation for each in constraint per timestamp with events (memory and time)
					\end{itemize}
				\item
					practical
					\begin{itemize}
						\item
							performance in scala-interpreter using large random generated traces
					\end{itemize}
			\end{itemize}
		\end{itemize}
	\item
		summary and outlook
\end{itemize}
\end{document}

\begin{document}

\begin{itemize}
	\item
		Abstract
	\item
		Einleitung
		\begin{itemize}
			\item
				Problemstellung
			\item
				AUTOSAR TIMEX Constraints
				\begin{itemize}
					\item
						Ziele und Grundlagen
					\item
						Probleme mit AUTOSAR und dessen Definitionen
						\begin{itemize}
							\item
							Ungenauigkeiten
							\item
							Nicht sinnvoll monitorbar
						\end{itemize}
					\item
						Timmo2Use
						\begin{itemize}
							\item
								Ziele und Grundlagen
						\end{itemize}
				\end{itemize}
			\item
				verwandte Arbeiten
			\item
				Aufbau der Arbeit
		\end{itemize}
	\item
		Grundlagen
		\begin{itemize}
			\item
				TeSSLa-Grundlagen
				\begin{itemize}
					\item
					Def. von Strömen u.s.w.
				\end{itemize}
			\item
				endliche Transducer
			\item
				Timed Moore Automata
			\item
				endliche Timed Transducer
				\begin{itemize}
					\item
						mit $\varepsilon$-Erweiterung
				\end{itemize}
		\end{itemize}
	\item
		Monitorbarkeit
		\begin{itemize}
			\item
				einfach monitorbar
			\item
				einfach monitorbar mit Delay
			\item
				Probleme mit indiskreten Zeitstempeln
		\end{itemize}
	\item
		Timmo2Use Constraints
		\begin{itemize}
			\item
				Auflistung der Timmo2Use-Constraints
				\begin{itemize}
					\item
						Beschreibung
					\item
						benötigter Speicherbedarf
					\item
						Klassifizierung Monitorbarkeit
				\end{itemize}
			\item
				Vergleich mit AUTOSAR TIMEX Constraints
				\begin{itemize}
					\item
						welche fehlen
					\item
						welche sind Teilmengen, welche Obermengen 
				\end{itemize}
		\end{itemize}
	\item
		Beschreibung/Dokumentation der Implementierung
		\begin{itemize}
			\item
				Evaluation
				\begin{itemize}
					\item
						theoretisch
					\item
						praktisch, anhand von zufallsgenerierten Traces
						\begin{itemize}
							\item
								Testfälle
							\item
								Ressourcenverbrauch
						\end{itemize}
				\end{itemize}
		\end{itemize}
	\item
		Zusammenfassung und Ausblick
\end{itemize}

\end{document}
