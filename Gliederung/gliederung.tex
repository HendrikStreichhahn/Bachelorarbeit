\documentclass[]{article}
\begin{document}

\begin{itemize}
	\item
		Abstract
	\item
		Einleitung
		\begin{itemize}
			\item
				verwandte Arbeiten
			\item
				Aufbau der Arbeit
		\end{itemize}
	\item
		Grundlagen
		\begin{itemize}
			\item
				TeSSLa-Grundlagen
				\begin{itemize}
					\item
					Def. von Strömen u.s.w.
				\end{itemize}
			\item
				endliche Transducer
			\item
				Timed Moore Automata
			\item
				endliche Timed Transducer
				\begin{itemize}
					\item
						mit $\varepsilon$-Erweiterung
				\end{itemize}
		\end{itemize}
	\item
		Monitorbarkeit
		\begin{itemize}
			\item
				einfach monitorbar
			\item
				einfach monitorbar mit Delay
			\item
				Probleme mit indiskreten Zeitstempeln
		\end{itemize}
	\item
		AUTOSAR TIMEX Constraints
		\begin{itemize}
			\item
				Ziele und Grundlagen
			\item
				Probleme mit AUTOSAR und dessen Definitionen
				\begin{itemize}
					\item
						Ungenauigkeiten
					\item
						Nicht sinnvoll monitorbar
				\end{itemize}
			\item
				Timmo2Use
				\begin{itemize}
					\item
						Ziele und Grundlagen
					\item
						Auflistung der Timmo2Use-Constraints
						\begin{itemize}
							\item
								Beschreibung
							\item
								benötigter Speicherbedarf
							\item
								Laufzeit pro Zeitstempel (?)
							\item
								Klassifizierung Monitorbarkeit
						\end{itemize}
					\item
						Vergleich mit AUTOSAR TIMEX Constraints
						\begin{itemize}
							\item
								welche fehlen
							\item
								welche sind Teilmengen, welche Obermengen 
						\end{itemize}
				\end{itemize}
		\end{itemize}
	\item
		Beschreibung/Dokumentation der Implementierung
		\begin{itemize}
			\item
				Evaluation
				\begin{itemize}
					\item
						theoretisch
					\item
						praktisch, anhand von zufallsgenerierten Traces
						\begin{itemize}
							\item
								Testfälle
							\item
								Ressourcenverbrauch
						\end{itemize}
				\end{itemize}
		\end{itemize}
	\item
		Zusammenfassung und Ausblick
\end{itemize}

\end{document}
