\documentclass{scrbook}

%!TEX root = thesis.tex

% Set german to default language and load english as well
\usepackage[english]{babel}

% Set UTF8 as input encoding
\usepackage[utf8]{inputenc}

% Set T1 as font encoding
\usepackage[T1]{fontenc}
% Load a slightly more modern font
\usepackage{lmodern}
% Use the symbol collection textcomp, which is needed by listings.
\usepackage{textcomp}
% Load a better font for monospace.
\usepackage{courier}
% Load package to configure header and footer
\usepackage{scrlayer-scrpage}

% Set some options regarding the document layout. See KOMA guide
\KOMAoptions{%
  paper=a4,
  fontsize=10.95pt,
  parskip=half,
  headings=normal,
  BCOR=1.5cm,
  headsepline,
  headsepline=0.5pt,
  DIV=15}

% do not align bottom of pages
\raggedbottom

% set style of captions
\setcapindent{0pt} % do not indent second line of captions
\setkomafont{caption}{\small}
\setkomafont{captionlabel}{\bfseries}
\setcapwidth[c]{0.9\textwidth}

% set the style of the bibliography
\bibliographystyle{alphadin}

% load extended tabulars used in the list of abbreviation
\usepackage{tabularx}

% load the color package and enable colored tables
\usepackage[table]{xcolor}

% define new environment for zebra tables
\newcommand{\mainrowcolors}{\rowcolors{1}{maincolor!25}{maincolor!5}}
\newenvironment{zebratabular}{\mainrowcolors\begin{tabular}}{\end{tabular}}
\newcommand{\setrownumber}[1]{\global\rownum#1\relax}
\newcommand{\headerrow}{\rowcolor{maincolor!50}\setrownumber1}

% add main color to section headers
\addtokomafont{chapter}{\color{maincolor}}
\addtokomafont{section}{\color{maincolor}}
\addtokomafont{subsection}{\color{maincolor}}
\addtokomafont{subsubsection}{\color{maincolor}}
\addtokomafont{paragraph}{\color{maincolor}}

% do not print numbers next to each formula
\usepackage{mathtools}
\mathtoolsset{showonlyrefs}
% left align formulas
\makeatletter
\@fleqntrue\let\mathindent\@mathmargin \@mathmargin=\leftmargini
\makeatother

% Allow page breaks in align environments
\allowdisplaybreaks

% header and footer
\pagestyle{scrheadings}
\setkomafont{pagenumber}{\normalfont\sffamily\color{maincolor}}
\setkomafont{pageheadfoot}{\normalfont\sffamily}
\setkomafont{headsepline}{\color{maincolor}}

% German guillemets quotes
%\usepackage[german=guillemets]{csquotes}

% load TikZ to draw diagrams
\usepackage{tikz}

% load additional libraries for TikZ
\usetikzlibrary{%
  automata,%
  positioning,%
}

% set some default options for TikZ -- in this case for automata
\tikzset{
  every state/.style={
    draw=maincolor,
    thick,
    fill=maincolor!18,
    minimum size=0pt
  }
}

% load listings package to typeset sourcecode
\usepackage{listings}

% set some options for the listings package
\lstset{%
  upquote=true,%
  showstringspaces=false,%
  captionpos=b,%
  basicstyle=\ttfamily,%
  keywordstyle=\color{keywordcolor}\slshape,%
  commentstyle=\color{commentcolor}\itshape,%
  stringstyle=\color{stringcolor}}
\renewcommand{\lstlistingname}{Quelltext}
\renewcommand{\lstlistlistingname}{Quelltextverzeichnis}

% enable german umlauts in listings
\lstset{
  literate={ö}{{\"o}}1
           {Ö}{{\"O}}1
           {ä}{{\"a}}1
           {Ä}{{\"A}}1
           {ü}{{\"u}}1
           {Ü}{{\"U}}1
           {ß}{{\ss}}1
}

% define style for pseudo code
\lstdefinestyle{pseudo}{language={},%
  basicstyle=\normalfont,%
  morecomment=[l]{//},%
  morekeywords={for,to,while,do,if,then,else},%
  mathescape=true,%
  columns=fullflexible}

% load the AMS math library to typeset formulas
\usepackage{amsmath}
\usepackage{amsthm}
\usepackage{thmtools}
\usepackage{amssymb}

% load the paralist library to use compactitem and compactenum environment
\usepackage{paralist}

% load varioref and hyperref to create nicer references using vref
%\usepackage[ngerman]{varioref}
\PassOptionsToPackage{hyphens}{url} % allow line break at hyphens in URLs
\usepackage{hyperref}

% setup hyperref
\hypersetup{breaklinks=true,
            pdfborder={0 0 0},
            %ngerman,
            pdfhighlight={/N},
            pdfdisplaydoctitle=true}

% Fix bugs in some package, e.g. listings and hyperref
\usepackage{scrhack}

% Allow todos
\usepackage{todonotes}

% define german names for referenced elements
% (vref automatically inserts these names in front of the references)
%\labelformat{figure}{Abbildung\ #1}
%\labelformat{table}{Tabelle\ #1}
%\labelformat{appendix}{Anhang\ #1}
%\labelformat{chapter}{Kapitel\ #1}
%\labelformat{section}{Abschnitt\ #1}
%\labelformat{subsection}{Unterabschnitt\ #1}
%\labelformat{subsubsection}{Unterunterabschnitt\ #1}
%\AtBeginDocument{\labelformat{lstlisting}{Quelltext\ #1}}

% define theorem environments
%\declaretheorem[numberwithin=chapter,style=plain]{Theorem}
%\labelformat{Theorem}{Theorem\ #1}

%\declaretheorem[sibling=Theorem,style=plain]{Lemma}
%\labelformat{Lemma}{Lemma\ #1}

%\declaretheorem[sibling=Theorem,style=plain]{Korollar}
%\labelformat{Korollar}{Korollar\ #1}

%\declaretheorem[sibling=Theorem,style=definition]{Definition}
%\labelformat{Definition}{Definition\ #1}

%\declaretheorem[sibling=Theorem,style=definition]{Beispiel}
%\labelformat{Beispiel}{Beispiel\ #1}

%\declaretheorem[sibling=Theorem,style=definition]{Bemerkung}
%\labelformat{Bemerkung}{Bemerkung\ #1}

%!TEX root = thesis.tex

% Use this file to define some macros you need in your thesis. A macro is a short command that inserts some mathematical symbols or texts you do not want to retype each time you need some. I recommend to use as many macros as possible, because you are able to change them later. For example if you use the same macro each time you need to give the formal semantics of an expression you can easily change the appearance of these brackets by updating the macro later on.

% Set of natural numbers
\newcommand{\N}{\mathbb{N}}

% The default epsilon does not look very nice
\let\epsilon\varepsilon

% If you need to use mathematical expressins like an epsilon in the section titles of your thesis you will end up with warnings that these special symbols cannot be included in the PDF favorites. The following macro uses the mathematical symbol during the text of the thesis and the string "Epsilon" in the PDF favorites.
\newcommand{\pdfepsilon}{\texorpdfstring{$\epsilon$}{Epsilon}}


\usepackage{makecell}

\usepackage{amssymb}

\usepackage{pdflscape}


% Set title and author used in the PDF meta data
\hypersetup{
  pdftitle={Monitoring der Autosar Timing Extensions mittesl TeSSLa},
  pdfauthor={Hendrik Streichhahn}
}

% Depending on which of the following two color schemes you import your thesis will be in color or grayscale. I recommend to generate a colored version as a PDF and a grayscale version for printing.

%%!TEX root = thesis.tex

% define color of example university
\xdefinecolor{exampleuniversity}{rgb}{1, 0.5, 0}

\colorlet{maincolor}{exampleuniversity}

\colorlet{stringcolor}{green!60!black}
\colorlet{commentcolor}{black!50}
\colorlet{keywordcolor}{maincolor!80!black}

\newcommand{\imagesuffix}{-color}
%!TEX root = thesis.tex

\colorlet{maincolor}{black}

\colorlet{stringcolor}{black}
\colorlet{commentcolor}{black!50}
\colorlet{keywordcolor}{black}

\newcommand{\imagesuffix}{-gray}

\newcommand{\duedate}{1.1. 1970}

\begin{document}
  \frontmatter
  %!TEX root = thesis.tex

\begin{titlepage}
  \thispagestyle{empty}

  \vskip1cm

  \pgfimage[height=2.5cm]{uni-logo-example\imagesuffix}
  
  \vskip2.5cm
  
  \LARGE
  
  \textbf{\sffamily\color{maincolor}Monitoring der AUTOSAR Timing Extensions mittels TeSSLa}

  \textit{Monitoring of the AUTOSAR Timing Extensions with TeSSLa}

  \normalfont\normalsize

  \vskip2em
  
  \textbf{\sffamily\color{maincolor}Bachelorarbeit}

  im Rahmen des Studiengangs \\
  \textbf{\sffamily\color{maincolor}Informatik} \\
  der Universität zu Lübeck

  \vskip1em

  vorgelegt von \\
  \textbf{\sffamily\color{maincolor}Hendrik Streichhahn}

  \vskip1em
  
  ausgegeben und betreut von \\
  \textbf{\sffamily\color{maincolor}Prof. Dr. Martin Leucker}

  \vskip1em

  mit Unterstützung von\\
  Martin Sachenbacher und\\
  Daniel Thoma

  \vskip1em


  \vfill

  Lübeck, den \duedate
\end{titlepage}

  %!TEX root = thesis.tex

\cleardoublepage
\thispagestyle{plain}
\vspace*{\fill}

\section*{Erklärung}

Ich erkläre hiermit an Eides statt, dass ich diese Arbeit selbständig verfasst und keine
anderen als die angegebenen Quellen und Hilfsmittel benutzt habe.

\vskip2cm

\rule{5cm}{0.4pt}\\
(Hendrik Streichhahn)\\
Lübeck, den \duedate

  %!TEX root = thesis.tex

\cleardoublepage
\thispagestyle{plain}

\pdfbookmark{Abstract}{abstract}
\paragraph{Abstract}
Satisfying given timing requirements is essential for the correct behavior of embedded real-time systems.
In the automotive domain, the AUTOSAR timing extensions are a recent and widely accepted standard for specifying timing requirements.
Previous work, such as the TIMMO-2-USE project, has focused on formalizing the AUTOSAR timing model and timing extensions in a mathematically rigorous way, in order to make them amenable for off-line system analysis tools such as automated model-checking and verification.\\
Because of computational problems, model-checking and offline verification is limited to relatively small-scale systems. Furthermore, not all types of specification violations can be detected at system development time, and sporadic, rare events typically require a capability for long-term observations.
Run-time verification is a more lightweight method that lies at the boundary between formal verification and testing. Run-time verification checks properties, expressed in temporal logic, on-the-fly during the operation of the system using finite-state monitors generated from the logical specifications. 
In this thesis, an analysis of the 18 TADL2 timing constraints defined in the TIMMO-2-USE project is made to decide, whether they can be expressed as finite-state monitors, thus making them monitorable by runtime verification. Further, a monitor for each of the TADL2 timing constraint is implemented in the temporal stream-based specification language TeSSLa.

\cleardoublepage
\thispagestyle{plain}

\foreignlanguage{german}{%
\pdfbookmark{Kurzfassung}{abstract}
\paragraph{Kurzfassung} 
	Die Einhaltung von Zeitschranken ist essentiell wichtig für das korrekte Verhalten von eingebetteten Echtzeitsystemen.
	In der Automobilindustrie werden in breiter Masse die AUTOSAR Timing Extensions (etwa \textit{AUTOSAR Zeiterweiterungen}) verwendet, mit denen das das Zeitverhalten von Hard- und Softwarekomponenten beschrieben werden kann. Andere Arbeiten, etwa das TIMMO-2-USE Projekt, haben daran gearbeitet, die AUTOSAR Timing Extensions zu formalisieren und somit einen Grundbaustein dafür zu legen, die Definitionen vom Zeitverhalten automatisiert zu kontrollieren, etwa durch Model Checking. Ein Problem von Model Checking und ähnlichen Ansätzen ist, dass diese aufgrund der extrem großen Laufzeit auf kleinere Systeme beschränkt sind. Runtime Verification ist eine leichtgewichtigere Methode der Analyse von Systemkomponenten, die einen Mittelweg zwischen formaler Analyse und Testen geht, wobei formal definierte Eigenschaften des Systems während der Laufzeit geprüft werden.\\
	Im Rahmen dieser Arbeit werden die 18 TADL2 Timing Constraints, welche im Rahmen des TIMMO-2-USE Projekt erarbeitet wurden, dahingehend überprüft, ob sie in mittels Runtime Verification überwacht werden können. Darauf aufbauend wird für jeden dieser Constraints ein Monitor in der Sprache TeSSLa, welche für die Überwachung von Zeiteigenschaften auf Strömen entwickelt wurde, implementiert.
}

  \cleardoublepage
  \phantomsection
  \pdfbookmark{Contents}{tableofcontents}
  \markboth{Contents}{}
  \tableofcontents

  % Remove this for the final version of the thesis!
  \cleardoublepage
  \phantomsection
  \pdfbookmark{Liste der Todos}{listoftodos}
  \listoftodos[Liste der Todos]

  \mainmatter
  %!TEX root = thesis.tex

\chapter{Introduction}

Timing behavior is one of the most important properties of computer systems. Especially in safety-critical applications, a wrong timed reaction of the system can have disastrous consequences, for example an intervention of pacemaker, that occurred too early or too late would risk the life of the patient. In Cyber-physical Systems, e.g. the Electronic Stability Control of a vehicle, wrong timing can also lead to property damage, injuries or deaths. Also environmental aspects are affected by timing cyber-physical systems, for example combustion engines need exact timing to produce as little emissions as possible.\\
The interconnection of several components in cyber-physical systems makes the design and analysis of the timing behavior of these systems significant hard, because not only the components on its own, but also the complete system must be considered. In this context, testing is a major problem, because it is hard to reproduce the exact state of the system, in which the error occurred. In many cases, the error does not lay in the component where it became visible, it was carried off to other parts, which results in a malfunctioning system, where it is extremely hard to find the bug that caused the problem. Online Monitoring is the key technique to address this problem, because you can isolate the error, without the need of storing and recreating the state of the system, when searching the error.\\
The goal of this thesis is to create a monitoring tool for the \emph{AUTOSAR} (\textbf{AUT}omotive \textbf{O}pen \textbf{S}ystem \textbf{AR}chitecture) Timing Extensions, which were created to increase the interoperability and exchangeability of car components.

%\section{Verwandte Arbeiten}
 
%\textbf{AUTOSAR} (AUTomotive Open System ARchitecture) ist eine Partnerschaft aus Automobilherstellern und dazugehörigen Software, Hardware Unternehmen, deren Zulieferern und weiteren. Ziel dieser Partnerschaft ist die Erstellung offener Standards für Soft- und Hardwarekomponenten im Automobilbereich, sowie deren Entwicklungsprozesse ~\cite{AUTOSAR_History}.\\
%Die AUTOSAR Timing Extensions (kurz \textbf{AUTOSAR TIMEX}) spezifieren Constraints, mit denen das Zeitverhalten von Komponenten, die mit Hilfe anderer AUTOSAR Standards definiert wurden, beschrieben werden kann ~\cite{TIMEX}.\\
%TODO ~\cite{TIMMO2USE}\\
%\textbf{TeSSLa} (Temporal Stream-based Specification Language) ist eine turingfähige Programmiersprache, die zur Analyse und zur Überwachung von Zeitverhalten, insbesondere das von Cyber-Physical Systems. Es nimmt dabei Ströme von Datenpunkten, die mit Zeitstempeln verknüpft sind, entgegen und führt auf diesen Berechnungen durch ~\cite{TeSSLa}.\\
%Damit eng verknüpft ist das \textbf{COEMS}-Project, in dem Möglichkeiten von hardwarebasierte, non-intrusive, online Stream Runtime Verification erarbeitet wurden. Hierbei werden vorhandene Debug-Informationen aus einem System mittels einer TeSSLa-Spezifikation, die auf eine FPGA-basierter Hardware übertragen wurde, analysiert.



% Eine wichtiger Abschnitt der Einleitung stellt einen Überblick über verwandte Arbeiten dar. Was wurde bereits in der Literatur untersucht und ist \emph{nicht} Thema dieser Arbeit?

%\section{Aufbau der Arbeit}

%Neben dieser Einleitung und der Zusammenfassung am Ende gliedert sich diese Arbeit in die folgenden drei Kapitel.
%\begin{description}
%  \item[\ref{chapter-basics}] beschreibt die für diese Arbeit benötigten Grundlagen. In diesem Kapitel werden \ldots, \ldots und \ldots eingeführt, da diese für die folgenden Kapitel dringend benötigt werden.
%  \item[\ref{chapter-konzept}] stellt das eigentliche Konzept vor. Dabei handelt es sich um ein Konzept zur Verbesserung der Welt. Das Kapitel gliedert sich daher in einen globalen und einen lokalen Ansatz, wie die Welt zum Besseren beeinflusst werden kann.
%  \item[\ref{chapter-evaluation}] beinhaltet eine Evaluation des Konzeptes aus dem vorherigen Kapitel. Anhand von Simulationen wird in diesem Kapitel untersucht, wie die Welt durch konkrete Maßnahmen deutlich verbessert werden kann.
%\end{description}


  %!TEX root = thesis.tex

\chapter{Timing Constraints}
\label{chapter-TimingConstraints}

\section{AUTOSAR Timing Extensions}
	AUTOSAR is a development partnership in the automotive industry. As stated before, the main goal is to define a standardized interface and increasing interoperability, exchangebility and re-usability of parts and therefore simplifying development and production. Three different layers are defined in the specification. \emph{Basic Software} is an abstraction layer from components, like network or diagnostic protocols, or operating systems. \emph{AUTOSAR-Software} defines the methods, how applications have to be build. For Basic Software and AUTOSAR Software, there are definitions for standardized Interfaces to enable the communication via the \emph{AUTOSAR Runtime Environment}. It works as middleware, in which the \emph{virtual function bus} is defined ~\cite{Virtual_Functional_Bus}.
	The AUTOSAR Timing Extension are describing timing constraints for actions and reactions of components, that are communicating via the Virtual Function Bus. They are defined via \emph{events}, which consists of a time and a data value, the type of the time and data value is arbitrary, the only restriction is, that the time values are strictly increasing. To describe the logical relationship between groups of events, \emph{event chains} are defined, which consist of a \emph{stimulus} and \emph{response} event. The \emph{response} event is understood as the answer to the \emph{stimulus} event.\\
	The AUTOSAR Release 4.4.0 (\cite{TIMEX}) is used for this thesis, there are 12 timing constraints defined in this version of the AUTOSAR Timing Extensions
	\begin{enumerate}
		\item
			The subset of 5 \textbf{EventTriggeringConstraints} are describing, at which points in time specific events may occur.
			\begin{enumerate}[1]
				\item
					The \textbf{PeriodicEventTriggering} defines repetitions of event with the same time distance and offers the possibility to set an allowed deviation from this pattern. Also the minimal distance between two subsequent events can be defined.
				\item
					The \textbf{SporadicEventTriggering} specifies sporadic event occurrences by defining the minimal and maximal distance between subsequent events. Optionally, periodic repetitions and allowed deviations from the period can be described.
				\item
					With the \textbf{ConcreteEventTriggering}, offsets between a set of subsequent events in a time interval can be described. These intervals may not overlap, and periodic repetitions of them can be defined optionally.
				\item
					The \textbf{BurstPatternEventTriggering} describes not overlapping event clusters with a minimal and maximal number of events and optionally periodic repetitions of these clusters.
				\item
					The \textbf{ArbitraryEventTriggering} defines the distance between subsequent event by defining \emph{ConfidenceIntervals}, which describe the probability, in which time interval the following event will occur.
			\end{enumerate}
		\item
			The \textbf{LatencyTimingConstraint} specifies the minimal, nominal and maximal time distance between the stimulus and response events of an event chain.
		\item
			The \textbf{AgeConstraint} is a simpler form of the \emph{LatencyTimingConstraint} by defining minimal and maximal age a event may have at the point of time, when it is processed.
		\item
			The \textbf{SynchronizationTimingConstraint} is used for describing events of different kinds, that occur synchronized in a time interval of a specific length.
		\item
			The \textbf{SynchronizationPointConstraint} defines two sets of executables and events. Every element of the first set must have finished or occurred, before the first element of the second set may start or occur.
		\item
			The \textbf{OffsetTimingConstraint} specifies the minimal and maximal time distance between corresponding \emph{source} and \emph{target} events.
		\item
			The \textbf{ExecutionOrderConstraint} offers the possibility to define a list of executables, which must start and finish in the order given in the list.
		\item
			The \textbf{ExecutionTimeConstraint} defines the minimal and maximal runtime of an executable, including or excluding the runtime of external functions and interruptions.
	\end{enumerate}

	In this simplified form, some constraints are redundant. The semantic differences will be shown in section~\ref{comparisonConstraints}.

	Problematic with the AUTOSAR Timing Extensions is, that the definitions are not very formal and have room left for interpretation. As example, the \emph{BurstPatternEventTriggering} will be analyzed in the following. This constraint describes events clusters, with events that occur with short time distances, with larger time distances between the clusters. The following attributes are needed:
	\begin{itemize}
		\item
			\textbf{\emph{maxNumberOfOccurrences}} (positive integer)\\
			Maximal number of events per burst
		\item
			\textbf{\emph{minNumberOfOccurrences}} (positive integer)\\
			Minimal number of events per burst (optional)
		\item
			\textbf{\emph{minimumInterArrivalTime}} (time value)\\
			Minimal distance between subsequent events
		\item
			\textbf{\emph{patternLength}} (time value)\\
			Length of each burst
		\item
			\textbf{\emph{patternPeriod}} (time value)\\
			Time distance between the starting points of subsequent burst(optional)
		\item
			\textbf{\emph{patternJitter}} (time value)\\
			Maximal allowed deviation from the periodic pattern	(optional)
	\end{itemize}

As example, we set:
\begin{itemize}
	\item
	$maxNumberOfOccurrences = 3$
	\item
	$minNumberOfOccurrences = 1$
	\item
	$minimumInterArrivalTime = 1$
	\item
	$patternLength = 3$
	\item
	$patternPeriod = 3.5$
	\item
	$patternJitter = 1.5$
\end{itemize}

\begin{figure}
	\centering
	\begin{tikzpicture}[thick]
	% time axis
	\foreach \x in {0,...,17}
	\draw (\x,-4) -- (\x,-4.2) node[anchor=north] {\x};
	\draw[->] (0, -4.1) -- (17.4, -4.1);
	\node at(17, -5) {time};
	
	% bursts
	\node[gray] at (0, -2){Bursts};
	\draw [fill=lightgray, lightgray] (1, -2) rectangle (4,-2.5);
	\draw [fill=lightgray, lightgray] (6, -2) rectangle (9,-2.5);
	\draw [fill=lightgray, lightgray] (11, -2) rectangle (14,-2.5);
	\draw [fill=lightgray, lightgray] (16, -2) rectangle (19,-2.5);
	% events
	\node at (0, -2.4){Events};
	%\node at (0, -2.15){events};
	\foreach \x in {1, 6, 11, 16}
	{
		% events
		\draw[-] (\x, -2) -- (\x, -2.5);
		\draw[-] (\x+1, -2) -- (\x+1, -2.5);
		\draw[-] (\x+2, -2) -- (\x+2, -2.5);
		% periods
		\draw[->, green] (\x, -3) -- (\x+3.5, -3);
		\node[green] at (\x+1.75, -3.3){period};
		%jitters
		\draw[->, blue] (\x+3.5,-3.4) -- (\x+5, -3.4);
		\node[blue] at (\x+4.25, -3.7) {jitter};
	}
	\end{tikzpicture}
	\caption{BurstPatternEventTriggering Period-Jitter \textbf{accumulating}}
	\label{fig:BurstPatternEventTriggering1}
\end{figure}
\begin{figure}
	\centering
	\begin{tikzpicture}[thick]
	% time axis
	\foreach \x in {0,...,17}
	\draw (\x,-4.2) -- (\x,-4.4) node[anchor=north] {\x};
	\draw[->] (0, -4.3) -- (17.4, -4.3);
	\node at(17, -5.2) {time};
	
	% bursts
	\node[gray] at (0, -2){Bursts};
	\draw [fill=lightgray, lightgray] (1, -2) rectangle (4,-2.5);
	\node at (0, -2.4){Events};
	
	%period 1
	\draw[->, green] (1, -3) -- (4.5, -3);
	\node[green] at (2.75, -3.3){period};
	
	% jitter 1
	\draw[->, blue] (4.5,-3.6) -- (6, -3.6);
	\node[blue] at (5.25, -3.9) {jitter};
	
	\foreach \y in {0, 1, 2}
	\draw[-] (1+\y, -2) -- (1+\y, -2.5);
	
	\foreach \x in {4.5, 8, 11.5, 15, 18.5}
	{
		% periods
		\draw[->, green] (\x, -3) -- (\x+3.5, -3);
		\node[green] at (\x+1.75, -3.3){period};
		% jitters
		\draw[->, blue] (\x+3.5,-3.6) -- (\x+5, -3.6);
		\node[blue] at (\x+4.25, -3.9) {jitter};
		%bursts	
		\draw [fill=lightgray, lightgray] (\x+1.5, -2) rectangle (\x+4.5,-2.5);
		% events
		\foreach \y in {0, 1, 2}
		\draw[-] (\x+\y+1.5, -2) -- (\x+\y+1.5, -2.5);
	}
	\end{tikzpicture}
	\caption{BurstPatternEventTriggering Period-Jitter \textbf{non-accumulating}}
	\label{fig:BurstPatternEventTriggering2}
\end{figure}

The combination of $patternPeriod$ and $patternJitter$ can be interpreted in an accumulating as seen in \ref{fig:BurstPatternEventTriggering1} or non-accumulating way as seen in \ref{fig:BurstPatternEventTriggering2} way. In the accumulating interpretation, the reference for the periodic occurrences is only the start point of the previous burst. In the non-accumulating way, there is an global reference point for the periodic repetitions.

With the definition of $patternLength$ (''time distance between the beginnings of subsequent repetitions of the given burst pattern'') you would think, that the accumulating variant is meant. Against that, the period attribute in $PeriodicEventTriggering$-Constraint is defined as ''distance between subsequent occurrences of the event'' in the text, hence it is also understandable the accumulating way, but there is the formal definition

\begin{math}
\exists t_{reference}\forall t_n: t_{reference}+(n+1)*period\leq t_n\leq t_{reference}+(n-1)*period+jitter,
\end{math}

where $t_n$ is the time of the $n$-th Event and $t_{reference}$ is a reference point, from which the periodic pattern starts, so the $PeriodicEventTriggering$-Constraint is meant to be understood in the non-accumulating way. It remains unclear, in which way the $BurstPatternEventTriggering$ is meant to be understood.

Another problem of the AUTOSAR Timing Extensions is, that they were made for design purposes, monitoring them can be difficult, as they may need continuously growing time and memory resources, which makes online monitoring unsuitable in nearly all scenarios (more on monitorability in \ref{chapter-monitorability}). As example, we will use the burst pattern again, this time using the attributes
\begin{itemize}
	\item
	$maxNumberOfOccurrences = INT\_MAX$
	\item
	$minNumberOfOccurrences = 1$
	\item
	$minimumInterArrivalTime = 0$
	\item
	$patternLength = 3$
	\item
	\textcolor{gray}{$patternPeriod$} \textcolor{gray}{unused}
	\item
	\textcolor{gray}{$patternJitter$} \textcolor{gray}{unused}
\end{itemize}

Figure~\ref{fig:BurstPatternEventTriggering3} shows the application of the \emph{BurstPatternEventTriggering} constraint with the given parameters on a stream with events at the timestamps 3, 3.5, 4, 4.5. The development of possible the burst cluster with ongoing time is visualized. The gray bars show, where the burst cluster can lay, the black lines show, where they definitely are. In timestamp 3 with only one event so far, only one burst has to be considered and it can lay between timestamp 0 and 6, the only limitation is, that it must include timestamp 3 with the event in that point. In Timestamp 3.5, there are two events (at 3 and 3.5) so far and there are two possibilities for burst placements. The first possibility with only one burst with both events in it, and the second possibility, where the events are in different bursts. The third graphic shows the trace in timestamp 4 with three different events so far (3, 3.5, 4) and three different possibilities for burst placements to consider. One possible burst contains all three events, the second possibility has one burst with the event at timestamp 3 and one burst with the events at 3.5 and 4 and the third possibility has one Burst with the events at 3 and 3.5 and one burst with the event at 4. The possible bursts in graphic 4 are analog to the third graphic, one possibility with one burst containing all 4 events and 3 possibilities with the first burst containing the first event, the first and second event or the first, the second and the third event and the second burst containing the remaining events.\\
In this Example, we see, that it is possible to create an unlimited number of possibilities for burst placements within one burst length, when the \textit{minimumInterArrivalTime}-attribute is 0, which results in an infeasible resource consumption, as unlimited memory and time is needed to check the constraint in following events. Therefore, online monitoring this constraint is unsuitable in most cases.

\begin{figure}
	\begin{tikzpicture}[thick]
	%time axis
	\foreach \x in {0,...,10}
	{
		\draw (\x,0) -- (\x,-0.2);
		\node at (0+\x, 0.2) {\x};
	}
	\foreach \x in {0,...,9}
	\draw (\x+0.5,0) -- (\x+0.5,-0.1);
	\draw[->] (0, -0.1) -- (10.4, -0.1);
	\node at(10, -0.4) {time};
	
	\node[red] at(-1, -0.45) {timestamp};
	%watched time
	\draw[->, red] (3, -0.7)--(3, -0.3);
	
	\node at (-1, -1.5) {Events};
	\draw[-] (3, -1.25) -- (3, -1.75);
	
	\node at (-1, -2.2) {Possible};
	\node at (-1, -2.6) {Burst};
	\node at (-1, -3) {ranges};
	
	%Burst range
	\draw[|-|, lightgray] (0, -2.4) -- (6, -2.4);
	\end{tikzpicture}
	
	\begin{tikzpicture}[thick]
	%time axis
	\foreach \x in {0,...,10}
	{
		\draw (\x,0) -- (\x,-0.2);
		\node at (0+\x, 0.2) {\x};
	}
	\foreach \x in {0,...,9}
	\draw (\x+0.5,0) -- (\x+0.5,-0.1);
	\draw[->] (0, -0.1) -- (10.4, -0.1);
	\node at(10, -0.4) {time};
	
	\node[red] at(-1, -0.45) {timestamp};
	%watched time
	\draw[->, red] (3.5, -0.7)--(3.5, -0.3);
	
	\node at (-1, -1.5) {Events};
	\draw[-] (3, -1.25) -- (3, -1.75);
	\draw[-] (3.5, -1.25) -- (3.5, -1.75);
	
	\node at (-1, -2.2) {Possible};
	\node at (-1, -2.6) {Burst};
	\node at (-1, -3) {ranges};
	
	%Burst range 1
	\draw[|-|, lightgray] (0.5, -2.2) -- (6, -2.2);
	\draw[-] (3, -2.2) -- (3.5, -2.2);
	% Burst range 2
	\draw[|-|, lightgray] (0, -2.6) -- (3.5, -2.6);
	\draw[-] (0.5, -2.6) -- (3, -2.6);
	\draw[|-|, lightgray] (3, -2.7) -- (6.5, -2.7);
	\draw[-] (3.5, -2.7) -- (6, -2.7);
	\end{tikzpicture}
	\begin{tikzpicture}[thick]
	%time axis
	\foreach \x in {0,...,10}
	{
		\draw (\x,0) -- (\x,-0.2);
		\node at (0+\x, 0.2) {\x};
	}
	\foreach \x in {0,...,9}
	\draw (\x+0.5,0) -- (\x+0.5,-0.1);
	\draw[->] (0, -0.1) -- (10.4, -0.1);
	\node at(10, -0.4) {time};
	
	\node[red] at(-1, -0.45) {timestamp};
	%watched time
	\draw[->, red] (4, -0.7)--(4, -0.3);
	
	\node at (-1, -1.5) {Events};
	\draw[-] (3, -1.25) -- (3, -1.75);
	\draw[-] (3.5, -1.25) -- (3.5, -1.75);
	\draw[-] (4, -1.25) -- (4, -1.75);
	
	\node at (-1, -2.2) {Possible};
	\node at (-1, -2.6) {Burst};
	\node at (-1, -3) {ranges};
	
	%Burst ranges poss. 1
	\draw[|-|, lightgray] (1, -2.2) -- (6, -2.2);
	\draw[-] (3, -2.2) -- (4, -2.2);
	% Burst ranges poss. 3 2.6,2.7
	\draw[|-|, lightgray] (0, -2.6) -- (3.5, -2.6);
	\draw[-] (0.5, -2.6) -- (3, -2.6);
	\draw[|-|, lightgray] (3, -2.7) -- (6.5, -2.7);
	\draw[-] (3.5, -2.7) -- (6, -2.7);
	% Burst ranges poss. 2
	\draw[|-|, lightgray] (0.5, -3.1) -- (4, -3.1);
	\draw[-] (1, -3.1) -- (3.5, -3.1);
	\draw[|-|, lightgray] (3.5, -3.2) -- (7, -3.2);
	\draw[-] (4, -3.2) -- (6.5, -3.2);
	\end{tikzpicture}
	\begin{tikzpicture}[thick]
	%time axis
	\foreach \x in {0,...,10}
	{
		\draw (\x,0) -- (\x,-0.2);
		\node at (0+\x, 0.2) {\x};
	}
	\foreach \x in {0,...,9}
	\draw (\x+0.5,0) -- (\x+0.5,-0.1);
	\draw[->] (0, -0.1) -- (10.4, -0.1);
	\node at(10, -0.4) {time};
	
	\node[red] at(-1, -0.45) {timestamp};
	%watched time
	\draw[->, red] (4.5, -0.7)--(4.5, -0.3);
	
	\node at (-1, -1.5) {Events};
	\draw[-] (3, -1.25) -- (3, -1.75);
	\draw[-] (3.5, -1.25) -- (3.5, -1.75);
	\draw[-] (4, -1.25) -- (4, -1.75);
	\draw[-] (4.5, -1.25) -- (4.5, -1.75);
	
	\node at (-1, -2.5) {Possible};
	\node at (-1, -2.9) {Burst};
	\node at (-1, -3.3) {ranges};
	
	%Burst ranges poss. 1
	\draw[|-|, lightgray] (1.5, -2.2) -- (6, -2.2);
	\draw[-] (3, -2.2) -- (4.5, -2.2);
	% Burst ranges poss. 2
	\draw[|-|, lightgray] (0, -2.6) -- (3.5, -2.6);
	\draw[-] (0.5, -2.6) -- (3, -2.6);
	\draw[|-|, lightgray] (3, -2.7) -- (6.5, -2.7);
	\draw[-] (3.5, -2.7) -- (6, -2.7);
	% Burst ranges poss. 3
	\draw[|-|, lightgray] (0.5, -3.1) -- (4, -3.1);
	\draw[-] (1, -3.1) -- (3.5, -3.1);
	\draw[|-|, lightgray] (3.5, -3.2) -- (7, -3.2);
	\draw[-] (4, -3.2) -- (6.5, -3.2);
	% Burst ranges poss. 4
	\draw[|-|, lightgray] (1, -3.6) -- (4.5, -3.6);
	\draw[-] (1, -3.6) -- (4, -3.6);
	\draw[|-|, lightgray] (4, -3.7) -- (7.5, -3.7);
	\draw[-] (4.5, -3.7) -- (7, -3.7);
	\end{tikzpicture}
	\caption{BurstPatternEventTriggering Possible bursts, \textcolor{red}{$\uparrow$} shows the current time}
	\label{fig:BurstPatternEventTriggering3}
	
%TODO weiteres Beispiel für unsaubere Definition von AUTOSAR-> falsches Beispiel für file:///home/hendrik/Documents/Uni/Semester8/Quellen%20Bachelorarbeit/AutoSar%20TimEx/MethodologyAndTemplates/AUTOSAR_TPS_TimingExtensions.pdf
\end{figure}

\newpage

\section{TADL2}
	As timing extension to EAST-ADL(\textbf{E}lectronics \textbf{A}rchitecture and \textbf{S}oftware \textbf{T}echnology-\textbf{A}rchitecture \textbf{D}escription \textbf{L}anguage), the TIMMO (\textbf{Tim}ing \textbf{Mo}del) project, and its successor TIMMO2USE, were initiated. EAST-ADL has similar goals as AUTOSAR, but the definitions are written in a more formalized fashion. The definitions of the AUTOSAR Timing Extensions are only textually described often, the TADL2-Definitions are defined in a more formal way, as they offer a formal definition of each constraint in a timing constraint logic \cite{TIMMO2USE}. EAST-ADL is much less used in the automotive industry, but the EAST-ADL Timing Constraints are partly compatible to the AUTOSAR Timing Extensions, as they are sub- or supersets of each other. Many of the AUTOSAR Timing Extensions can be defined via a combination of TADL2 Constraints, as explained in section~\ref{comparisonConstraints}.\\
	The timing constraints are defined on events or event chains, similar to the AUTOSAR Timing Extensions. In TADL2, all events of an event chain have a color attribute, which shows the logical connection of these events. This attribute is defined as abstract and possibly infinite datatype. The only restriction is, that an equality test on these color values must exist. TADL2 offers 18 timing constraints, which will briefly explained in the following:
	\begin{itemize}
		\item
			The \textbf{StrongDelayConstraint} defines the minimal and maximal time distance of the events from two event sets (\emph{source} and \emph{target}).
		\item
			The \textbf{DelayConstraint} is a less strict variant of the \textbf{StrongDelayConstraint}, because it allows additional events in \emph{target}.
		\item
			The \textbf{RepeatConstraint}, \textbf{RepetitionConstraint}, \textbf{PeriodicConstraint}, \textbf{SporadicConstraint} and \textbf{ArbitraryConstraint} are describing the time distance between subsequent events, whereby they are having small semantic differences. An exact distinction between these constraints will be given in section~\ref{tadl2Constraints}.
		\item
			The \textbf{SynchronizationConstraint} and \textbf{StrongSynchronizationConstraint} define groups of event sets, whose events occur in common time intervals. The SynchronizationConstraint allows more than one event of each group per interval, the StrongSynchronizationConstraint does not.
		\item
			The \textbf{ExecutionTimeConstraint} is used to set a minimum and a maximum for the runtime of a task, not considering interruptions in the execution.
		\item
			The \textbf{OrderConstraint} defines that the $n^{th}$ event of one event set must occur before or at the $n^{th}$ event of a second event set.
		\item
			The \textbf{ComparisonConstraint} is used to describe ordering relations of timestamps.
		\item
			The \textbf{PatternConstraint} defines the time distance between periodic points in time to several events.
		\item
			The \textbf{BurstConstraint} regulates the maximum number of events in time intervals of a specific length.
		\item
			The \textbf{ReactionConstraint} describes the minimal and maximal time a response event must occur after the associated stimulus event. Additional response events are allowed, additional stimulus events not.
		\item
			The \textbf{AgeConstraint} is similar to the ReactionConstraint, but it is defined the other way around. Therefore, it describes the minimal and maximal time a stimulus event must occur before the associated response event.  Additional stimulus events are allowed, additional response events not.
		\item
			The \textbf{OutputSynchronizationConstraint} is used to describe groups of event chains, which all have the same response events. The response events of the event chain must occur in common time intervals, like in the SynchronizationConstraint. In the \textbf{InputSynchronizationConstraint}, the roles of the stimulus and response events are swapped.			
	\end{itemize}
	
\subsection{Parenthesis - Simple and Flexible Timing Constraint Logic}
	The formal definition of the TADL2 timing constraint are written in \emph{Timing Constraint Logic} (short: \emph{TiCL}), which was developed as part of the TIMMO-2-USE project. TiCL was formally introduced in \cite{TiCL}, for better understanding the key aspects of this article will be explained in the following.\\
	The main goal of TiCL is to be formal and expandable and offering the possibility of defining finite and infinite behaviors of events. In TiCL, only points in time, when events occur, are considered, therefore an events only consists of a real number as timestamp, without the possibility of adding a data value. There are 7 syntactic categories in TiCL
	\begin{align*}
		\mathbb{R} &\text{(arithmetic constants)}\\
		Avar &\text{(arithmetic variables)}\\
		AExp &\text{(arithmetic expressions)}\\[10pt]
		%\vspace{1cm}
		Svar &\text{(set variables)}\\
		SExp &\text{(set expressions)}\\[10pt]
		%\vspace{1cm}
		TVar &\text{(time variables)}\\
		CExp &\text{(constraint expressions)}
	\end{align*}
	Arithmetic expressions can be defined as arithmetic constants, arithmetic variables, application of $+,-,*,/$ on arithmetic expressions, application of the cardinality operator on a set ($|E|$, $E\in SExp$) or as measure $\lambda(E)$ ($E\in SExp$). $\lambda(E)$ is defined as Lebesgue measure, which is figuratively speaking, the length of all continuous intervals of $E$. In figure~\ref{fig:TiCLMeasureExample} an example of the measure operator $\lambda$ is visualized. The set $E$ contains all Events between the timestamps $1$ and $9$, the set $F$ contains the events at the timestamps between 2 and 4 and 6 and 7, therefore $E\setminus F$ contains the events at the timestamps $\{1, 1.5, 4.5, 5, 5.5, 7.5, 8, 8.5, 9\}$.
	$E$ consists of one continuous interval from timestamp 1 to 9 with the length of 8, $F$ consists of two continuous intervals from 2 to 4 with the length of 2 and from 6 to 7 with the length of 1, therefore $\lambda(F)=3$. $E\setminus F$ consists of three continuous intervals, the first from 1 to 1.5 (length = 0.5), the second from 4.5 to 5.5 (length = 1) and the last from 7.5 to 9 (length = 1.5), so the total length of the continuous intervals of $E\setminus F$ is 3.\\
	% TODO richtig? vgl. executionTimeConstraint
	\begin{figure}
		\begin{tikzpicture}[thick]
			\foreach \x in {0,...,10}
			{
				\draw[very thin, lightgray](\x, 1.8) -- (\x, -4);
				
				\draw (\x,-3.6) -- (\x,-3.9);
				\node at (0+\x, -4.2) {\x};
			}
			\draw[->] (-0.2, -3.75) -- (10.5, -3.75);
			
			\node at (-1,  0) {E};
			\draw [fill=gray,gray] (1,0.25) rectangle (9,-0.25); 
			
			\node at (-1, -1.5) {F};
			\draw [fill=gray,gray] (2,-1.25) rectangle (4,-1.75); 
			\draw [fill=gray,gray] (6,-1.25) rectangle (7,-1.75);
			
			\node at (-1, -3) {$E \setminus F$};
			\draw [fill=gray,gray] (1,-2.75) rectangle (1.5,-3.25); 
			\draw [fill=gray,gray] (4.5,-2.75) rectangle (5.5,-3.25);
			\draw [fill=gray,gray] (7.5,-2.75) rectangle (9,-3.25);
			
			\node at (11, 0) {$\lambda(E) = 8$};
			\node at (11, -1.5) {$\lambda(F) = 3$};
			\node at (11.36, -3) {$\lambda(E\setminus F) = 3$};
			
			\node at (-1.5, 1.5){Events};
			\foreach \x in {0, 0.5, ...,10}
			\draw (\x, 1.7) -- (\x, 1.3);
		\end{tikzpicture}
		\caption{Graphical example of $\lambda(E), \lambda(F)$ and $\lambda(E\setminus F)$}
		\label{fig:TiCLMeasureExample}
	\end{figure}\\
	Set expressions can be defined as set variables, or as set of time variables that fulfill a given constraint expression.\\
	Constraint expressions can be defined as application of the $\leq$-operator on time or arithmetic expressions, the $\in$ operator on time variables and set expressions, the logical conjunction on constraint expressions, the negation of constraint expressions and the $\forall$-Quantifier on arithmetic, set and time variables over an constraint expression.\\
	As extension to this definition, well known syntactic abbreviations like $true\equiv 0\leq 1$ or the $\exists$-quantifier will be used, but there are also some TiCL-specific syntactic abbreviations, like interval constructors, which will be defined and explained in the following.\\
	\subsubsection{Interval Constructors}
		Let $x, y\in Tvar$ and $E, F\in SExp$.\\
		The constructor $[x\leq]$($[x<]$) is defined as $\{y: x \leq y\}$($\{y: x < y\}$), therefore the interval contains all points in time laying behind of $x$, possibly containing $x$.\\
		$[\leq x]$($[< x]$) is defined as complement of $[x<]$($[x\leq]$) and contains all timestamps laying before $x$.\\
		$[x..y]$ is defined as $[x\leq]\cap[<y]$, so all points of time after $x$ and before $y$, including $x$ but not $y$, are part of this interval.\\
		$[E \leq]$ is defined as $\{y : \exists x \in E : x \leq y\}$, this interval contains all point of time at and after the first timestamp in $E$. $[E<]$ is equal to $\{y : \forall x \in E : x < y\}$, therefore it defines the interval containing all timestamps after the latest point of time in $E$.\\
		$[\leq E]$ ($[< E]$) is defined as $[E<]^C$ ($[E\leq]^C$), analogous to the operators on time variables.\\
		$[E]$ is equal to $[E\leq]\cap[\leq E]$. It defines the time interval between the first and last element of $E$, including these points in time.\\
		$E_{x<}$($E_{<x}$) is defined as $E\cap [x<]$($E\cap [<x]$). This operators filters the timestamps in $E$ so that only the points in time before (after) remain.\\
		$[x..E]$ equals $[x\leq]\cap[<(E_{x<})]$. The interval begins at $x$ and ends right before the first element of $E$ after $x$.\\
		$[E..F]$ is defined as $\{x:\exists y\in E:x\in[y..F]\}$ and describes the intervals, where the previous operator is applied on every element of $E$.\\
	% TODO ? es gibt noch mehr, nur einfügen, wenn benötigt.
	% TODO x-y/y-x
	% TODO Index in Eventmenge -> hier oder in nächster subsection (in TADL neu definiert)
	% TODO X\leq Y -> X is subsequence of Y
\subsection{TADL2-Timing Constraints}
	\label{tadl2Constraints}
	For better understanding of the following chapters, the TADL Constraints will be presented next. As abbreviation and unification, all timing expressions are defined as set $\mathbb{T}$, which are understood as real numbers but expanded with $\infty$ and $-\infty$ in this chapter, but other value ranges for time expressions are possible and will be used in other parts of this thesis.\\
	We define an event as a time value, possibly combined with an data value. The range of the data values are arbitrary, infinite data types are possible, also as empty data types, when only the point in time is relevant for the event. All TADL constraints are defined with attributes, which can be events, timing or arithmetic expressions or sets of them. Also, \emph{EventChains} can be used as attributes. An \emph{EventChain} consists of two sets of events (\emph{stimulus} and \emph{response}),  which are causally related. All events in an \emph{EventChain} must have a color value in their data field. This color possibly has an infinite type and an equality check on this must be defined. It is used to check, which events of an \emph{EventChain} are directly related.
	\subsubsection{DelayConstraint}
		The \emph{DelayConstraint} has 4 attributes
		\begin{align*}
			\emph{source} & \hspace{.5cm}\text{event set}\\
			\emph{target} & \hspace{.5cm}\text{event set}\\
			\emph{lower}  & \hspace{.5cm}\text{$\mathbb{T}$ (time expression)}\\
			\emph{upper}  & \hspace{.5cm}\text{$\mathbb{T}$}
		\end{align*}
		and is defined as\\[10pt]
		\begin{math}
			\forall x\in source:\exists y\in target: lower\leq y-x\leq upper.
		\end{math}\\[10pt]
		For all events $x$ in \emph{source}, there must be an $y$ event in \emph{target}, so that $y$ lays between \emph{lower} and \emph{upper} after $x$. Note, that \emph{lower} and \emph{upper} can have negative values and that additional events in \emph{target}, without an associated \emph{source} event are allowed.\\
		Figure~\ref{fig:delayConstraintExample} shows a visualized example of the \emph{DelayConstraint} with the attributes $lower=2$, $upper=3$, $source=\{1, 5, 6\}$ and $target=\{2, 3.5, 5, 7, 8.2, 9\}$. The first element of source at timestamp 1 results in a required event in target between the timestamp 3 and 4 that is fulfilled by the event at 3.5. The second event of source requires an target event between 7 and 8, fulfilled by the event at 7. The last event of source is satisfied by the target event at 8.2 and 9.
		\begin{figure}
			\begin{tikzpicture}
				% source events
				\node[] at (0,-0.2){source};
				\draw (1, 0) -- (1, -0.4);
				\draw (5, 0) -- (5, -0.4);
				\draw (6, 0) -- (6, -0.4);
				
				% upper/lower 1
				\draw [fill=lightgray, lightgray] (3, -0.7) rectangle (4,-2.3);
				\node at (2, -0.5){lower};
				\node at (2, -1){upper};
				\draw[->] (1,-0.7) -- (3, -0.7);
				\draw[->] (1, -0.8) -- (4, -0.8);
				
				
				% upper/lower 2
				\draw [fill=lightgray, lightgray] (7, -0.7) rectangle (8,-2.3);
				\node at (6, -0.5){lower};
				\node at (6, -1){upper};
				\draw[->] (5,-0.7) -- (7, -0.7);
				\draw[->] (5, -0.8) -- (8, -0.8);
				
				% upper/lower 3
				\draw [fill=lightgray, lightgray] (8, -0.7) rectangle (9,-2.3);
				\node at (7, -1.1){lower};
				\node at (7, -1.6){upper};
				\draw[->] (6,-1.3) -- (8, -1.3);
				\draw[->] (6, -1.4) -- (9, -1.4);
				% target events
				\node[] at (0,-2.1){target};
				\draw (2, -1.9) -- (2, -2.3);
				\draw (3.5, -1.9) -- (3.5, -2.3);
				\draw (5, -1.9) -- (5, -2.3);
				\draw (7, -1.9) -- (7, -2.3);
				\draw (8.2, -1.9) -- (8.2, -2.3);
				\draw (9, -1.9) -- (9, -2.3);
				
				\foreach \x in {0, 1, ..., 10}{
					\draw (\x, -3.1) -- (\x, -3.3);
					\node at(\x, -3.6) {\x};
				}
			
				\foreach \x in {0.5, 1.5, ..., 9.5}
					\draw (\x, -3.1) -- (\x, -3.2);
				\draw[->] (-0.1, -3.2) -- (10.2, -3.2);
					
			\end{tikzpicture}
			\caption{Example DelayConstraint - $lower = 2$, $upper = 3$}
			\label{fig:delayConstraintExample}
		\end{figure}
	\subsubsection{StrongDelayConstraint}
		The \emph{StrongDelayConstraint} has 4 attributes
		\begin{align*}
			\emph{source} & \hspace{.5cm}\text{event set}\\
			\emph{target} & \hspace{.5cm}\text{event set}\\
			\emph{lower}  & \hspace{.5cm}\text{$\mathbb{T}$}\\
			\emph{upper}  & \hspace{.5cm}\text{$\mathbb{T}$}
		\end{align*}
		and is defined as\\[10pt]
		\begin{math}
			|source| = |target| \land\\
			\forall i: \forall x: x=source(i) \Rightarrow \exists y: y=target(i)\land lower\leq y-x\leq upper.
		\end{math}\\[10pt]
		The \emph{StrongDelayConstraint} is a stricter version of the \emph{DelayConstraint}, as it requires a bijective assignment between the source and target events, therefore additional events in target without matching source event are not allowed. Figure~\ref{fig:StrongDelayConstraintExample} shows an example of the \emph{StrongDelayConstraint}. The example is the same as in the previous constraint, but without the additional target events at 2, 5 and 8.2.
 		\begin{figure}
		 	\begin{tikzpicture}
		 	% source events
		 	\node[] at (0,-0.2){source};
		 	\draw (1, 0) -- (1, -0.4);
		 	\draw (5, 0) -- (5, -0.4);
		 	\draw (6, 0) -- (6, -0.4);
		 	
		 	% upper/lower 1
		 	\draw [fill=lightgray, lightgray] (3, -0.7) rectangle (4,-2.3);
		 	\node at (2, -0.5){lower};
		 	\node at (2, -1){upper};
		 	\draw[->] (1,-0.7) -- (3, -0.7);
		 	\draw[->] (1, -0.8) -- (4, -0.8);
		 	
		 	
		 	% upper/lower 2
		 	\draw [fill=lightgray, lightgray] (7, -0.7) rectangle (8,-2.3);
		 	\node at (6, -0.5){lower};
		 	\node at (6, -1){upper};
		 	\draw[->] (5,-0.7) -- (7, -0.7);
		 	\draw[->] (5, -0.8) -- (8, -0.8);
		 	
		 	% upper/lower 3
		 	\draw [fill=lightgray, lightgray] (8, -0.7) rectangle (9,-2.3);
		 	\node at (7, -1.1){lower};
		 	\node at (7, -1.6){upper};
		 	\draw[->] (6,-1.3) -- (8, -1.3);
		 	\draw[->] (6, -1.4) -- (9, -1.4);
		 	% target events
		 	\node[] at (0,-2.1){target};
		 	\draw (3.5, -1.9) -- (3.5, -2.3);
		 	\draw (7, -1.9) -- (7, -2.3);
		 	\draw (9, -1.9) -- (9, -2.3);
		 	
		 	\foreach \x in {0, 1, ..., 10}{
		 		\draw (\x, -3.1) -- (\x, -3.3);
		 		\node at(\x, -3.6) {\x};
		 	}
		 	
		 	\foreach \x in {0.5, 1.5, ..., 9.5}
		 	\draw (\x, -3.1) -- (\x, -3.2);
		 	\draw[->] (-0.1, -3.2) -- (10.2, -3.2);
		 	
		 	\end{tikzpicture}
		 	\caption{Example StrongDelayConstraint - $lower = 2$, $upper = 3$}
		 	\label{fig:StrongDelayConstraintExample}
		 \end{figure}
	\subsubsection{RepeatConstraint}
		The \emph{RepeatConstraint} also has 4 attributes
		\begin{align*}
			\emph{event} & \hspace{.5cm}\text{event set}\\
			\emph{lower} & \hspace{.5cm}\text{$\mathbb{T}$}\\
			\emph{upper} & \hspace{.5cm}\text{$\mathbb{T}$}\\
			\emph{span}	 & \hspace{.5cm}\text{$integer$}\\
		\end{align*}
		and is defined as\\[10pt]
		\begin{math}
			\forall X\leq event: |X|=span+1\Rightarrow lower \leq \lambda([X])\leq upper.
		\end{math}\\[10pt]
		As reminder, the $A\leq B$-operator over two sets of events $A, B$ describes, that $A$ is a sub-sequence of $B$, the $\lambda(A)$-function calculates the total length of all continuous intervals in $A$ and the $[A]$ returns the time interval between the oldest and newest event in $A$.\\
		The definition specifies that the length of each time interval containing $span+1$ consecutively events must be between $upper$ and $lower$.\\
		The idea behind this constraint is to define repeated occurrences of events, with the possibility of overlapping, specified by the \emph{span} attribute. After any event $x$, there are $span-1$ events and than the next event must be between $lower$ and $upper$ after $x$.\\
		Figure~\ref{fig:RepeatConstraintExample1} shows an example of the RepeatConstraint with the attributes $event=\{3,5,8,...\}$, $lower=upper=2$ and $span=1$. Because $lower$ is equal $upper$ and $span$ is 1, the events are following a strictly periodic pattern after the first event. Figure~\ref{fig:RepeatConstraintExample2} shows a more complex example with events at $\{0, 2, 4, 7, 9, 11,...\}$, $lower=4$, $upper=5$ and $span=2$. The $span$-attribute is 2, so the time distance between all subsequent events with an even index are considered, just like the subsequent events with an uneven index. 
		
		\begin{figure}
			\begin{tikzpicture}
				% events
				\node at(-1 , 0.2){Event}; 
				\foreach \x in {3,5,...,9}{
					\draw (\x, 0.4) -- (\x, 0);
					\draw[->] (\x, -0.6) -- (\x+2, -0.6);
					\node at(\x+1, -0.8){lower};
					\node at(\x+1, -0.4){upper};
				}
				\draw (11, 0.4) -- (11, 0);
				% time axis
				\foreach \x in {0,...,11}{
					\draw (\x, -1.4) -- (\x, -1.6);
					\node at(\x, -1.8){\x};
				}
				\draw[->] (-0.1, -1.5) -- (11.2, -1.5);
			\end{tikzpicture}
			\caption{Example RepeatConstraint - $lower = 2$, $upper = 2$, $span = 1$}
			\label{fig:RepeatConstraintExample1}
		\end{figure}
	
		\begin{figure}
			\begin{tikzpicture}
				\node at(-1, 0.2){Event};
				%eventreihe 1
				\foreach \x in {0, 4, 9} {
					\draw (\x,0) -- (\x, 0.4);
				}
				\foreach \x in {0, 9} {
					\node at (\x+2, 0) {upper};
					\draw[->] (\x, -0.2) -- (\x+5, -0.2);
					\draw[->] (\x, -0.4) -- (\x+4, -0.4);
					\node at (\x+2, -0.6) {lower};
				}
				\foreach \x in {4} {
					\node at (\x+2, -0.4) {upper};
					\draw[->] (\x, -0.6) -- (\x+5, -0.6);
					\draw[->] (\x, -0.8) -- (\x+4, -0.8);
					\node at (\x+2, -1) {lower};
				}
				%eventreihe 2
				\foreach \x in {2, 7, 11} {
					\draw (\x,-1.4) -- (\x, -1.8);
				}
				\foreach \x in {2, 11} {
					\node at (\x+2, -1.8) {upper};
					\draw[->] (\x, -2) -- (\x+5, -2);
					\draw[->] (\x, -2.2) -- (\x+4, -2.2);
					\node at (\x+2, -2.4) {lower};
				}
				\foreach \x in {7} {
					\node at (\x+2, -2.2) {upper};
					\draw[->] (\x, -2.4) -- (\x+5, -2.4);
					\draw[->] (\x, -2.6) -- (\x+4, -2.6);
					\node at (\x+2, -2.8) {lower};
				}
				%time axis
				\foreach \y in {-3.2}{
					\foreach \x in {0,...,11}{
						\draw (\x, \y) -- (\x, \y-0.2);
						\node at(\x, \y-0.4){\x};
					}
					\draw[->] (-0.1, \y-0.1) -- (11.2, \y-0.1);
				}
			\end{tikzpicture}
			\caption{Example RepeatConstraint - $lower = 4$, $upper = 5$, $span = 2$}
			\label{fig:RepeatConstraintExample2}
		\end{figure}
	
	\subsubsection{RepetitionConstraint}
		The \emph{RepetitionConstraint} has 5 attributes
		\begin{align*}
			\emph{event} & \hspace{.5cm}\text{event set}\\
			\emph{lower} & \hspace{.5cm}\text{$\mathbb{T}$}\\
			\emph{upper} & \hspace{.5cm}\text{$\mathbb{T}$}\\
			\emph{span}	 & \hspace{.5cm}\text{$integer$}\\
			\emph{jitter}& \hspace{.5cm}\mathbb{T}
		\end{align*}
		and is defined via the \emph{RepeatConstraint} and the \emph{StrongDelayConstraint} as\\[10pt]
		\begin{math}
			\exists X: RepeatConstraint(X, lower, upper, span) \land \\
			\text{\hspace{1cm}}StrongDelayConstraint(X, event, 0, jitter)
		\end{math}\\[10pt]
		where $X$ is a set of arbitrary time stamps, that follow the structure of the \emph{RepeatConstraint}(various(\emph{span}) loose periodic repetitions). The actual points in time of \emph{event} lay between the timestamps of $X$ and $jitter$ after that. For each point of time there is one, and only one, corresponding timestamp in $X$.
		Figure~\ref{fig:RepetitionConstraintExample} shows an example of the \emph{RepetitionConstraint} with the attributes $event=\{0.5, 3.3, 4.7, 7.6, 9.9, ...\}$, $lower=4$, $upper=5$, $span=2$ and $jitter=1$. The shown timestamps of $X$ are only one possibility and may change due to later elements of $event$.
		
		\begin{figure}
			\begin{tikzpicture}[thick]
				% time axis
				\foreach \y in {-4}{
					\foreach \x in {0,...,11}
					\draw (\x,\y) -- (\x,\y-0.2) node[anchor=north] {\x};
					\foreach \x in {0.5,1.5,...,10.5}
					\draw (\x,\y) -- (\x,\y-0.1);
					\draw[->] (0,\y-0.1) -- (11.2, \y-0.1);
				}
			\node[] at (-1, 0) {event};
			\foreach \x in {0.5, 3.3, 4.7, 7.6, 9.9}
				\draw (\x, 0.2) -- (\x, -0.2);
			
			\node[] at (-1, -0.9) {X};
			\foreach \x in {0, 3, 4.5, 7.5, 9}{
				\draw (\x, -0.7) -- (\x, -1.1);
				\draw[->] (\x, -1.3) -- (\x+1, -1.3);
				\node at (\x+0.5, -1.5) {jitter};
			}
			\foreach \x in {0, 4.5, 9}{
				\node at (\x+2, -1.8) {lower};
				\draw[|->] (\x, -2) -- (\x+4, -2);
				\draw[|->] (\x, -2.2) -- (\x+5, -2.2);
				\node at (\x+2, -2.4) {upper};
			}
			\foreach \x in {3, 7.5}{
				\node at (\x+2, -3) {lower};
				\draw[|->] (\x, -3.2) -- (\x+4, -3.2);
				\draw[|->] (\x, -3.4) -- (\x+5, -3.4);
				\node at (\x+2, -3.6) {upper};
			}
			\end{tikzpicture}
			\caption{Example RepetitionConstraint - $lower = 4$, $upper = 5$, $span = 2$, $jitter=1$}
			\label{fig:RepetitionConstraintExample}
		\end{figure}
		
	
	\subsubsection{SynchronizationConstraint}
		The \emph{SynchronizationConstraint} has 2 attributes
		\begin{align*}
			\emph{event} & \hspace{.5cm}\text{set of event sets, $|event|\geq 2$}\\
			\emph{tolerance} & \hspace{.5cm}\mathbb{T}
		\end{align*}
		and is defined via the \emph{DelayConstraint} as\\[10pt]
		\begin{math}
			\exists X: \forall i: DelayConstraint(X, event_i, 0, tolerance) \land\\
			\text{\hspace{1cm}}DelayConstraint(event_i, X, -tolerance, 0)
		\end{math}\\[10pt]
		$X$ is a set of arbitrary point in time and there must be at least one timestamp in each set of \emph{event}, that is between an element of $X$ and $tolerance$ after that. Also, for each element in any set of \emph{event}, there must be a matching element of $X$.\\
		In figure~\ref{fig:SynchronizationConstraintExample} is an example of the \emph{SynchronizationConstraint} with the attributes $event=\{\{0.5, 3, 7, 7.5\}, \{0.7, 2.5, 7.3, 7.8\}, \{1.2, 3.2, 3.3, 3.4, 7.6, 8.4\}\}$ and $tolerance = 1$. The first points in time of each element of event form the first cluster, the corresponding element of $X$ can be between $0.2$ and $0.5$. For simplification, only the latest possible value for the element of $X$ are shown, which is the first event of the synchronization cluster. In the second cluster of events it can bee seen that multiple timestamps from one element of $event$ can be associated with a single element of $X$. The third and fourth cluster show, that overlapping is also possible.
		\begin{figure}			
			\begin{tikzpicture}[thick]
				% tolerance rectangles
				\foreach \x in {0.5, 2.5, 7}
					\draw [fill=lightgray, lightgray] (\x, 0.2) rectangle (\x+1, -3.3);
				\draw [fill=lightgray, lightgray] (7.5, 0.2) rectangle (8.5, -3.5);
				\draw [fill=gray, gray] (7.5, 0.2) rectangle (8, -3.5);
				% time axis
				\foreach \y in {-4}{
					\foreach \x in {0,...,11}
						\draw (\x,\y) -- (\x,\y-0.2) node[anchor=north] {\x};
					\foreach \x in {0.5,1.5,...,10.5}
						\draw (\x,\y) -- (\x,\y-0.1);
					\draw[->] (0,\y-0.1) -- (11.2, \y-0.1);
				}
				
				\node at (-1, 0){$event_1$};
				\foreach \x in {0.5, 3, 7, 7.5}
					\draw (\x, 0.2) -- (\x, -0.2);
				
				\node at (-1, -1){$event_2$};
				\foreach \x in {0.7, 2.5, 7.3, 7.8}
					\draw (\x, -0.8) -- (\x, -1.2);
				
				\node at (-1, -2){$event_3$};
				\foreach \x in {1.2, 3.2, 3.3, 3.4, 7.6, 8.4}
					\draw (\x, -1.8) -- (\x, -2.2);
					
				\node at (-1, -3){$X$};
				\foreach \x in {0.5, 2.5}{
					\draw (\x, -2.8) -- (\x, -3.2);
					\draw[->] (\x, -3.3) -- (\x+1, -3.3);
					\node at (\x+0.5, -3.5){$tolerance$};
				}
				\foreach \x in {7}{
					\draw (\x, -2.8) -- (\x, -3.2);
					\draw[->] (\x, -3.3) -- (\x+1, -3.3);
				}
				\foreach \x in {7.5}{
					\draw (\x, -2.8) -- (\x, -3.2);
					\draw[->] (\x, -3.5) -- (\x+1, -3.5);
					\node at (\x+0.5, -3.7){$tolerance$};
				}
					
			\end{tikzpicture}
			\caption{Example SynchronizationConstraint - $tolerance = 1$}
			\label{fig:SynchronizationConstraintExample}
		\end{figure}
		
		
		
	\subsubsection{StrongSynchronizationConstraint}
		The \emph{StrongSynchronizationConstraint} has the same two attributes as the \emph{SynchronizationConstraint}
		\begin{align*}
			\emph{event} & \hspace{.5cm}\text{set of event sets, $|event|\geq 2$}\\
			\emph{tolerance} & \hspace{.5cm}\mathbb{T}
		\end{align*}
		and is defined as\\[10pt]
		\begin{math}
			\exists X: \forall i: StrongDelayConstraint(X, event_i, 0, tolerance)
		\end{math}\\[10pt]
		The \emph{StrongSynchronizationConstraint} is a stricter variant of the \emph{SynchronizationConstraint}, as it requires a bijective assignment between the elements of $X$ to one element of each set of $event$. For every $x\in X$, only one corresponding timestamp per set in $event$ is allowed, like seen in figure~\ref{fig:StrongSynchronizationConstraintExample}, which shows the same example as the one for the \emph{SynchronizationConstraint}, but the excess time stamps at $3.2$ and $3.3$ have been removed.
			\begin{figure}			
				\begin{tikzpicture}[thick]
				% tolerance rectangles
				\foreach \x in {0.5, 2.5, 7}
				\draw [fill=lightgray, lightgray] (\x, 0.2) rectangle (\x+1, -3.3);
				\draw [fill=lightgray, lightgray] (7.5, 0.2) rectangle (8.5, -3.5);
				\draw [fill=gray, gray] (7.5, 0.2) rectangle (8, -3.5);
				% time axis
				\foreach \y in {-4}{
					\foreach \x in {0,...,11}
					\draw (\x,\y) -- (\x,\y-0.2) node[anchor=north] {\x};
					\foreach \x in {0.5,1.5,...,10.5}
					\draw (\x,\y) -- (\x,\y-0.1);
					\draw[->] (0,\y-0.1) -- (11.2, \y-0.1);
				}
				
				\node at (-1, 0){$Event_1$};
				\foreach \x in {0.5, 3, 7, 7.5}
				\draw (\x, 0.2) -- (\x, -0.2);
				
				\node at (-1, -1){$Event_2$};
				\foreach \x in {0.7, 2.5, 7.3, 7.8}
				\draw (\x, -0.8) -- (\x, -1.2);
				
				\node at (-1, -2){$Event_3$};
				\foreach \x in {1.2, 3.4, 7.6, 8.4}
				\draw (\x, -1.8) -- (\x, -2.2);
				
				\node at (-1, -3){$X$};
				\foreach \x in {0.5, 2.5}{
					\draw (\x, -2.8) -- (\x, -3.2);
					\draw[->] (\x, -3.3) -- (\x+1, -3.3);
					\node at (\x+0.5, -3.5){$tolerance$};
				}
				\foreach \x in {7}{
					\draw (\x, -2.8) -- (\x, -3.2);
					\draw[->] (\x, -3.3) -- (\x+1, -3.3);
				}
				\foreach \x in {7.5}{
					\draw (\x, -2.8) -- (\x, -3.2);
					\draw[->] (\x, -3.5) -- (\x+1, -3.5);
					\node at (\x+0.5, -3.7){$tolerance$};
				}
			
			\end{tikzpicture}
			\caption{Example StrongSynchronizationConstraint - $tolerance = 1$}
			\label{fig:StrongSynchronizationConstraintExample}
		\end{figure}
		
		
	\subsubsection{ExecutionTimeConstraint}
		The \emph{ExecutionTimeConstraints} takes 6 attributes
		\begin{align*}
			\emph{start} & \hspace{.5cm}\text{set of events}\\
			\emph{stop} & \hspace{.5cm}\text{set of events}\\
			\emph{preempt} & \hspace{.5cm}\text{set of events}\\
			\emph{resume} & \hspace{.5cm}\text{set of events}\\
			\emph{lower} & \hspace{.5cm}\mathbb{T}\\
			\emph{upper} & \hspace{.5cm}\mathbb{T}\\
		\end{align*}
		and is defined as\\[10pt]
		\begin{math}
			\forall x\in start: lower\leq \lambda([x..stop]\setminus[preempt..resume]) \leq upper
		\end{math}\\[10pt]
		The interval constructor $\forall x\in start: [x..stop]$ defines the time interval between each point in time of $start$ until the next element of $stop$, excluding the $stop$ timestamp. $[preempt..resume]$, which is removed from the considered interval length, defines the intervals between each element of preempt until the next timestamp of resume.\\
		The Idea behind this constraint is to test the run time of a task, without counting interruptions.\\
		Figure~\ref{fig:ExecutionTimeConstraintExample} shows an example of the \emph{ExecutionTimeConstraints} with $start=\{1\}$, $end=\{7\}$, $preempt=\{2, 5\}$ and $resume = \{3, 6.5\}$. Therefore, $[start..end]$ spans the interval from time 1 to 7 with the length of 6 and $[preempt..resume]$ spans two intervals, 2 to 3 and 5 to 6.5 with the length 1 and 1.5. As result, $\lambda([x..stop]\setminus[preempt..resume])$ for $x = 1$ is 3.5 and the constraint is fulfilled, if, and only if, lower is equal or \emph{lower} than 3.5 and \emph{upper} is greater than that.\\
		\begin{figure}
   			\begin{tikzpicture}[thick]
				% time axis
				\foreach \y in {-4}{
					\foreach \x in {0,...,11}
					\draw (\x,\y) -- (\x,\y-0.2) node[anchor=north] {\x};
					\foreach \x in {0.5,1.5,...,10.5}
					\draw (\x,\y) -- (\x,\y-0.1);
					\draw[->] (0,\y-0.1) -- (11.2, \y-0.1);
				}
				% start
				\node at (-1, -0.2) {start};
				\draw[] (1,0) -- (1,-0.4);
				% end
				\node at (-1, -0.8) {end};
				\draw[] (7, -0.6) -- (7, -1);
				%preempt
				\node at (-1, -1.4) {preempt};
				\draw[] (2, -1.2) -- (2, -1.6);
				\draw[] (5, -1.2) -- (5, -1.6);
				%resume
				\node at (-1, -2) {resume};
				\draw[] (3, -1.8) -- (3, -2.2);
				\draw[] (6.5, -1.8) -- (6.5, -2.2);
				
				\node at (-1, -2.8) {[start..end]};
				\draw [fill=lightgray, lightgray] (1, -2.6) rectangle (7, -3.0);
				
				\node at (-1, -3.4) {[preempt..resume]};
				\draw [fill=lightgray, lightgray] (2, -3.2) rectangle (3, -3.6);
				\draw [fill=lightgray, lightgray] (5, -3.2) rectangle (6.5, -3.6);
			\end{tikzpicture}
			\caption{Example ExecutionTimeConstraint}
			\label{fig:ExecutionTimeConstraintExample}
		\end{figure}
		
	\subsubsection{OrderConstraint}
		The \emph{OrderConstraint} takes two attributes
		\begin{align*}
			\emph{source} & \hspace{.5cm}\text{set of events}\\
			\emph{target} & \hspace{.5cm}\text{set of events}
		\end{align*}
		and is defined as\\[10pt]
		\begin{math}
			|source| = |target| \land \forall i:\exists x: x=source(i)\Rightarrow \exists y: y=target(i)\land < x \leq y
		\end{math}\\[10pt]
		This constraints ensures the order of events, so that the $i$-th event of $target$ is after the $i$-th event of $source$. Also, the number of events in \emph{source} and \emph{target} must be equal.\\
		Figure~\ref{fig:OrderConstraintExample} visualizes an example of the \emph{OrderConstraint} with $source = \{1, 4, 6, 7\}$ and $target = \{3, 5, 9, 9.5\}$. The constraint is fulfilled, because the number of elements is equal and each $i$-th timestamp in \emph{target} is later that the $i$-th timestamp of $source$.
		\begin{figure}
			\begin{tikzpicture}[thick]
				% time axis
				\foreach \y in {-1.5}{
					\foreach \x in {0,...,11}
					\draw (\x,\y) -- (\x,\y-0.2) node[anchor=north] {\x};
					\foreach \x in {0.5,1.5,...,10.5}
					\draw (\x,\y) -- (\x,\y-0.1);
					\draw[->] (0,\y-0.1) -- (11.2, \y-0.1);
				}
				% start
				\node at (-1, -0.2) {source};
				\foreach \x in {1, 4, 6, 7}{
					\draw[] (\x,0) -- (\x,-0.4);
				}
				\node at (1, 0.2) {1};
				\node at (4, 0.2) {2};
				\node at (6, 0.2) {3};
				\node at (7, 0.2) {4};
				% end
				\node at (-1, -0.8) {target};
				\foreach \x in {3, 5, 9, 9.5}{
					\draw[] (\x,-0.6) -- (\x,-1);
				}
				\node at (3, -1.2) {1};
				\node at (5, -1.2) {2};
				\node at (9, -1.2) {3};
				\node at (9.5, -1.2) {4};
			\end{tikzpicture}
			\caption{Example OrderConstraint}
			\label{fig:OrderConstraintExample}
		\end{figure}
		
		
	\subsubsection{ComparisonConstraint}
		The \emph{ComparisonConstraint} is significant different to all previous and following constraints, as it does not describe the behavior of events and only compares two time expressions. It takes 3 attributes
		\begin{align*}
			\emph{leftOperand} 	& \hspace{.5cm}\mathbb{T}\\
			\emph{rightOperand} & \hspace{.5cm}\mathbb{T}\\
			\emph{operator}		& \hspace{.5cm} \text{comparisonOperator}(\in \{LessThanOrEqual, LessThan,\\
								& \hspace{3.5cm} GreaterThanOrEqual, GreaterThan, Equal\})
		\end{align*}
		The definition is pretty straight forward as it only applies the given operator to the operands:\\[10pt]
		\begin{math}
			ComparisonConstraint(leftOperand, rightOperand, LessThanOrEqual)\\
				\Leftrightarrow leftOperand \leq rightOperand\\[5pt]
			ComparisonConstraint(leftOperand, rightOperand, LessThan)\\
				\Leftrightarrow leftOperand < rightOperand\\[5pt]
			ComparisonConstraint(leftOperand, rightOperand, GreaterThanOrEqual)\\
				\Leftrightarrow leftOperand \geq rightOperand\\[5pt]
			ComparisonConstraint(leftOperand, rightOperand, GreaterThan)\\
				\Leftrightarrow leftOperand > rightOperand\\[5pt]
			ComparisonConstraint(leftOperand, rightOperand, Equal)\\
				\Leftrightarrow leftOperand = rightOperand
		\end{math}\\[10pt]
		Due to the simplicity of this constraint, no explicit example is given.
		
	\subsubsection{SporadicConstraint}
		The \emph{SporadicConstraint} takes 5 attributes
		\begin{align*}
			\emph{event} 	& \hspace{.5cm}\text{set of events}\\
			\emph{lower} 	& \hspace{.5cm}\mathbb{T}\\
			\emph{upper} 	& \hspace{.5cm}\mathbb{T}\\
			\emph{jitter}	& \hspace{.5cm}\mathbb{T}\\
			\emph{minimum}	& \hspace{.5cm}\mathbb{T}
		\end{align*}
		and is defined as combination of the \emph{RepetitionConstraint} and the \emph{RepeatConstraint} as\\[10pt]
		\begin{math}
			RepetitionConstraint(event, lower, upper, 1, jitter)\\
			\land RepeatConstraint(event, minimum, \infty, 1)
		\end{math}\\[10pt]
		The second part of the definition, using the \emph{RepeatConstraint}, ensures that all events in \emph{event} lay at least \emph{minimum} apart. The application of the \emph{RepetitionConstraint} generates a set of events $X$, that lay between $lower$ and $upper$ apart from each other. For each point in time in $X$, there must be exactly one timestamp in \emph{event}, that is not before the corresponding element of $X$ and not later than \emph{jitter} after that.\\
		Figure~\ref{fig:SporadicConstraintExample} shows a possible application of the \emph{SporadicConstraint} with the attributes $lower=2$, $upper=2.5$, $jitter=1$, $minimum=2$ and $event=\{1, 3.5, 6, 8.2, 10.5,...\}$. Like in the \emph{RepetitionConstraint}, the exact position of the timestamps in $X$ is variable and may need to be changed due to later entries in $event$.
		%TODO minimum in example
		\begin{figure}
			\begin{tikzpicture}[thick]
			% time axis
			\foreach \y in {-2.5}{
				\foreach \x in {0,...,11}
				\draw (\x,\y) -- (\x,\y-0.2) node[anchor=north] {\x};
				\foreach \x in {0.5,1.5,...,10.5}
				\draw (\x,\y) -- (\x,\y-0.1);
				\draw[->] (0,\y-0.1) -- (11.2, \y-0.1);
			}
			% event
			\node at (-1, 0.4) {event};
			\foreach \x in {1, 3.5, 6, 8.2, 10.5}{
				\draw[] (\x,0.2) -- (\x,+0.6);
				%minimum
				\draw[->] (\x, 0) -- (\x+2, 0);
				\node at (\x+1, 0.2) {min.};
			}
			% X
			\node at (-1, -0.8) {X};
			\foreach \x in {0.8, 2.8, 5.3, 7.8, 10}{
				\draw[] (\x,-0.6) -- (\x,-1);
				%jitter
				\draw[->] (\x, -0.25) -- (\x+1, -0.25);
				\node at (\x+0.5, -0.5) {jitter};
			}
			%lower/upper
			\foreach \x in {0.8, 5.3}{
				\node at (\x+1, -1.1){lower};
				\draw[->] (\x, -1.3) -- (\x+2, -1.3);
				\draw[->] (\x, -1.5) -- (\x+2.5, -1.5);
				\node at (\x+1, -1.7){upper};
			}
			\foreach \x in {2.8, 7.8}{
				\node at (\x+1, -1.6){lower};
				\draw[->] (\x, -1.8) -- (\x+2, -1.8);
				\draw[->] (\x, -2) -- (\x+2.5, -2);
				\node at (\x+1, -2.2){upper};
			}

			\end{tikzpicture}
			\caption{Example SporadicConstraint - $lower=2$, $upper=2.5$, $jitter=1$, $minimum=2$}
			\label{fig:SporadicConstraintExample}
		\end{figure}
	
	\subsubsection{PeriodicConstraint}
		The \emph{PeriodicConstraint} takes 4 attribute
		\begin{align*}
			\emph{event} 	& \hspace{.5cm}\text{set of events}\\
			\emph{period} 	& \hspace{.5cm}\mathbb{T}\\
			\emph{jitter}	& \hspace{.5cm}\mathbb{T}\\
			\emph{minimum}	& \hspace{.5cm}\mathbb{T}
		\end{align*}
		and defines a specialized form of the \emph{SporadicConstraint}\\[10pt]
		\begin{math}
			SporadicConstraint(event, period, period, jitter, minimum)
		\end{math}\\[10pt]
		The variable timestamps in the set $X$ are now following a strictly periodic pattern, where subsequent elements of this set lay exactly \emph{period} apart. Each element of \emph{event} lays between one element of $X$ and \emph{jitter} after that. Again, there must be bijective mapping between the elements of \emph{event} and $X$.\\
		In figure~\ref{fig:PeriodicConstraintExample}, the \emph{PeriodicConstraint} with the attributes $period=3$, $jitter=1$, $minimum=2.5$ and $event = \{1.2, 4.0, 8, 10.6, ...\}$ is visualized. The timestamps of $X$ lay exactly \emph{period} apart and the $events$ behind that in the previously described way. Also, the minimum time distance between all points of time in \emph{event} is \emph{minimum}.
		\begin{figure}
			\begin{tikzpicture}[thick]
			% time axis
			\foreach \y in {-2}{
				\foreach \x in {0,...,11}
				\draw (\x,\y) -- (\x,\y-0.2) node[anchor=north] {\x};
				\foreach \x in {0.5,1.5,...,10.5}
				\draw (\x,\y) -- (\x,\y-0.1);
				\draw[->] (0,\y-0.1) -- (11.2, \y-0.1);
			}
			% event
			\node at (-1, 0.6) {event};
			\foreach \x in {1.2, 4.0, 8, 10.6}{
				%timestamp
				\draw[] (\x,0.2) -- (\x,+0.6);
				%minimum
				\draw[->] (\x, 0) -- (\x+2.5, 0);
				\node at (\x+1.25, 0.2) {min.};
			}
			% X
			\node at (-1, -0.8) {X};
			\foreach \x in {1, 4, 7, 10}{
				%timestamp
				\draw[] (\x,-0.6) -- (\x,-1);
				%jitter
				\draw[->] (\x, -0.25) -- (\x+1, -0.25);
				\node at (\x+0.5, -0.5) {jitter};
				%period
				\draw[->] (\x, -1.25) -- (\x+3, -1.25);
				\node at (\x+1.5, -1.5) {period};
			}
			\end{tikzpicture}
			\caption{Example PeriodicConstraint - $period=3$, $jitter=1$, $minimum=2.5$}
			\label{fig:PeriodicConstraintExample}
		\end{figure}
		
		
	\subsubsection{PatternConstraint}
		The \emph{PatternConstraint} takes 5 attributes
		\begin{align*}
			\emph{event} 	& \hspace{.5cm}\text{set of events}\\
			\emph{period} 	& \hspace{.5cm}\mathbb{T}\\
			\emph{offset}	& \hspace{.5cm} \text{set of }\mathbb{T}\\
			\emph{jitter}	& \hspace{.5cm}\mathbb{T}\\
			\emph{minimum}	& \hspace{.5cm}\mathbb{T}
		\end{align*}
		and is defined as \\[10pt]
		\begin{math}
			\exists X: PeriodicConstraint(X, period, 0, 0)\\
			\text{\hspace{.5cm}} \land \forall i: DelayContraint(X, event, offset_i, offset_i+jitter)\\
			\text{\hspace{.5cm}} \land RepeatConstraint(event, minimum, \infty, 1)
		\end{math}\\[10pt]
		This constraint can be understood as a modification of the \emph{PeriodicConstraint}, as it describes periodic behavior, but not from single events, but from groups of $|offset|$ subsequent events, that follow specific time distances (specified by $offset$) after the strictly periodic timestamps of $X$.\\
		Figure~\ref{fig:PatternConstraintExample} shows an application of the \emph{PeriodicConstraint} with attributes $period=5$, $offset=\{1, 2, 2.5\}$, $jitter=0.5$, $minimum=0.5$ and\\
		$event = \{1.2, 2.2, 2.8, 6, 7, 8, 11.5, 12, 12.5, ...\}$. Like in the previous describes constraint, the exact position of all points in time of $X$ may change due to later timestamps of $event$.
		
		\begin{figure}
			\begin{tikzpicture}[thick]
			% time axis
			\foreach \y in {-3.5}{
				\foreach \x in {0,...,14}
				\draw (\x,\y) -- (\x,\y-0.2) node[anchor=north] {\x};
				\foreach \x in {0.5,1.5,...,13.5}
				\draw (\x,\y) -- (\x,\y-0.1);
				\draw[->] (0,\y-0.1) -- (14.2, \y-0.1);
			}
			% event
			\node at (-1, 0.6) {event};
			\foreach \x in {1.2, 2.2, 2.8, 6, 7, 8, 11.5, 12, 12.5}{
				%timestamp
				\draw[] (\x,0.2) -- (\x,+0.6);
				%minimum
				\draw[->] (\x, 0.8) -- (\x+.5, 0.8);
				\node at (\x+0.25, 1) {min.};
			}
			% X
			\node at (-1, -2.2) {X};
			\foreach \x in {0,5, 10}{
				%timestamp
				\draw[] (\x,-2) -- (\x,-2.4);
				%offset 1
				\draw[->] (\x, -0.5) -- (\x+1, -0.5);
				\node at (\x+0.5, -0.7) {offset$_1$};
				\draw[->] (\x+1, -0.5) -- (\x+1.5, -0.5);
				\node at (\x+1.25, -0.3) {jitter};
				%offset 2
				\draw[->] (\x, -1) -- (\x+2, -1);
				\node at (\x+1, -1.2) {offset$_2$};
				\draw[->] (\x+2, -1) -- (\x+2.5, -1);
				\node at (\x+2.25, -0.8) {jitter};
				%offset 3
				\draw[->] (\x, -1.5) -- (\x+2.5, -1.5);
				\node at (\x+1.25, -1.7) {offset$_3$};
				\draw[->] (\x+2.5, -1.5) -- (\x+3, -1.5);
				\node at (\x+2.75, -1.3) {jitter};
				%period
				\draw[->] (\x, -2.6) -- (\x+5, -2.6);
				\node at (\x+2.5, -2.83) {period};
			}
			\end{tikzpicture}
			\caption{Example PatternConstraint - $period=5$, $offset=\{1, 2, 2.5\}$, $jitter=0.5$, $minimum=0.5$}
			\label{fig:PatternConstraintExample}
		\end{figure}
		
	\subsubsection{ArbitraryConstraint}
		The \emph{ArbitraryConstraint} takes 3 attributes
		\begin{align*}
			\emph{event} 	& \hspace{.5cm}\text{set of events}\\
			\emph{minimum}	& \hspace{.5cm}\text{set of }\mathbb{T}\\
			\emph{maximum}	& \hspace{.5cm}\text{set of }\mathbb{T}
		\end{align*}
		where $|minimum|=|maximum|$. It is defined as \\[10pt]
		\begin{math}
			\forall i: RepeatConstraint(event, minimum_i, maximum_i, i)
		\end{math}\\[10pt]
		The Idea behind the \emph{ArbitraryConstraint} is to describe the time distance between each event and several following events. The first entry of \emph{minimum} and \emph{maximum} define the distance between every event and it direct successor. The second entries, where the \emph{span} attribute of the  \emph{RepeatConstraint} is 2, set the distance between one event and its next but one successor and so on.\\
		Figure~\ref{fig:ArbitraryConstraintExample} shows an example of the \emph{ArbitraryConstraint} with the attributes $minimum=\{1,2,3\}$, $maximum=\{5,6,7\}$ and $event=\{1, 2, 3, 5, 8, 10, ...\}$. The time distances between subsequent events with 0, 1 and 2 skipped events are shown in table~\ref{tab:ArbitraryConstraintExampleTable}, the relevant distances are written in \textbf{bold} font. Apparently, the time distances are matching the ranges, given by the $minimum$- and $maximum$ attribute.\\
		\begin{table}
			\begin{tabular}{|c|c|c|c|c|c|c|}
				\hline
				& 1 & 2 & 3 & 5 & 8 & 10 \\
				\hline
				1 & 0 & \textbf{1} & \textbf{2} & \textbf{4} & 7 & 9 \\
				\hline
				2 &  & 0 & \textbf{1} & \textbf{3} & \textbf{6} & 8 \\
				\hline
				3 &  &  & 0 & \textbf{2} & \textbf{5} & \textbf{7} \\
				\hline
				5 &  &  &  & 0 & \textbf{3} & \textbf{5} \\
				\hline
				8 &  &  &  &  & 0 & \textbf{2} \\
				\hline
				10 &  &  &  &  &  & 0 \\
				\hline
			\end{tabular}
			\centering
			\caption{Time distances as seen in figure~\ref{fig:ArbitraryConstraintExample}}
			\label{tab:ArbitraryConstraintExampleTable}
		\end{table}

		\begin{figure}
			\begin{tikzpicture}[thick]
			% time axis
			\foreach \y in {-6}{
				\foreach \x in {0,...,11}
				\draw (\x,\y) -- (\x,\y-0.2) node[anchor=north] {\x};
				\foreach \x in {0.5,1.5,...,10.5}
				\draw (\x,\y) -- (\x,\y-0.1);
				\draw[->] (0,\y-0.1) -- (11.2, \y-0.1);
			}
			% event
			\node at (-1, 0.4) {event};
			\node at (-1, -0.85) {$minimum_1$};
			\node at (-1, -1.65) {$maximum_1$};
			
			\node at (-1, -2.55) {$minimum_2$};
			\node at (-1, -3.45) {$maximum_2$};
			
			
			\node at (-1, -4.35) {$minimum_2$};
			\node at (-1, -5.25) {$maximum_2$};
			\foreach \x in {1,2,3,5,8,10}{
				%timestamp
				\draw[] (\x,0.2) -- (\x,+0.6);
			}
			% minimum 1
			\draw[->] (1, -0.6) -- (2, -0.6);
			\draw[->] (2, -0.7) -- (3, -0.7);
			\draw[->] (3, -0.8) -- (4, -0.8);
			\draw[->] (5, -0.9) -- (6, -0.9);
			\draw[->] (8, -1.0) -- (9, -1.0);
			\draw[->] (10, -1.1) -- (11, -1.1);
			% maximum 1
			\draw[->] (1, -1.4) -- (6, -1.4);
			\draw[->] (2, -1.5) -- (7, -1.5);
			\draw[->] (3, -1.6) -- (8, -1.6);
			\draw[->] (5, -1.7) -- (10, -1.7);
			\draw[->] (8, -1.8) -- (13, -1.8);
			\draw[->] (10, -1.9) -- (15, -1.9);
			
			% minimum 2
			\draw[->] (1, -2.3) -- (3, -2.3);
			\draw[->] (2, -2.4) -- (4, -2.4);
			\draw[->] (3, -2.5) -- (5, -2.5);
			\draw[->] (5, -2.6) -- (7, -2.6);
			\draw[->] (8, -2.7) -- (10, -2.7);
			\draw[->] (10, -2.8) -- (12, -2.8);
			
			% maximum 2
			\draw[->] (1, -3.2) -- (7, -3.2);
			\draw[->] (2, -3.3) -- (8, -3.3);
			\draw[->] (3, -3.4) -- (9, -3.4);
			\draw[->] (5, -3.5) -- (11, -3.5);
			\draw[->] (8, -3.6) -- (14, -3.6);
			\draw[->] (10, -3.7) -- (16, -3.7);
			
			% minimum 3
			\draw[->] (1, -4.1) -- (4, -4.1);
			\draw[->] (2, -4.2) -- (5, -4.2);
			\draw[->] (3, -4.3) -- (6, -4.3);
			\draw[->] (5, -4.4) -- (8, -4.4);
			\draw[->] (8, -4.5) -- (11, -4.5);
			\draw[->] (10, -4.6) -- (13, -4.6);
			
			% maximum 3
			\draw[->] (1, -5.0) -- (8, -5.0);
			\draw[->] (2, -5.1) -- (9, -5.1);
			\draw[->] (3, -5.2) -- (10, -5.2);
			\draw[->] (5, -5.3) -- (12, -5.3);
			\draw[->] (8, -5.4) -- (15, -5.4);
			\draw[->] (10, -5.5) -- (17, -5.5);
			
			
			\end{tikzpicture}
			\caption{Example ArbitraryConstraint - $period=5$, $offset=\{1, 2, 2.5\}$, $jitter=0.5$, $minimum=0.5$}
			\label{fig:ArbitraryConstraintExample}
		\end{figure}
		
	\subsubsection{BurstConstraint}
		The \emph{BurstConstraint} takes 4 attributes
		\begin{align*}
			\emph{event} 			& \hspace{.5cm}\text{set of events}\\
			\emph{length}			& \hspace{.5cm} \mathbb{T}\\
			\emph{maxOccurrences}	& \hspace{.5cm} integer\\
			\emph{minimum}			& \hspace{.5cm} \mathbb{T}
		\end{align*}
		and is defined as \\[10pt]
		\begin{math}
			RepeatConstraint(event, length, \infty, maxOccurrences)\\
			\text{\hspace{.5cm}}\land RepeatConstraint(event, minimum, \infty, 1)
		\end{math}\\[10pt]
		 The idea of this constraint is to describe the maximum number of events that may occur in a time interval of the given $length$. Additionally all subsequent event must be at least \emph{minimum} apart. Therefore, the intuition is different to the AUTOSAR \emph{BurstPatternEventTriggering}, where clusters of events are described. A complete comparison of these constraints will be done in section~\ref{comparisonConstraints}.\\
		 In figure~\ref{fig:BurstConstraintExample} the \emph{BurstConstraint} with the attributes $length=5$, $maxOccurrences=3$, $minimum=0.8$ and $event = \{1,2,3,7,8,9\}$ is visualized. In every interval of the length 5, there are three or less events, also all subsequent events lay at least $0.8$ apart. Therefore, the constraint is fulfilled.
 		\begin{figure}
		 	\begin{tikzpicture}[thick]
			 	% time axis
			 	\foreach \y in {-2}{
			 		\foreach \x in {0,...,11}
			 		\draw (\x,\y) -- (\x,\y-0.2) node[anchor=north] {\x};
			 		\foreach \x in {0.5,1.5,...,10.5}
			 		\draw (\x,\y) -- (\x,\y-0.1);
			 		\draw[->] (0,\y-0.1) -- (11.2, \y-0.1);
			 	}
			 	% event
			 	\node at (-1, 0.4) {event};
			 	\foreach \x in {1, 2, 3, 7, 8, 9}{
			 		%timestamp
			 		\draw[] (\x,0.2) -- (\x,+0.6);
			 		%minimum
			 		\draw[->] (\x, 0) -- (\x+0.8, 0);
			 		\node at (\x+0.4, -0.2) {min.};
			 	}
		 		% length
		 		\node at (-1, -1.2) {length};
		 		\draw[->] (1, -1) -- (6, -1);
	 			\draw[->] (2, -1.2) -- (7, -1.2);
 				\draw[->] (3, -1.4) -- (8, -1.4);
 				\draw[->] (7, -1) -- (12, -1);
 				\draw[->] (8, -1.2) -- (13, -1.2);
 				\draw[->] (9, -1.4) -- (14, -1.4);
		 	\end{tikzpicture}
		 	\caption{Example BurstConstraint - $length=5$, $maxOccurences=3$ $minimum=0.8$}
		 	\label{fig:BurstConstraintExample}
		 \end{figure}
	
	\subsubsection{ReactionConstraint}
		The \emph{ReactionConstraint} takes 3 attributes
		\begin{align*}
			\emph{scope} 	& \hspace{.5cm} EventChain\\
			\emph{minimum}	& \hspace{.5cm} \mathbb{T}\\
			\emph{maximum}	& \hspace{.5cm} \mathbb{T}
		\end{align*}
		and is defined as \\[10pt]
		\begin{math}
			\forall x\in scope.stimulus: \exists y\in scope.response:\\
			\text{\hspace{.5cm}}x.color=y.color\\
			\text{\hspace{.5cm}}\land (\forall y'\in scope.response: y'.color=y.color\Rightarrow y\leq y')\\
			\text{\hspace{.5cm}}\land minimum \leq y-x \leq maximum
		\end{math}\\[10pt]
		The definition says, that after every event $x$ of $scope.stimulus$, there is an event $y$ in $scope.response$ with the same color. The time distance between these events must be at least $minimum$ and at most $maximum$. Additional events with the same color as $y$ in $scope.response$ are allowed, if they lay behind $y$.  The definition implies, that additional events with other colors are allowed in $scope.response$, but not in $scope.stimulus$ and every color is only allowed once in $scope.stimulus$.\\
		A visualized example with the attributes $minimum=1$, $maximum=3$,\\
		$scope.stimulus=\{(1, red), (5, green), (5.5, purple), (8, orange)\}$ and $scope.response=\{(0.8, blue), (2.1, red), (4.5, blue), (6.6, purple),
		(6.7, purple), (9.5, purple), (7.5, green), \\
		(10, orange)\}$ can be seen in figure~\ref{fig:ReactionConstraint}. The red $stimulus$-event is followed by the red $response$-event at 2.1, the green $stimulus$ event at 5 by the $response$ event at 7.5 and so on. The blue $response$ events at 1 and 4.5 are additional events without an associated stimulus event. The purple events at 6.7 and 9.5 are the second and third event of this color in $scope.response$ and therefore, their time distance to the $stimulus$ event with the same color is irrelevant.
 		\begin{figure}
			\begin{tikzpicture}[thick]
				% time axis
				\foreach \y in {-3}{
					\foreach \x in {0,...,11}
					\draw (\x,\y) -- (\x,\y-0.2) node[anchor=north] {\x};
					\foreach \x in {0.5,1.5,...,10.5}
					\draw (\x,\y) -- (\x,\y-0.1);
					\draw[->] (0,\y-0.1) -- (11.2, \y-0.1);
				}
				% stimulus
				\node at (-1, 0.4) {stimulus};
				\draw[red, very thick]    (1  ,0.2) -- (1,+0.6);
				\draw[green, very thick]  (5  ,0.2) -- (5,+0.6);
				\draw[purple, very thick] (5.5,0.2) -- (5.5,+0.6);
				\draw[orange, very thick] (8  ,0.2) -- (8,+0.6);
				
				% minimum
				\draw[->, red] (1, -0.2) -- (2, -0.2);
				\node at (1.5, 0) {min.};
				\draw[->, green] (5, -0.2) -- (6, -0.2);
				\node at (5.5, 0) {min.};
				\draw[->, purple] (5.5, -1.1) -- (6.5, -1.1);
				\node at (6, -0.9) {min.};
				\draw[->, orange] (8, -0.2) -- (9, -0.2);
				\node at (8.5, 0) {min.};
				%maximum
				\draw[->, red] (1, -0.4) -- (4, -0.4);
				\node at (1.5, -0.6) {max.};
				\draw[->, green] (5, -0.4) -- (8, -0.4);
				\node at (5.5, -0.6) {max.};
				\draw[->, purple] (5.5, -1.3) -- (8.5, -1.3);
				\node at (6, -1.5) {max.};
				\draw[->, orange] (8, -0.4) -- (11, -0.4);
				\node at (8.5, -0.6) {max.};
				%response
				\node at (-1, -2.1) {response};
				\draw[blue, very thick]    (.8  ,-1.9) -- (.8,-2.3);
				\draw[blue, very thick]    (4.5  ,-1.9) -- (4.5,-2.3);
				\draw[red, very thick]    (2.1  ,-1.9) -- (2.1,-2.3);
				\draw[green, very thick]  (7.5  ,-1.9) -- (7.5,-2.3);
				\draw[purple, very thick] (6.6,-1.9) -- (6.6,-2.3);
				\draw[purple, very thick] (6.7,-1.9) -- (6.7,-2.3);
				\draw[purple, very thick] (9.5,-1.9) -- (9.5,-2.3);
				\draw[orange, very thick] (10  ,-1.9) -- (10,-2.3);
			\end{tikzpicture}
			\caption{Example ReactionConstraint - $minimum=1$, $maximum=3$}
			\label{fig:ReactionConstraint}
		\end{figure}
		
	
	\subsubsection{AgeConstraint}
		The \emph{AgeConstraint} takes 3 attributes
		\begin{align*}
			\emph{scope} 	& \hspace{.5cm} EventChain\\
			\emph{minimum}	& \hspace{.5cm} \mathbb{T}\\
			\emph{maximum}	& \hspace{.5cm} \mathbb{T}
		\end{align*}
		and is defined as \\[10pt]
		\begin{math}
			\forall y\in scope.response: \exists x\in scope.stimulus:\\
			\text{\hspace{.5cm}}x.color=y.color\\
			\text{\hspace{.5cm}}\land (\forall x'\in scope.stimulus: x'.color=x.color\Rightarrow x'\leq x)\\
			\text{\hspace{.5cm}}\land minimum \leq y-x \leq maximum
		\end{math}\\[10pt]
		The \emph{AgeConstraint} is a turned around counterpart to the \emph{ReactionConstraint}. For every event of $scope.response$, there must be an event with the same color in $scope.stimulus$, that is between $minimum$ and $maximum$ older than the $response$ event. Additional events are only allowed in $scope.stimulus$, and only before the event that matches with a $response$ event.\\
		Figure~ \ref{fig:AgeConstraint} shows an application of the \emph{AgeConstraint} with the attributes $minimum=1$, $maximum=3$, $scope.stimulus=\{(0.8, blue), (1, red), (2, green), (4.5, green),\\
		(5, green), (5.5, purple), (8, orange)\}$ and $scope.response=\{(3.5, red), (7.5, green),\\
		 (6.6, purple), (10, orange)\}$. The blue timestamps are additional events without matching events in $scope.response$.
 		\begin{figure}
			\begin{tikzpicture}[thick]
				% time axis
				\foreach \y in {-3}{
					\foreach \x in {0,...,11}
					\draw (\x,\y) -- (\x,\y-0.2) node[anchor=north] {\x};
					\foreach \x in {0.5,1.5,...,10.5}
					\draw (\x,\y) -- (\x,\y-0.1);
					\draw[->] (0,\y-0.1) -- (11.2, \y-0.1);
				}
				% stimulus
				\node at (-1, 0.4) {stimulus};
				\draw[red, very thick]    (1  ,0.2) -- (1,+0.6);
				\draw[green, very thick]  (2  ,0.2) -- (2,+0.6);
				\draw[green, very thick]  (5  ,0.2) -- (5,+0.6);
				\draw[purple, very thick] (5.5,0.2) -- (5.5,+0.6);
				\draw[orange, very thick] (8  ,0.2) -- (8,+0.6);
				\draw[blue, very thick] (0.8, 0.2) -- (0.8,0.6);
				\draw[blue, very thick] (4.5, 0.2) -- (4.5,0.6);
				
				% minimum
				\draw[->, red] (3.5, -0.2) -- (2.5, -0.2);
				\node at (3, 0) {min.};
				\draw[->, green] (7.5, -0.2) -- (6.5, -0.2);
				\node at (7, 0) {min.};
				\draw[->, purple] (6.6, -1.1) -- (5.6, -1.1);
				\node at (6.1, -0.9) {min.};
				\draw[->, orange] (10, -1.1) -- (9, -1.1);
				\node at (9.5, -0.9) {min.};
				%maximum
				\draw[->, red] (3.5, -0.4) -- (0.5, -0.4);
				\node at (3, -0.6) {max.};
				\draw[->, green] (7.5, -0.4) -- (4.5, -0.4);
				\node at (7, -0.6) {max.};
				\draw[->, purple] (6.6, -1.3) -- (3.6, -1.3);
				\node at (6.1, -1.5) {max.};
				\draw[->, orange] (10, -1.3) -- (7, -1.3);
				\node at (9.5, -1.5) {max.};
				%response
				\node at (-1, -2.1) {response};
				\draw[red, very thick]    (3.5  ,-1.9) -- (3.5,-2.3);
				\draw[green, very thick]  (7.5  ,-1.9) -- (7.5,-2.3);
				\draw[purple, very thick] (6.6,-1.9) -- (6.6,-2.3);
				%\draw[purple, very thick] (6.7,-1.9) -- (6.7,-2.3);
				%\draw[purple, very thick] (9.5,-1.9) -- (9.5,-2.3);
				\draw[orange, very thick] (10  ,-1.9) -- (10,-2.3);
			\end{tikzpicture}
			\caption{Example AgeConstraint - $minimum=1$, $maximum=3$}
			\label{fig:AgeConstraint}
		\end{figure}
		
	\subsubsection{OutputSynchronizationConstraint}
		The \emph{OutputSynchronizationConstraint} takes 2 attributes
		\begin{align*}
			\emph{scope} 	& \hspace{.5cm} \text{Set of }EventChain\\
			\emph{tolerance}	& \hspace{.5cm} \mathbb{T}
		\end{align*}
		where all elements of \emph{scope} have the same $stimulus$. It is defined as \\[10pt]
		\begin{math}
			\forall x\in scope_1.stimulus: \exists t: \forall i: \exists y\in scope_i.response:\\
			\text{\hspace{.5cm}} x.color = y.color\\
			\text{\hspace{.5cm}}\land (\forall y'\in scope_i.response: y'.color=y.color \Rightarrow y\leq y')\\
			\text{\hspace{.5cm}}\land 0\leq y-t\leq tolerance
		\end{math}\\[10pt]
		The definition says, that after each event $x$ in $scope_1.stimulus$, there must be a interval with the length of $tolerance$, in which every $scope_i.response$ must have an event $y$ with the same color as $x$. Additional response events with this color are only allowed after $y$.
		Figure~\ref{fig:OutputSynchronizationConstraint} shows an example of the \emph{OutputSynchronizationConstraint} with the attributes $tolerance = 1$,\\
		 $scope[1].stimulus=scope[2].stimulus=scope[3].stimulus=\{(1, red), (4, green), (5, purple)\}$,\\
		$scope[1].response=\{(2, red), (6, purple), (6.2, purple), (8.2, green)\}$,\\
		$scope[2].response=\{(2.6, red), (6.2, purple), (8, green), (10.5, green)\}$,\\
		$scope[3].response=\{(2.3, red), (6.5, purple), (8.5, green)\}$.\\
		
 		\begin{figure}
			\begin{tikzpicture}[thick]
				% time axis
				\foreach \y in {-3.5}{
					\foreach \x in {0,...,11}
					\draw (\x,\y) -- (\x,\y-0.2) node[anchor=north] {\x};
					\foreach \x in {0.5,1.5,...,10.5}
					\draw (\x,\y) -- (\x,\y-0.1);
					\draw[->] (0,\y-0.1) -- (11.2, \y-0.1);
				}
				% stimulus
				\node at (-1, 0.4) {scope\textcolor{gray}{[1]}.stimulus};
				\draw[red, very thick]    (1  ,0.2) -- (1,+0.6);
				\draw[green, very thick]  (4  ,0.2) -- (4,+0.6);
				\draw[purple, very thick] (5,0.2) -- (5,+0.6);
				
				\node at(-1, -0.6) {scope[1].response};
				\draw[red, very thick]    (2, -0.4) -- (2, -0.8);
				\draw[green, very thick]  (8.3, -0.4) -- (8.3, -0.8);
				\draw[purple, very thick] (6, -0.4) -- (6, -0.8);
				\draw[purple, very thick] (6.2, -0.4) -- (6.2, -0.8);
				
				\node at(-1, -1.2) {scope[2].response};
				\draw[red, very thick]    (2.6, -1) -- (2.6, -1.4);
				\draw[green, very thick]    (8, -1) -- (8, -1.4);
				\draw[green, very thick]    (10.5, -1) -- (10.5, -1.4);
				\draw[purple, very thick]    (6.2, -1) -- (6.2, -1.4);
				
				\node at(-1, -1.8) {scope[3].response};
				\draw[red, very thick]    (2.3, -1.6) -- (2.3, -2);
				\draw[green, very thick]  (8.5, -1.6) -- (8.5, -2);
				\draw[purple, very thick] (6.5, -1.6) -- (6.5, -2);
				
				\node at (-1, -2.8) {t \& tolerance};
				\foreach \i in {2, 6, 8} {
					\draw[very thick] (\i, -2.6) -- (\i, -3);
					\draw[->] (\i, -2.8) -- (\i+1, -2.8);
				}

			\end{tikzpicture}
			\caption{Example OutputSynchronizationConstraint - $tolerance=1$}
			\label{fig:OutputSynchronizationConstraint}
		\end{figure}
		
	\subsubsection{InputSynchronizationConstraint}
	The \emph{InputSynchronizationConstraint} takes 2 attributes
	\begin{align*}
		\emph{scope} 	& \hspace{.5cm} \text{Set of }EventChain\\
		\emph{tolerance}& \hspace{.5cm} \mathbb{T}
	\end{align*}
	where all elements of \emph{scope} have the same $response$. It is defined as \\[10pt]
	\begin{math}
		\forall y\in scope_1.response: \exists t: \forall i: \exists x\in scope_i.stimulus:\\
		\text{\hspace{.5cm}} x.color = y.color\\
		\text{\hspace{.5cm}}\land (\forall x'\in scope_i.stimulus: x'.color=x.color \Rightarrow x\leq x')\\
		\text{\hspace{.5cm}}\land 0\leq x-t\leq tolerance
	\end{math}\\[10pt]
	The \emph{InputSynchronizationConstraint} is a counterpart of the \emph{OutputSynchronizationConstraint}, as the \emph{stimulus} events must be synchronized, not the \emph{response} events.\\
	Figure~\ref{fig:InputSynchronizationConstraint} contains an example of the \emph{InputSynchronizationConstraint} with the attributes $tolerance=1$\\
	$scope[1].stimulus=\{(1, red), (1.5, green), (4.6, green), (8, purple)\}$\\
	$scope[2].stimulus=\{(1.2, red), (4, green), (8.3, purple), (8.5, purple)\}$\\
	$scope[3].stimulus=\{(1.5, red), (4, green), (8.9, purple)\}$\\
	$scope[1].response=scope[2].response=scope[3].response=\{(2.5, red), (6, green), (10, purple)\}$\\
	\begin{figure}
		\begin{tikzpicture}[thick]
			% time axis
			\foreach \y in {-3.5}{
				\foreach \x in {0,...,11}
				\draw (\x,\y) -- (\x,\y-0.2) node[anchor=north] {\x};
				\foreach \x in {0.5,1.5,...,10.5}
				\draw (\x,\y) -- (\x,\y-0.1);
				\draw[->] (0,\y-0.1) -- (11.2, \y-0.1);
			}
			% stimulus
			\node at (-1, 0.4) {scope[1].stimulus};
			\draw[red, very thick]    (1  ,0.2) -- (1,+0.6);
			\draw[green, very thick]  (1.5  ,0.2) -- (1.5,+0.6);
			\draw[green, very thick]  (4.6  ,0.2) -- (4.6,+0.6);
			\draw[purple, very thick] (8,0.2) -- (8,+0.6);
			
			\node at (-1, -0.2) {scope[2].stimulus};
			\draw[red, very thick]    (1.2,0) -- (1.2, -0.4);
			\draw[green, very thick]  (4  ,0) -- (4,-0.4);
			\draw[purple, very thick] (8.3,  0) -- (8.3,-0.4);
			\draw[purple, very thick] (8.5,  0) -- (8.5,-0.4);
			
			\node at (-1, -0.8) {scope[3].stimulus};
			\draw[red, very thick]    (1.5,-0.6) -- (1.5, -1);
			\draw[green, very thick]  (4,  -0.6) -- (4,-1);
			\draw[purple, very thick] (8.9,-0.6) -- (8.9,-1);
			
			\node at(-1, -1.8) {scope[1].response};
			\draw[red, very thick]    (2.5, -1.6) -- (2.5, -2);
			\draw[green, very thick]  (6, -1.6) -- (6, -2);
			\draw[purple, very thick] (10, -1.6) -- (10, -2);
			
			\node at (-1, -2.8) {t \& tolerance};
			\foreach \i in {1, 4, 8} {
				\draw[very thick] (\i, -2.6) -- (\i, -3);
				\draw[->] (\i, -2.8) -- (\i+1, -2.8);
			}
			
		\end{tikzpicture}
		\caption{Example InputSynchronizationConstraint - $tolerance=1$}
		\label{fig:InputSynchronizationConstraint}
	\end{figure}
			
\subsection{Comparison TADL2 - AUTOSAR Timing Extension}
\label{comparisonConstraints}
	As said before, the \emph{TADL2 Timing Constraints} and the \emph{AUTOSAR Timing Extension} are compatible in parts and many of the \emph{AUTOSAR Timing Extension} can be expressed as equivalent combinations of the \emph{TADL2 Timing Constraints}. In \cite{TIMMO2USE}, the relation between these constraints is shown, but this comparison is based on an outdated version of the AUTOSAR Timing Extensions and some of the constraints have been updated, therefore each of the \emph{AUTOSAR Timing Extensions} will be listed in this chapter and it will be explained, if and how they can be expressed using \emph{TADL2 Timing Constraints}.\\
	The types used in the AUTOSAR Timing Extension are similar to the ones in TADL2. TADL2 \emph{Events} are called \emph{TimingDescriptionEvent} in AUTOSAR, the same goes for \emph{EventChains}, which are called \emph{TimingDescriptionEventChains}. A larger difference can be seen in the definition of time. While TADL2 defines time as real numbers, the time definition used in the AUTOSAR Timing Extension can also be multidimensional, for example when the real time and the angle of the crankshaft is regarded. For simplification, all timestamps are considered as real numbers in the following, but an extension to multidimensional time stamps is possible, as AUTOSAR requires a strict order between all time stamps. \emph{Executable entities} as defined in the AUTOSAR Timing Extension describe things, that can be executed, for example a function. For the timing constraints, only striking point in times of these entities are relevant, for example the start or end points. It should be noted, that the set of TADL2 timing constraints are not equal to the AUTOSAR Timing Extension and that there are constraint, that cannot be expressed using the corresponding counterpart.

	\subsubsection{PeriodicEventTriggering}
		The \emph{PeriodicEventTriggering} defined in AUTOSAR with the attributes\\ $(\textcolor{gray}{event}, period, jitter, minimumInterArrivalTime)$ is equivalent to the \emph{TADL2} \emph{PeriodicConstraint} with the same attributes.
		
	\subsubsection{SporadicEventTriggering}
		AUTOSARs \emph{SporadicEventTriggering} with the attributes\\
		 $(\textcolor{gray}{event}, jitter, maximumInterArrivalTime,  minimumInterArrivalTime, period)$ is equivalent to the \emph{TADL2} \emph{SporadicConstraint}, but the names of the attributes are different:\\
		\begin{math}
			lower=period\\
			upper=maximumInterArrivalTime\\
			jitter=jitter\\
			minimum=minimumInterArrivalTime
		\end{math}
	
	\subsubsection{ConcretePatternEventTriggering}
		The idea behind the \emph{ConcretePatternEventTriggering} from AUTOSAR is the same as behind TADL2s \emph{PatternConstraint}, but subtleties are different. Both define a periodic behavior and offsets, that describe time distances between the periods and the actual events. The main difference is the \emph{jitter} attribute. In AUTOSARs \emph{ConcretePatternEventTriggering}, the \emph{patternJitter} attribute defines the allowed deviation of the start points of the periodic repetitions, as in TADL2 the $jitter$-value describes the deviation between the offsets and the actual event.\\
		The \emph{ConcretePatternEventTriggering} from AUTOSAR additionally defines an \emph{patternLength} attribute, which describes the length of the intervals, in which the clusters of events will occur. It is constrained by\\[10pt]
		\begin{math}
			0\leq max(offset)\leq patternLength\\
			\land patternLength + patternJitter < patternPeriod
		\end{math}\\[10pt]
		The \emph{patternLength} attribute can not be described with TADL2 timing constraints, as it would require to determine the distance of filtered events, which is not possible with the TADL2 constraints.\\
		TADL2 defines the \emph{minimum} attribute for the \emph{PatternConstraint} that describes the minimal time distance between subsequent events. In AUTOSAR, this must be described by using the \emph{ArbitraryEventTriggering}, where $minimumDistance_1$ is \emph{minimum} and $maximumDistance_1$ is $\infty$.
		
	\subsubsection{BurstPatternEventTriggering}
		The \emph{BurstPatternEventTriggering} as defined in AUTOSAR and TADL2s \emph{BurstConstraint} share the same target, as they define a maximum number of events that may occur in a specific time interval, but the \emph{BurstPatternEventTriggering} is way more complex. Additionally to the attributes of TADL2s \emph{BurstConstraint}, that define the \emph{length} of the time interval, the \emph{maxOccurrences} of the event in this interval and the minimal time between subsequent events, the \emph{BurstPatternEventTriggering} allows to define the minimal number of events in the interval and periodic repetitions of the burst interval.\\
		Every set of attributes fulfilling the TADL2 \emph{BurstConstraint} fulfill the \emph{BurstPatternEventTriggering}, when the attributes are renamed to the AUTOSAR equivalents ($length\rightarrow patternLength$, $maxOccurences\rightarrow maxNumberOfOccurences$, $minimum\rightarrow minimumInterArrivalTime$). This does not work the other way around, even if the attributes, that are in \emph{BurstPatternEventTriggering} and not in \emph{BurstConstraint} are unused. The reason for this is, that the observed interval must start at an event in the TADL2 \emph{BurstConstraint}, in the \emph{BurstPatternEventTriggering} those can start in any point of time.
		
	\subsubsection{ArbitraryEventTriggering}
		AUTOSARs \emph{ArbitraryEventTriggering} is similar to the \emph{ArbitraryConstraint} as defined in TADL2, but \emph{ArbitraryEventTriggering} allows to set a list of \emph{ConfidenceInterval}s, to describe the probability, how far the events may lay apart. These probabilities can not be expressed in TADL2.
		
	\subsubsection{LatencyTimingConstraint}
		The \emph{LatencyTimingConstraint} of AUTOSAR takes 5 attributes, a latency type $latencyConstraintType\in \{age, reaction\}$, three time values $maximum$, $minimum$ and $nominal$ and an event chain $scope$, consisting of the stimulus and response events. The $nominal$-value is not relevant for a formal definition of the Constraint, therefore there is no counterpart to this in the TADL2 Constraints. If the $latencyConstraintType$ of the \emph{LatencyTimingConstraint} is $age$, it can be expressed with $AgeConstraint$ defined in TADL2. The \emph{LatencyTimingConstraint} with the $latencyConstraintType$ $reaction$ is equivalent to the $reactionConstraint$.
	
	\subsubsection{AgeConstraint}
		The goal of the \emph{AgeConstraint} is the same for the \emph{LatencyTimingConstraint} with the \emph{latencyConstraintType} $age$, but scope is defined with the type \emph{TimingDescriptionEvent}, not as \emph{TimingDescriptionEventChain}. The reason for this is, that it can be used in earlier development phases, as it is an abstract type. There is no differentiation like this in TADL2 and the \emph{AgeConstraint} is semantically the same in AUTOSAR and TADL2.
		%TODO unterschiede-> latency ist auf chain definiert, ageConstraint auf einzelnem Event-> genauer darstellen
		
	\subsubsection{SynchronizationTimingConstraint}
		The \emph{SynchronizationTimingConstraint} is similar to the \emph{SynchronizationConstraint}, the \emph{StrongSynchronizationConstraint}, the \emph{OutputSynchronizationConstraint}, the \emph{InputSynchronizationConstraint} or combinations of them, depending on the attributes. \ref{ComparisonSynchronizationConstraints} shows, with which attributes the \emph{SynchronizationTimingConstraint} is equivalent to which TADL2 Constraint(s).
		\begin{table}
			\begin{tabular}{|c|c|c|c|c|}
				\hline
				\makecell{event\\Occurrence-\\Kind} 	& \makecell{scope/\\scopeEvent}  & \makecell{synchronization-\\ConstraintType} 	& tolerance & TADL2 Constraints\\
				\hline
				\makecell{multiple\\Occurrences} & scopeEvent & \emph{not set} & tolerance & \makecell{SynchronizationConstraint\\\hspace{.5cm}(scopeEvent, tolerance)}\\
				\hline
				\makecell{single\\Occurrences}  & scopeEvent & \emph{not set} & tolerance & \makecell{StrongSynchronizationConstraint\\\hspace{.5cm}(scopeEvent, tolerance)}\\
				\hline
				\makecell{multiple\\Occurrences}  & scope & \makecell{response\\Synchronization} & tolerance & \makecell{OutputSynchronizationConstraint\\\hspace{.5cm}(scope, tolerance)\\ $\land$ SynchronizationConstraint\\\hspace{.5cm}(scope.response, tolerance)}\\
				\hline
				\makecell{single\\Occurrences}  & scope & \makecell{response\\Synchronization} & tolerance & \makecell{OutputSynchronizationConstraint\\\hspace{.5cm}(scope, tolerance)\\ $\land$ StrongSynchronizationConstraint\\\hspace{.5cm}(scope.response, tolerance)}\\
				\hline
				\makecell{multiple\\Occurrences}  & scope & \makecell{stimulus\\Synchronization} & tolerance & \makecell{InputSynchronizationConstraint\\\hspace{.5cm}(scope, tolerance)\\ $\land$ SynchronizationConstraint\\\hspace{.5cm}(scope.stimulus, tolerance)}\\
				\hline
				\makecell{single\\Occurrences}  & scope & \makecell{stimulus\\Synchronization} & tolerance & \makecell{InputSynchronizationConstraint\\\hspace{.5cm}(scope, tolerance)\\ $\land$ SynchronizationConstraint\\\hspace{.5cm}(scope.stimulus, tolerance)}\\
				\hline
			\end{tabular}
			\caption{SynchronizationTimingConstraint $\Leftrightarrow$ TADL2 Constraints}
			\label{ComparisonSynchronizationConstraints}
		\end{table}
	
	\subsubsection{SynchronizationPointConstraint}
		The \emph{SynchronizationPointConstraint} describes, that a list of executables and a list of events or executable entities, defined in \emph{sourceEec} and \emph{sourceEvent},  must finish and occur, before the executables and events in \emph{targetEec} and \emph{targetEvent} will start or occur. There is no counterpart to this in the TADL2 constraints.
		
	\subsubsection{OffsetTimingConstraint}
		The \emph{OffsetTimingConstraint}, defined in AUTOSAR Timing Extensions, is semantically the same as the TADL2 \emph{DelayConstraint}, just some attributes are named differently. The \emph{maximum} attribute of the \emph{OffsetTimingConstraint} is named \emph{upper} and the \emph{minimum} attribute \emph{lower} in the \emph{DelayConstraint}.
		
	\subsubsection{ExecutionOrderConstraint}
		The goal of \emph{ExecutionOrderConstraint} of the AUTOSAR Timing Extensions is to describe the order of events or the execution order of executable entities, defined as \emph{orderedElement} attribute. There is no constraint in TADL2 that describes exactly this, but if the \emph{ExecutionOrderConstraint} is used to describe only the order of events, it can be described as \\[10pt]
		\begin{math}
			OrderConstraint(orderedElement_1, orderedElement_2)\\
			\land ... \land\\
			OrderConstraint(orderedElement_{n-1}, orderedElement_n)
		\end{math}\\[10pt]
		If the \emph{ExecutionOrderConstraint} is used for executable entities, each executable entity must be turned into one or more events to be described via TADL2 Constraints, depending on the other attributes. For example, if the attribute \emph{executionOrderConstraintType} is set to \emph{ordinaryEOC}, the start and finish points of the entities define the observed events.
		
	\subsubsection{ExecutionTimeConstraint}
		The idea behind the \emph{ExecutionTimeConstraint} is similar in AUTOSAR and TADL2. Both describe the minimal and maximal allowed run time of an executable entity, but the TADL2 constraint is capable of not counting interruptions in the execution. On the other hand, the AUTOSAR constraint offers the possibility, to determine if external calls should be included in the run time or not. Also, AUTOSARs \emph{ExecutionTimeConstraint} is defined directly on an executable entity and the TADL2 constraint on events describing the start, stop, preemption and resume timestamps. Therefore the executable entity must be turned into these events to express the AUTOSAR \emph{ExecutionTimeConstraint} via TADL2 constraints. The start and stop times need to be turned into their obvious counterparts. If the run time of external calls should not be counted (the \emph{executionTimeType} attribute is set to \emph{net}), the start and stop points of these external calls must be moved into the \emph{preempt} and \emph{resume} event sets. If the \emph{executionTimeType} attribute is set to \emph{gross}, external calls are ignored and the complete time from the start to the end of the execution is counted, therefore the \emph{preempt} and \emph{resume} event remain empty.
		% TODO: AUTOSAR definiert "interruption and external calls."

  %%!TEX root = thesis.tex

\chapter{Monitor Models}
\label{chapter-Monitor-Models}

\section{Runtime Verification}
	Monitoring the AUTOSAR Timing Extensions is the goal of this thesis. As monitoring plays a major role in runtime verification, a short overview of this will be given. The definitions of \cite{RuntimeVerification} are used, in which \emph{Runtime Verification} is a technique that can detect deviations between the run of a system and its formal specification by checking correctness properties. A \emph{run}, which might also be called \emph{trace}, is sequence of the system states, which might be infinite and an \emph{execution} is an finite prefix of this run. A \emph{monitor} reads the trace and decides, whether it fulfills the correctness properties or violates them.\\
	A distinction is made between \emph{offline} and \emph{online} monitoring. Offline monitoring is using a stored trace, that has been recorded before. Therefore, the complete trace (or the complete part of the trace, that should be analyzed) is known in the analysis. Online monitoring checks the properties, while the system is running, which means that the analysis must be done incrementally. Because of memory and time limitations, not all previous states can be read again in online monitoring, more detailed contemplations on the limitations of online monitors will be given in \ref{chapter-monitorability}.
	%TODO Überleitung!


\section{TeSSLa}

TeSSLa (\textbf{Te}mporal \textbf{S}tream-based \textbf{S}pecification \textbf{La}nguage) is a functional programming language, build for runtime verification of streams. In TeSSLa, \textbf{streams} are defined as traces of events, each event consists of one data value from a data set $\mathbb D$ and a time value from a discrete time domain $\mathbb T$. This time domain needs a total order and subsequent timestamps must have increasing time values. A TeSSLa Specification can have several streams with different data sets, but each of these streams must use the same time domain $\mathbb T$, which timestamps are increasing over all streams. Each stream can have only one event per timestamp, but it is possible to have events on different streams at the same timestamp.\\
A distinction between synchronous and asynchronous streams is made. A set of synchronous streams have events in the exact same time stamps, events in asynchronous streams do not have this restriction. It is easy to see, that synchronous streams are a subset of the asynchronous ones, therefore we will only use asynchronous streams from now on.\\
In TeSSLa, calculations are done, when new events are arriving. Based on the specification, output streams are generated with events on the same timestamps as the used input streams, but filtering is possible, where not all input events produce output events. With the \emph{delay}-operator, it is possible to create new timestamps. This possibility will take a large role in this thesis, more on that later.\\
At the timestamps, in which events arrived and calculations are done, you only have direct access to the youngest event of each stream, but with the use of the \emph{last}-operator, which can be used recursively, the event before that can be accessed. The \emph{lift}-operator applies a function, which is defined on data values $\mathbb D$, on each event of one or more streams. Similar to this, the \emph{slift}-operator (signal lift) first applies the given function, when there was at least one event of each input stream. The \emph{time}-operator returns the time value of an event.\\
% TODO formal definition of streams
% TODO \mathbb D bei Daten, aktuell keine Daten und bisher keine Daten
% TODO describe options of monitoring-> online, offline, non-intrusive


\section{Finite Transducers}

%In ~\cite{TeSSLa} werden verschiedene Fragmente von TeSSLa beschrieben, die unterschiedliche Mächtigkeiten haben und äquivalent zu verschiedenen Transduktormodellen sind. Im Fragment \emph{TeSSLa$_{bool}$} sind die Datentypmengen der Ströme auf boolesche Werte beschränkt, als Operatoren sind nur der oben genannte \emph{last}-Operator, der \emph{lift}-Operator
  %!TEX root = thesis.tex

\chapter{Monitorability}
\label{chapter-monitorability}
	Because of time and memory restrictions, online monitoring on a possible infinite run of a system is only reasonable, if several constraints are fulfilled. Therefore, the term of \emph{Finite Monitorability} will be introduced, which ensures, that a property can be supervised using an online monitor.

\section{Finite Monitorability}
	\subsection{Timestamps}
		\label{monitorability_timestamps}
		As we consider streams that can be infinite, the time value of events can also grow into infinity. This is problematic, because it leads to infinite memory and runtime requirements, which cannot be meet, especially not in the context of online monitoring. Therefore, the time domain $\mathbb{T}$ must be restricted by the following constraints:
		\begin{itemize}
			\item
				$\mathbb{T}$ must be discrete.
			\item
				The first used timestamp has the value $t_0=0$
			\item
				All used timestamps must be smaller than $t_{max}$.\\
				$t_{max}$ must be big enough, so it is not reached in practical use \footnote{for example, a 64-bit unsigned integer variable is enough, to cover nanoseconds for 584.55 years}.
			\item
				The distance between two subsequent time values is small enough to observe the wanted constraints.
		\end{itemize}
		%Additionally, a table $\tau \in \mathbb{T}'\times\mathbb{T}$ is defined, where $\mathbb{T}'$ is a set of indices. $\tau$ stores exactly the %timestamps, which are part of the current state of the monitor, which will be introduced in \ref{monitorability_state}. 
	\begin{figure}
		\begin{tikzpicture}
			\node[] (inputRight){$\mathbb{D}_1\times\mathbb{T}, ..., \mathbb{D}_n\times\mathbb{T}$};
			\node[draw, below of=inputRight] (fRight){$f$};
			\node[below of=fRight] (stateRight){$\mathbb{D}_{state}\times\mathbb{T}$};
			\node[draw, below of=stateRight] (delayRight){$Delay$};
			\node[below of=delayRight] (stateDelayRight){$\mathbb{D}_{state}\cup\{timeout\}\times\mathbb{T}$};
			\node[draw, below of=stateDelayRight] (gRight){$g$};
			\node[below of=gRight] (outputRight){$\{true_{until}, false\}\times\mathbb{T}$};
			\draw[->] (inputRight) -- (fRight);
			\draw[->] (fRight) -- (stateRight);
			\draw[->] (stateRight) -- (delayRight);
			\draw[->] (delayRight) -- (stateDelayRight);
			\draw[->] (stateDelayRight) -- (gRight);
			\draw[->] (gRight) -- (outputRight);
			\node [below of=outputRight, align=center] (h0){Finite Monitorability\\with delay};

			
			\node[left of = inputRight] (h1){};
			\node[left of = h1] (h2){};
			\node[left of = h2] (h3){};
			\node[left of = h3] (h4){};
			
			\node[left of = h4] (inputLeft){$\mathbb{D}_1\times\mathbb{T}, ..., \mathbb{D}_n\times\mathbb{T}$};
			\node[draw, below of=inputLeft] (fLeft){$f$};
			\node[below of = fLeft] (h5){};
			\node[below of=h5] (stateLeft){$\mathbb{D}_{state}\times\mathbb{T}$};
			\node[, below of=stateLeft] (delayLeft){};
			\node[draw, below of=delayLeft] (gLeft){$g$};
			\node[below of=gLeft] (outputLeft){$\{true_{until}, false\}\times\mathbb{T}$};
			\draw[->] (inputLeft) -- (fLeft);
			\draw[->] (fLeft) -- (stateLeft);
			\draw[->] (stateLeft) -- (gLeft);
			\draw[->] (gLeft) -- (outputLeft);
			\node [below of=outputLeft] {Finite Monitorability};
		\end{tikzpicture}
		\centering
		\caption{Overview Finite Monitorability - with or without \emph{delay}}
		\label{fig:OverviewMonitorability}
	\end{figure}
	\subsection{Finite Monitorability}
		%TODO mention figure
		For the definitions of streams and functions defined on them, TeSSLa-like syntax is used. Also, some standard TeSSLa functions are used in the definitions. 
		\subsubsection{Input Streams}
			Let $S_1, S_2, ..., S_n$ be input streams with\\
			$\forall i:$ $S_i=(\mathbb{T}\cdot \mathbb{D}_i)^\omega\cup(\mathbb{T}\cdot \mathbb{D}_i)^+\cup(\mathbb{T}\cdot \mathbb{D}_i)^*\cdot(\mathbb{T}_\infty\cup\mathbb{T}\cdot\{\bot\})$ and\\
			All types $D_i$ have a finite size.
			
		\subsubsection{State Stream}
			\label{monitorability_state}
			Let $S_{state}$ with\\
			$S_{state}= (\mathbb{T}\cdot \mathbb{D}_{state})^+\cup(\mathbb{T}\cdot \mathbb{D}_{state})^*$\\
			be a state stream, where $\mathbb{D}_{state}$ has a finite size.\\
			Further let $f: S_1 \times S_2 \times ... \times S_n \times S_{state}\rightarrow S_{state}\times \mathbb{T}$ a state transition function, which defines the state stream in an incremental fashion:\\
			$\forall t\in \mathbb T \exists i\in \{1,2,...,n\}: S_i(t)\in\mathbb D_i$\\
			$\rightarrow S_{state}(t)= f(S_1(t), S_2(t), ..., S_n(t), last(S_{state}, merge(S_1, S_2, ..., S_n))(t))$\\
			The runtime of $f$ is in $\mathcal{O}(1)$.
		\subsubsection{Output Stream}
			Let $S_{output}= (\mathbb{T}\cdot \{true_{until}, false\})^+\cup(\mathbb{T}\cdot \{true_{until}, false\})^*$\\
			be the output stream, which is defined via a function\\
			$g: \mathbb{D}_{state}\times \mathbb{T}\rightarrow \{true_{until}, false\}\times \mathbb{T}$\\
			The runtime of $g$ is in $\mathcal{O}(1)$.
		\subsubsection{Evaluation}
			A property of a set of streams is called \emph{Finite Monitorable}, if a function $f$ with type $\mathbb{D}_{state}$ and a function $g$ exist, which fulfill the characteristics called above, and which outputs $true_{until}$, as long as the property is fulfilled and $false$, in any other case. It should be noted that these definitions are \emph{timestamp conservative}, because the streams $S_{state}$ and $S_{output}$ can only change their data value at the timestamps of input events.
		% TODO transducer
			
	\subsection{Finite Monitorability with Delay}
		%TODO mention figure
		Not all of the TADL2 constraints can be monitored in a \emph{timestamp conservative}. For example, the \emph{RepeatConstraint} with the attributes $lower=upper=4$ and $span=1$ expects subsequent events to have a time distance of $4$. If one event is missing, the output of a timestamp conservative monitor would still be $true_{until}$, until the next input event arrives. Therefore, the monitor cannot not check the constraint correctly. Because of this problem, the definition of \emph{Finite Monitorability} is expanded by the ability of introducing new timestamps.
		\subsubsection{Input Streams}
			The definition of the input streams are unchanged.
		\subsubsection{State Stream}
			The function $f$ remains unchanged, but the state stream $S_{state}$ is expanded by an \emph{timeout} value, which is inserted after a specific period of time, in which no input event has arrived.
		\subsubsection{Delay}
			A \emph{delay generator}, which copies the states produced by the function $f$ and inserts a \emph{timeout} state into the state stream, if there was no input event for a specific period of time, is added to the definition. The length of this period depends on the current state of the monitor.
		\subsubsection{Output Stream}
			The output function $g$ is expanded by the \emph{timeout} value:\\
			$g: (\mathbb{D}_{state}\cup\{timeout\})\times \mathbb{T}\rightarrow \{true_{until}, false\}\times \mathbb{T}$\\
			The definition of the output stream $S_{output}$ remains unchanged.
		\subsubsection{Evaluation}
			A property of a set of streams is called \emph{Finite Monitorable with Delay}, if a function $f$ with type $\mathbb{D}_{state}$, a delay generator and a function $g$ exist, which fulfill the characteristics called above, and which outputs $true_{until}$, as long as the property is fulfilled and $false$, in any other case.
			
		
		
	
  %!TEX root = thesis.tex
\chapter{Analysis of the Monitorability of the TADL2 Timing Constraints}
\label{chapter-TADL2}

\section{DelayConstraint}
\section{StrongDelayConstraint}
\section{RepeatConstraint}
\section{RepetitionConstraint}
\section{SynchronizationConstraint}
\section{StrongSynchronizationConstraint}
\section{ExecutionTimeConstraint}
\section{OrderConstraint}
\section{ComparisonConstraint}
\section{SporadicConstraint}
\section{PeriodicConstraint}
\section{PatternConstraint}
\section{ArbitraryConstraint}
\section{BurstConstraint}
\section{ReactionConstraint}
\section{AgeConstraint}
\section{OutputSynchronizationConstraint}
\section{InputSynchronizationConstraint}
  %!TEX root = thesis.tex

\chapter{Implementierungen}
\label{chapter-implementation}
	\section{Implementation Of The TADL2 Constraints}
	In this chapter, the implementation of the monitor of each constraint will be explained. This is done by giving a short documentation of each monitor is given. Additionally, the worst case memory usage and the worst case and average run time per event is determined. In section~\ref{sec:performance}, each monitor is run on traces, which were generated to match the constraints with specific parameters, to evaluate which performance can be expected in a practical usage of the implementation.\\
	All implementations have in common that they consist of 2 or 3 sections, similar to the state transition, delay (if needed) and output as defined in chapter~\ref{chapter-monitorability}. These sections are the basis for the analysis of the computational complexity, because the generated state defines the required memory capacity and the state transition function, the output function and the calculation of the required delay define the required time per timestamp with input events.\\
	The implementations are programmed and tested for version 1.0.12 of the TeSSLa JAR archive.
	 
\subsection{DelayConstraint}
	The implementation of the \emph{DelayConstraint} monitor stores a list of $source$ events, which did not have a matching $target$ event yet as state. This list is expanded by every $source$ event, which is appended at the end of the list. If a $target$ event occurs, all matching $source$ events (possibly none) are removed from the list. Like stated in section~\ref{monitorability_DelayConstraint}, this list can grow infinitely in worst cases, when the time domain defined in an uncountable way. In these worst cases, an infinite number of $source$ events may occur, before any event can be removed from the list, when a matching $target$ event occurs.\\
	The used TeSSLa version is using integer values as time domain, therefore it is countable and the list cannot grow infinitely, because at most $upper$ $stimulus$ events need to be stored and the largest possible length of the list is linear dependent of the parameter $upper$. Because this list is the only growable memory usage, the algorithm is in $\mathcal{O}(upper)$ in terms of memory.\\
	In timestamps with a $target$ event, all events in the list, which are in the right time distance, are removed from the list. This means that in worst cases, all events in the list must be checked and removed, which means, the worst case run time of the state transition is linear dependent of the length of the list and therefore is $\mathcal{O}(upper)$. %In normal cases, only a few or none events must be removed, which are in the beginning of the list. Therefore, a nearly constant time behaviour can be expected.\\
	The output function checks, if the updated list of unmatched $source$ events is either empty or the event in the head of the updated list is not older than $upper$. Therefore, it is in $\mathcal{O}(1)$.\\
	The required delay period is calculated by adding $upper$ to the timestamp of the head of the list of unmatched $source$ events, subtracted by the timestamp of the current event ($\mathcal{O}(1)$).
	
\subsection{StrongDelayConstraint}
	The \emph{StrongDelayConstraint} is implemented very similarly to the \emph{DelayConstraint}. The only difference in the state transition is that exactly one event, which is the head of the list of unmatched $source$ events, is removed, when a matching $target$ event occurs. Therefore, the maximal memory usage is the same ($\mathcal{O}(upper)$), but the run time of the state transition is constant per input timestamp.
	In addition to the \textit{DelayConstraint}, the output function of this constraint checks, if each $target$ event occurrence has exactly one matching $source$ event (which always is in the head of the list). Therefore, it is still in $\mathcal{O}(1)$. The calculation of the delay period remains unchanged.
	
\subsection{RepeatConstraint}
	The implementation of the \emph{RepeatConstraint} stores the timestamps of the $span$ previous events as state, using TeSSLa's $last$ operator recursively (a macro called $nLastTime$ was programmed for this). Therefore, $span$ timestamps are stored and the $last$ operator is called $span$ times, which means the state transition function is linear dependent of $span$ in terms of time and the memory usage is likewise.\\
	The required delay is calculated by adding $upper$ to the  $span^{th}$ oldest event (or the first event, if there hasn't been $span$ events before) minus the current timestamp, therefore, the time for the calculation is linear dependent on the $span$ parameter, because the entire recursive definition of the state mentioned above must be walked trough.\\
	The output function checks, if the $span^{th}$ oldest event is not older than $upper$ and not younger than $lower$. If there hasn't been $span$ events before, it is checked, if the first event is not older than $upper$. Like in the calculation of the required delay, the entire recursive definition of the considered for the evaluation. Therefore, the output function is in $\mathcal{O}(span)$.
	
\subsection{RepetitionConstraint}
	The  \emph{RepetitionConstraint} is defined as\\[10pt]
		$RepetitionConstraint(s, lower, upper, span, jitter)$\\
		$\equiv \exists X\subset \mathbb{T}: RepeatConstraint (X, lower, upper, span)$\\
		\hspace{7cm}$\land$ $StrongDelayConstraint(X, s, 0, jitter)$\\[10pt]
	The implementations of the \emph{Repeat-} and the \emph{StrongDelayConstraint} cannot be used for the implementation of this constraint, because the timestamps of $X$ are unknown.\\
	Relevant for the monitoring are the upper and lower bounds of the elements of $X$, which precede the actual events in the event stream $s$. The bounds are stored as two lists with the length of $span$. One list is containing the lower bounds for the next $span$ $X$, the other list is containing the upper bounds. At every input event, the new boundaries for the $span^{th}$ next $X$ are calculated, the lower bound by $max(List\_head(last(LowerBoundX, e)), time(e)-jitter)$ and the upper bound by $min(List\_head(last(UpperBoundX, e)), time(e))$. These new boundaries are appended to the end of the lists, while the oldest entries in the head of the lists are removed. These two lists with the size of $span$ are the only storage, which size is dependent on the input, therefore the algorithm is in $\mathcal{O}(span)$ in terms of memory. The run time of the state transition function is constant (removing the lists head and appending an entry to the lists).\\
	The output function checks, if the current timestamp is between the lower bound for the current timestamp of $X$ and $jitter$ behind the upper bound for that value. If this is the case, the output is \emph{true}, in any other case, it is false. Because the upper and lower bound for the current $X$ value can be directly accessed (they are the head of the lists), the output function is in $\mathcal{O}(1)$.
	
\subsection{SynchronizationConstraint}
	The \emph{SynchronizationConstraint} is defined via an application of the \emph{DelayConstraint}, but the application uses a set of unknown timestamps($\exists X: ...$), therefore the \emph{DelayConstraint} cannot be used for the implementation of this constraint. Because TeSSLa does not allow to define macros or functions with a variable number of input streams, events of each input timestamp must be placed into an integer list, which contains the index (starting at 1) of all streams, which have an event in this timestamp. This is list then used as a parameter to the implementation. The creation of this list is already implemented for up to 5 streams.\\
	The implementation of the \emph{SynchronizationConstraint} stores all events that occured not longer than $tolerance$ ago in a list. In each entry, this list contains the stream, in which the event occurred, the timestamp of the event occurrence and a boolean variable, that expresses if a fulfilled synchronization cluster for this event has already been found.\\
	This list is updated in every input timestamp in three steps. First, each event occurrences in this timestamp is appended to this list. Second, the list is separated into two parts, one with events older and one with events younger than $tolerance$. The part of old events is still stored in this timestamp, but removed after it. The younger events form the state that is stored for the next event occurrences. Third it is checked, if at least one event of every stream is part of the list of younger events. In this case, a fulfilled synchronization cluster has been found and the boolean variable, that states if a synchronization cluster is found for this event, is set to $true$ for all events in this list.\\
	Similar to the \emph{DelayConstraint}, this list can grow infinitely, when the time domain is uncountable, which is not the case in the used TeSSLa version. Because the TeSSLa uses integers as time domain, at most $|event|\footnote{|event| is the number of streams, not the number of events.}*tolerance$ events can occur in the $tolerance$ interval. Therefore, the algorithm is in $\mathcal{O}(|event|*tolerance)$ in terms of memory. The first step of the state transition is in $\mathcal{O}(|event|*tolerance)$, because at most $|event|$ events must be appended to the list and the list has the maximum length $tolerance$. In worst cases, every event in the list(which is in ascending order) is older than tolerance, therefore the worst case runtime of the separation in the second step of the state transition is in $\mathcal{O}(|event|*tolerance)$ in terms of time. In the third step, the complete stored list of young events must be examined, to check if the cluster is fulfilled and, if needed, every event in the list must be set to fulfilled. Therefore the third step is in $\mathcal{O}(|event|*tolerance)$ in terms of time.\\
	The output function checks, if the boolean variable of each event in the list of events, which are older than $tolerance$, is set to true. If not, the constraint is not fulfilled. Because this list can have the size $|event|*tolerance$, the output function also is in $\mathcal{O}(|event|*tolerance)$ in terms of time.\\
	The required delay is calculated by adding $tolerance$ to the timestamp of the oldest stored unsatisfied event, subtracted by the timestamp of the current timestamp. The list is in ascending order, but the only unsatisfied events are relevant for the delay, which means, the entire list must be checked in worst cases. Therefore, the calculation of the required delay is in $\mathcal{O}(|event|*tolerance)$.
	
\subsection{StrongSynchronizationConstraint}
	The \emph{StrongSynchronizationConstraint} is defined as application of the \emph{StrongDelayConstraint}, but this application cannot be used for the implementation, like in the previous constraint. Similar to the \emph{SynchronizationConstraint}, the events of the input streams must be merged into list.\\
	The difference between the \emph{Synchronization-} and the \emph{StrongSynchronizationConstraint} is, that each event can only be part of one synchronization cluster in the \emph{StrongSynchronizationConstraint}. Therefore, the implementation is different to the implementation of the previous constraint. Not every event is stored separately, only information about synchronization clusters, containing their start time and in which stream an event occurred in this cluster, are stored.\\
	The information about synchronization clusters are stored in a list, which contains a time expression, which marks the starting point and a map, containing a boolean variable for every input stream, which shows, if there already was an event in this stream for this cluster. At event occurrences, the event is either added to one existing synchronization cluster, or a new cluster with the start time of the event is added to the list. For the search of a matching cluster, each event of the list is considered in worst cases, therefore the run time of this part of the state transition is linear to the number of active clusters. In worst cases this number is $tolerance$, when one event occurs in every timestamp in always the same stream. In a second step, every stored cluster is checked, if it is fulfilled. If so, it is removed from the list. To check, if a cluster is fulfilled, one boolean check must be done for every input stream, therefore at most boolean $tolerance*|event|$ checks must be done and the worst case run time of the state transition is in $\mathcal{O}(tolerance * |event|)$. When the events occur in separated synchronization clusters, the list is significantly shorter than $tolerance$ and the run time may be expected to be linear to the number of input streams.\\
	The list storing the clusters is at most $tolerance$ long and the size of individual entries of the list is linear dependent on the number of streams, because they store a boolean variable for every stream. Because of these length restrictions of the list, the algorithm is in $\mathcal{O}(|event|*tolerance)$ in terms of memory.\\
	The output function checks, if the oldest stored (therefore unfulfilled) cluster is older than $tolerance$. This cluster is in the head of the list, therefore the output function is in $\mathcal{O}(1)$ in terms of time. The required delay is calculated by adding $tolerance$ to the timestamp of the oldest stored unsatisfied cluster, subtracted by the timestamp of the current timestamp ($\mathcal{O}(1)$).
	
\subsection{ExecutionTimeConstraint}
	The implementation of the \emph{ExecutionTimeConstraint} is using TeSSLa's \emph{runtime} operator on the \emph{start} and \emph{stop} events, which calculates the absolute runtime without any interruptions. The time of interruptions is also calculated by the calculated by this operator and then summed up. The sum of these interruptions is reseted by every $start$ event. The calculation of this sum with resets, a macro called \textit{resetSum} was programmed, which is a modified version of TeSSLas \textit{ResetCount} function.\\
	TeSSLa's \emph{runtime} operator subtracts the timestamps of the events of the second parameter (in this case \emph{stop} and \emph{resume}) from the timestamps of the events of the first parameter(\emph{start} and \emph{preempt}), therefore it stores the timestamps of the \emph{start} and \emph{preempt} events are stored, additionally to the sum the preemptions.
	For the output, the runtime can be calculated by subtracting the second application (with $preempt$ and $resumse$ as parameters) of TeSSLa's $runtime$ operator from the sum of the first applications (with $start$ and $stop$ as parameters) of this operator. If the runtime should be checked in timestamps without a \emph{stop} event, the second parameter of the first application of the $runtime$ operator must be replaced by a current event. In the implementation this is done by merging all include streams and the delay stream.\\
	The resulting runtime must be smaller or equal to $upper$ in any point of time and greater or equal to $lower$ at \emph{stop} events. The required delay is calculated subtracting the runtime so far from upper. All of these operations are simple arithmetic functions on timestamps, therefore the algorithm is in $\mathcal{O}(1)$ in terms of time. The required storage space is fixed, therefore it is also in $\mathcal{O}(1)$ in terms of memory.

\subsection{OrderConstraint}
	The \emph{OrderConstraint} is defined in a way, so that the number of events on the $source$ stream is equal to the number of events on the $target$ stream and that the $i^{th}$ $source$ event occurs before the  $i^{th}$ $target$ event. The first described property can only be checked, when it is known that no further events will occur. In TeSSLa, this is generally unknown, therefore the implementation has a third input stream, which requires to have exactly one event at the end of the observation.
	
	The implementation counts the number of events on the $source$ and $target$ stream and checks, if the number of $source$ events is larger or equal to the number of $target$ events. In the end of the streams, the number of events on both streams must be equal. Therefore, the stored state consists of two integers and the algorithm is in $\mathcal{O}(1)$ in terms of memory. The incrementations of these counters and the comparison between them are simple arithmetic operations, therefore the state transition and the output function are both in $\mathcal{O}(1)$ in terms of time. The introduction of new timestamps is not required for this constraint, except the one defining the end of the observation, therefore no delay period must be calculated.
	
\subsection{ComparisonConstraint}
	The \emph{ComparisonConstraint} defines comparisons between timestamps. These functionalities are already defined in TeSSLa, therefore no implementation is given as part of this thesis.  
	
\subsection{SporadicConstraint}
	The \emph{SporadicConstraint} is defined as simple application of the \emph{Repetition-} and the \emph{RepeatConstraint}, therefore the \emph{SporadicConstraint} is also implemented as application of them. The implementations of the \emph{Repetition-} and the \emph{RepeatConstraint} are both in $\mathcal{O}(span)$ in terms of time and memory. Because $span$ is fixed to 1 in the \textit{SporadicConstraint}, the implementation is in $\mathcal{O}(1)$ in terms of memory and time.
	
\subsection{PeriodicConstraint}
	The \emph{PeriodicConstraint} is defined as application of the \emph{SporadicConstraint} and is also implemented like this. Because the \emph{SporadicConstraint} is in $\mathcal{O}(1)$ in terms of memory and time, the \emph{PeriodicConstraint} is also.
	
\subsection{PatternConstraint}
	The \emph{PatternConstraint} is defined as application of the \emph{Periodic-}, \emph{Delay-} and \emph{RepeatConstraint}. Because of the set of unknown timestamps $X$, the \emph{Periodic-} and \emph{DelayConstraint} cannot be used for the implementation. The set $X$ is not used in the application of the \emph{RepeatConstraint}, therefore its implementation is used as part of the output function.\\
	The implementation of the \emph{RepeatConstraint} is in $\mathcal{O}(span)$ in terms of time memory. The $span$ attribute is set to 1 in the application, therefore the run time and memory usage is constant in this part.\\
	In the implementation of the \emph{PatternConstraint}, the lower and upper bound for the current timestamp of $X$ is stored. At every event, these bounds are further enclosed, taking the previous known bounds and the bounds implied by the current event
	\begin{align}
		x\in X: &time(event)-\text{\emph{offset}}_{count(event)\text{ mod }|\text{\emph{offset}}|}-jitter \leq x\\
			     &\leq  time(event)-\text{\emph{offset}}_{count(event)\text{ mod }|\text{\emph{offset}}|}
	\end{align}
	into account. The new lower bound is set by using the maximum of the previous lower bound and the lower bound implied by the current event, the new upper bound by using the minimum of the previous upper bound and the upper bound implied by the current event. At every $|\text{\emph{offset}}|^{th}$ event, $period$ is added to the current bounds. The output function checks, if the timestamp of the current event is between the lower bound + $\text{\emph{offset}}_{count(event)\text{ mod }|\text{\emph{offset}}|}$ and the upper bound plus $\text{\emph{offset}}_{count(event)\text{ mod }|\text{\emph{offset}}|}$ plus $jitter$. The access of the map entries is done in constant time, therefore the calculation of these new borders is done in constant time, and the state transition function in $\mathcal{O}(1)$ in terms of time.\\
	The required delay is defined by the time distance between the current timestamp and the upper bound for X, plus the allowed offset of the following event, plus the allows deviation ($jitter$).\\
	The only state stored in the implementation are the upper and lower bound for the  current $x$-value, therefore the implementation itself is in $\mathcal{O}(1)$ in terms of memory, but the size of the \textit{offset}-parameter, which is a map, is not limited in size and the complete algorithm, including the parameters, is $\mathcal{O}(|\text{\textit{offset}}|)$ in terms of memory.
	
\subsection{ArbitraryConstraint}
	The \emph{ArbitraryConstraint} is defined as multiple applications of the \emph{RepeatConstraint} and is also implemented this way. The number of applications of the \emph{RepeatConstraint} is dependent on the number of elements in the $minimum$ and $maximum$ parameters. The runtime of the \emph{RepeatConstraint} is in $\mathcal{O}(1)$ per application and event, therefore it the \emph{ArbitraryConstraint} is in $\mathcal{O}(|minimum|)$ in terms of time. The memory usage of the \emph{RepeatConstraint} is in $\mathcal{O}(span)$. In the application of the \emph{RepeatConstraint}, the $span$ parameter increases for each of the $|minimum| = |maximum|$ applications. Therefore, the implementation is in $\mathcal{O}(\sum_{i=1}^{|minimum|}i)\widehat{=}\mathcal{O}(|minimum|^2+|minimum|)$ in terms of time.

\subsection{BurstConstraint}
	The \textit{BurstConstraint} is defined as twofold application of the \emph{RepeatConstraint} and is also implemented this way. The \textit{RepeatConstraint} is in $\mathcal{O}(span)$ in terms of time and memory. Because the $span$ attribute is set to 1 and $maxOccurrences$ in the applications of the \textit{RepeatConstraint}, the implementation of the \emph{BurstConstraint} is in $\mathcal{O}(maxOccurrences)$ in terms of memory and time.

\subsection{ReactionConstraint}
	The correctness of the \textit{EventChain} is assumed in the implementation. If this property is unknown, it must be checked individually.\\
	The implementation of the \emph{ReactionCostraint} stores a map, which maps the color of $stimulus$ events, which did not have a matching $response$ event yet, to their timestamp. This state is updated at every input event. $Stimulus$ events are inserted into the map, $response$ events remove, if possible, a event from the map called above. Similar to the \emph{DelayConstraint}(the \emph{ReactionCostraint} is an extension of the \emph{DelayConstraint}, that additionally considers the color of events), the maximal number of entries in the map is the maximal number of $stimulus$ events, that could possibly occur in an interval of the length $maximum$, which is $maximum$.  Therefore, the algorithm is in $\mathcal{O}(maximum)$ in terms of memory. The state transition (insertion, lookup and possibly remove in map) is effectively in $\mathcal{O}(1)$ in terms of time.\\
	The required delay is calculated by adding $maximum$ to the timestamp of the oldest entry in the map mentioned above, and subtracting the current timestamp. Because the map is unsorted, every entry of the map must be considered for this. Therefore, the calculation of the required delay is in the time complexity class $\mathcal{O}(maximum)$.\\
	The output function checks, if the oldest entry in the map is not older than $maximum$. The run time of this operation is linear in the size of the map, which is at most $maximum$. Like the calculation of the required delay, the evaluation function is in $\mathcal{O}(maximum)$ in terms of time.
	
\subsection{AgeConstraint}
	Like before, the correctness of the \textit{EventChain} is assumed in the implementation. If this property is unknown, it must be checked individually.\\
	Similar to the implementation of the \emph{ReactionCostraint}, the \emph{AgeConstraint} monitor stores a map containing the latest $stimulus$ event, which are younger than maximum. The $color$ value is used as key and the timestamp is used as map value. This map has the maximal size $maximum$ and is updated at every input event. $Stimulus$ events are inserted or updated, and entries, that are older than $maximum$ are removed. To make this update faster, a list containing the colors of the events in the map is stored additionally. The maximal size of this list is also $maximum$ and the colors are stored in chronological order, so that the color, that occurred the longest time ago, is in the head of the list. The update is done by looking at the head of the list and removing this entry from the list and the corresponding entry with the same color from the map, if the entry is older than $maximum$. These operations are done in constant time, but has to be repeated, as long as the color in the head of the map is too old, so at most $maximum$ times. Inserting or updating the $stimulus$ event to the map is done in effectively constant time, but inserting or updating the list requires to remove any previous entry with the color of the current event. For this, every entry in the map has to be processed, which means this operation takes $maximum$ steps in worst cases. Consecutively, the state and the state transition is in $\mathcal{O}(maximum)$ in terms of memory and time. The creation of new timestamps is not needed in this constraint, because only previous events need to be considered, upcoming events not.\\
	In timestamps containing a $response$ event, the output function checks, if a $stimulus$ event with the same color is in the map and if the time distance between them is greater or equal to $minimum$ and smaller or equal to $maximum$. The lookup in the map and the comparisons are done in effectively constant time.

\subsection{OutputSynchronizationConstraint}
	%TODO aufruf von InputSynchronizationConstraint and OutputSynchronizationConstraint
	Similar to the \emph{SynchronizationConstraint} and \emph{StrongSynchronizationConstraint}, the input streams cannot be directly used as parameter. For the \emph{OutputSynchronizationConstraint}, a stream of maps must be created, which represents the events of each timestamp. The key of the entries is the index of the stream (0 for the $stimulus$ stream, 1, 2, ... for the response streams) and the value the color of the event. Again, the creation of this map is already implemented for up to 5 $response$ streams.\\
	In the \emph{OutputSynchronizationConstraint}, for each $stimulus$ event, there must be one synchronization cluster of the length $tolerance$, in which each $response$ stream must have at least one event of the same color as the $stimulus$ event. There is no time distance between the cluster and the $stimlus$ event defined, it just has to be before the end of the streams. Therefore, a additional event, which shows the end of the observation, is needed, similar to the \emph{OrderConstraint}.\\
	The implementation of the \emph{OutputSynchronizationConstraint} is storing four different informations as state. First, a list of every color that occurred in $stimulus$. This is updated at every $stimulus$ event by appending its color to the list(run time: $\mathcal{O}(1)$, memory: $\mathcal{O}(count(stimulus))$).\\
	Second, a map containing information about all synchronization clusters, that were not finished until this point in time is stored. This map is using the color attribute as key and the start time stamp and a map as value. This inner map contains a boolean variable for each $response$ stream, which shows, whether there was an event for this synchronization cluster in this stream or not. This map is updated at every $response$ event. For each of these $response$ events, it is checked, if a synchronization cluster with a matching color exists, if not a new synchronization cluster with the color of the event is created. The check per event (two lookups in maps)  is done in constant time, therefore this update is in $\mathcal{O}(|response|)$ in terms of time per input timestamp. In worst cases, each event results in the creation of a new synchronization cluster, which must be stored at least for the length of $tolerance$. The size of each information about one synchronization cluster is linear dependent on the number of $response$ streams and in each interval of the length $tolerance$, $tolerance*|response|$ events can occur and create a new synchronization cluster, therefore this information is in $\mathcal{O}(tolerance*|response|^2)$ in terms of memory. 
	The third stored information is similar to the second, but the clusters, that either older than tolerance or fulfilled are removed from the map. Therefore, the worst case memory consumption is the also $\mathcal{O}(tolerance*|response|^2)$. To remove fulfilled clusters, it is checked for each cluster in the map, if there was at least one event in each $response$ stream of the color of the cluster. Therefore, this update is in $\mathcal{O}(tolerance*|response|^2)$ in terms of time.
	The fourth stored information is a list of all colors, that had a fulfilled synchronization cluster in the $response$ streams until this point in time. Appending items into a list is done in time. The number of fulfilled synchronization clusters is at most the number events in all $response$ streams, divided by the number of the $response$ streams. Therefore, the required memory of this information is in $\mathcal{O}\left(\frac{\sum_i count(response_i)}{|response|}\right)$, the runtime for updating this information is likewise.\\
	The combined time complexity class is $\mathcal{O}(tolerance*|response|^2)$. The maximum of the memory complexity classes, which defines the memory complexity of the algorithm, is $\mathcal{O}\left(count(stimulus)+\frac{\sum_i count(response_i)}{|response|}+tolerance*|response|^2\right)$.\\
	The required delay is calculated by adding $tolerance$ to the start time of the oldest unfinished cluster and subtracting the current timestamp ($\mathcal{O}(tolerance*|response|^2)$).\\
	The output function checks that all stored synchronization clusters are either younger than $tolerance$ or fulfilled.
	Because the entries of the map, that stores the synchronization clusters, cannot be accessed in way, that is sorted by age, every entry of the map must be checked for its age (at most $tolerance*|response|$ checks). For every synchronization cluster, that is older than $tolerance$, it must be checked, if this cluster is fulfilled. The check of a single cluster requires to check the boolean variables of each stream. Per timestamp, at most $|response|$ synchronization cluster can be started, therefore at most $response$ clusters grow older than $tolerance$ per timestamp. Therefore, the output function is in $\mathcal{O}(tolerance*|response|^2)$ in terms of time per input timestamp.\\
	At the end of the observation, it must be checked, if each $stimulus$ event had a matching synchronization cluster. For each of the at most $count(stimulus)$ $stimulus$ colors, a lookup in a set must be done, therefore this check is in $\mathcal{O}(count(stimulus))$ and the complete output function, including the check at the end of observation, is in $\mathcal{O}(tolerance*|response|^2 + count(stimulus))$ in terms of time.



\subsection{InputSynchronizationConstraint}
	The input streams must be transformed into a $map[Int, Int]$ stream, similar to the previous constraint, but this time the index 0 indicate the $response$ stream and the indices 1, 2, ... indicate the $stimulus$ streams.\\
	The \emph{InputSynchronizationConstraint} is defined very similar to the \emph{OutputSynchronizationConstraint}. The difference is, that the synchronization occurs in a set of $stimulus$ events, not in $response$ events.\\
	Despite the similarities, monitoring the \emph{InputSynchronizationConstraint} is simpler. As state, a map that uses the numbers 1 to $|stimulus|$ as keys and as values a second map that uses colors (integer) as key and the timestamp of the latest occurrence of this color in the stream (the stream is defined by the key of the outer map). This map is updated at every $stimulus$ event, at which either the timestamp of the latest occurrence of this color in this stream is updated, or a new inner map entry is created for this color.  The update of an existing entry is done in constant time, but the time for initializing a new entry is linear depenedent on the number of $stimulus$ streams. Because $stimulus$ events may occur and introduce a new color in each timestamp, the state transition is in $\mathcal{O}(|stimulus|^2)$ in terms of time. The memory size of this information is in $\mathcal{O}(|stimulus|*count(stimulus))$, because the map described above possibly stores every input event of the $stimulus$ streams, when they introduce a new color and therefore a new entry in the inner map of the stream must be created. $Response$ events are not considered for the creation of new timestamps.\\
	The creation of new timestamps is not needed in this constraint, because only previous events need to be considered. Therefore, the calculation of a delay span is not required.\\
	For the output function, in timestamps containing a $response$ event, it must be checked, if the last occurrences of the corresponding color in the $stimulus$ stream form a valid synchronization cluster. This is done by searching the youngest and oldest event with this color in the map of latest $stimulus$ events. If a event of this color is missing, the age is interpreted as $infty$ and $-\infty$. Because the color value is the key of the inner map, the time for searching the oldest and youngest event of this color is linear to the number of $stimulus$ streams. Therefore, the output function is in $\mathcal{O}(|stimulus|)$ in terms of time.
	
\subsection{EventChain}
	Additionally to the 18 TADL2 timing constraints, a monitor, which checks the correctness of \textit{EventChain} was implemented. A \textit{EventChain} is defined on a $stimulus$ and a $response$ stream as:\\[10pt]
	$\forall x \in stimulus:\forall y\in response: x.color=y.color\Rightarrow x<y$\\[10pt]
	As a state, a set, which contains all colors that previously occurred in $reponse$ is stored. This set is updated at each $response$  event by a an insertion into a set ($\mathcal{O}(1)$). The maximal size of this map is the number of events in $response$, therefore the state is in $\mathcal{O}(count(response))$ in terms of memory.\\
	The output function checks, if any occurring $stimulus$ event is not in the set of $response$ events, which is checked in constant time.
	
\subsection{Conclusion}
Table~\ref{tab:complexityClasses} gives an overview of the worst case memory consumption and the worst case run time per input timestamp. The worst case memory requirement and the runtime per input timestamp of the \textit{Repeat-}, \textit{Repetition-}, \textit{ExecutionTime-}, \textit{Sporadic-}, \textit{Periodic-}, \textit{Pattern-}, \textit{Arbitrary-} and \textit{BurstConstraint}, which are the \textit{simple monitorable} constraints, are either constant, or they are only limited by the parameters of the constraint, not by the input traces. The implementations of the \textit{Delay-}, \textit{StrongDelay-}, \textit{Synchronization-}, \textit{StrongSynchronization-}, \textit{Reaction-} and \textit{AgeConstraint} are limited by the events, which may occur in time intervals of a specific length. Monitoring the correctness of \textit{EventChains}, the \textit{OutputSynchronization-} or the \textit{InputSynchronizationConstraint} with these implementations require continuously growing memory resources and in the \textit{OutputSynchronizationConstraint}, the run time per input timestamp is continuously growing too. The implementation of the \textit{OrderConstraint} is in $\mathcal{O}(1)$ in terms of memory and time per event, although it is classified as \textit{Not simple monitorable}. This is, because integers of a fixed length are used for the implementation of the constraint and only a finite subset of all streams that fulfill the constraint can be monitored correctly.

	\begin{table}
		\begin{tabular}{|c|c|c|}
			\hline
			& Memory & \makecell{Run Time per Input\\Timestamp} \\
			\hline
			{DelayConstraint} & $\mathcal{O}(upper)$ & $\mathcal{O}(upper)$ \\
			\hline
			{StrongDelayConstraint} &  $\mathcal{O}(upper)$ &  $\mathcal{O}(1)$ \\
			\hline
			{RepeatConstraint} & $\mathcal{O}(span)$ & $\mathcal{O}(span)$ \\
			\hline
			{RepetitionConstraint} & $\mathcal{O}(span)$ & $\mathcal{O}(1)$ \\
			\hline
			SynchronizationConstraint & $\mathcal{O}(|event|*tolerance)$ & $\mathcal{O}(|event|*tolerance)$ \\
			\hline
			StrongSynchronizationConstraint & $\mathcal{O}(|event|*tolerance)$ & $\mathcal{O}(|event|*tolerance)$ \\
			\hline
			ExecutionTimeConstraint & $\mathcal{O}(1)$ & $\mathcal{O}(1)$ \\
			\hline
			OrderConstraint & $\mathcal{O}(1)$ & $\mathcal{O}(1)$ \\
			\hline
			SporadicConstraint & $\mathcal{O}(1)$ & $\mathcal{O}(1)$ \\
			\hline
			PeriodicConstraint & $\mathcal{O}(1)$ & $\mathcal{O}(1)$ \\
			\hline
			PatternConstraint & $\mathcal{O}(1)$ & $\mathcal{O}(1)$ \\
			\hline
			ArbitraryConstraint & $\mathcal{O}(|minimum|)$& \makecell{$\mathcal{O}(|minimum|^2$\\$+|minimum|)$} \\
			\hline
			BurstConstraint & $\mathcal{O}(maxOccurrences)$ & $\mathcal{O}(maxOccurrences)$ \\
			\hline
			ReactionConstraint & $\mathcal{O}(maximum)$ & $\mathcal{O}(maximum)$ \\
			\hline
			AgeConstraint & $\mathcal{O}(maximum)$ & $\mathcal{O}(maximum)$ \\
			\hline
			OutputSynchronizationConstraint& \makecell{$\mathcal{O}(count(stimulus)$\\+$\frac{\sum_i count(response_i)}{|response|}$\\+$tolerance$\\$*|response|^2)$} &  \makecell{$\mathcal{O}(tolerance$\\$*|response|^2)$\tablefootnote{ $\mathcal{O}(tolerance*|response|^2 + count(stimulus))$ at the end of the observation}} \\
			\hline
			InputSynchronizationConstraint& \makecell{$\mathcal{O}(|stimulus|$\\$*count(stimulus))$} & $\mathcal{O}(|stimulus|^2)$ \\
			\hline
			EventChain & $\mathcal{O}(count(response))$ & $\mathcal{O}(1)$\\
			\hline
		\end{tabular}
		\centering
		\label{tab:complexityClasses}
		\caption{Worst Case Run Times of the Implementations}
	\end{table}

	
	

  %!TEX root = thesis.tex

\chapter{Zusammenfassung und Ausblick}
\label{chapter-fazit}

Die Zusammenfassung greift die in der Einleitung angerissenen Bereiche wieder auf und erläutert, zu welchen Ergebnissen diese Arbeit kommt. Dabei wird insbesondere auf die neuen Erkenntnisse und den Nutzen der Arbeit eingegangen.

Im anschließenden Ausblick werden mögliche nächste Schritte aufgezählt, um die Forschung an diesem Thema weiter voranzubringen. Hier darf man sich nicht scheuen, klar zu benennen, was im Rahmen dieser Arbeit nicht bearbeitet werden konnte und wo noch weitere Arbeit notwendig ist.

% TODO variablenüberläufe, z.B. bei orderConstraint
% TODO manche (patternConstraint) finite monitorable benutzen map/list und sind somit nicht auf FGPA kompilierbar
% Optimierungspotential, z.B. ArbitraryConstraint

  \appendix

  %!TEX root = thesis.tex

\chapter{Anhang}

Dieser Anhang enthält tiefergehende Informationen, die nicht zur eigentlichen Arbeit gehören.

\section{Abschnitt des Anhangs}

In den meisten Fällen wird kein Anhang benötigt, da sich selten Informationen ansammeln, die nicht zum eigentlichen Inhalt der Arbeit gehören. Vollständige Quelltextlisting haben in ausgedruckter Form keinen Wort und gehören daher weder in die Arbeit noch in den Anhang. Darüber hinaus gehören Abbildungen bzw. Diagramme, auf die im Text der Arbeit verwiesen wird, auf keinen Fall in den Anhang.

\section{Event Feeder}
\label{sec:Evet_Feeder}

  \backmatter

  \cleardoublepage
  \phantomsection
  \pdfbookmark{Abbildungsverzeichnis}{listoffigures}
  \listoffigures

  \cleardoublepage
  \phantomsection
  \pdfbookmark{Tabellenverzeichnis}{listoftables}
  \listoftables

  \cleardoublepage
  \phantomsection
  \pdfbookmark{Definitions- und Theoremverzeichnis}{listoftheorems}
  \renewcommand{\listtheoremname}{Definitions- und Theoremverzeichnis}
  %\listoftheorems[ignoreall,show={Lemma,Theorem,Korollar,Definition}]

  \cleardoublepage
  \phantomsection
  \pdfbookmark{Quelltextverzeichnis}{listoflistings}
  \lstlistoflistings

  %!TEX root = thesis.tex

\cleardoublepage
\phantomsection
\pdfbookmark{Abkürzungsverzeichnis}{abbreviations}
\chapter*{Abbreviations}
\label{section-abbrevs}

\begin{tabularx}{\textwidth}{lX}
	AUTOSAR & Automotive Open System Architecture\\
	EAST-ADL  & Electronics Architecture and Software Technology-Architecture Description Language\\
	TIMMO	& Timing Model (EAST-ADL)\\
	TIMMO-2-USE & Updated version of TIMMO\\
	TADL2 & Timing Augmented Description Language V2\\
  	DFST & Deterministic Finite State Transducer\\
  	TDFST & Timed Deterministic Finite State Transducer\\
  	LTL & Linear Temporal Logic\\
  	RV	& Runtime Verification

\end{tabularx}


  \cleardoublepage
  \phantomsection
  \pdfbookmark{Literaturverzeichnis}{bibliography}
  \bibliography{literature}
\end{document}
