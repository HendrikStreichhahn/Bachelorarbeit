\documentclass{scrbook}

%!TEX root = thesis.tex

% Set german to default language and load english as well
\usepackage[english]{babel}

% Set UTF8 as input encoding
\usepackage[utf8]{inputenc}

% Set T1 as font encoding
\usepackage[T1]{fontenc}
% Load a slightly more modern font
\usepackage{lmodern}
% Use the symbol collection textcomp, which is needed by listings.
\usepackage{textcomp}
% Load a better font for monospace.
\usepackage{courier}
% Load package to configure header and footer
\usepackage{scrlayer-scrpage}

% Set some options regarding the document layout. See KOMA guide
\KOMAoptions{%
  paper=a4,
  fontsize=10.95pt,
  parskip=half,
  headings=normal,
  BCOR=1.5cm,
  headsepline,
  headsepline=0.5pt,
  DIV=15}

% do not align bottom of pages
\raggedbottom

% set style of captions
\setcapindent{0pt} % do not indent second line of captions
\setkomafont{caption}{\small}
\setkomafont{captionlabel}{\bfseries}
\setcapwidth[c]{0.9\textwidth}

% set the style of the bibliography
\bibliographystyle{alphadin}

% load extended tabulars used in the list of abbreviation
\usepackage{tabularx}

% load the color package and enable colored tables
\usepackage[table]{xcolor}

% define new environment for zebra tables
\newcommand{\mainrowcolors}{\rowcolors{1}{maincolor!25}{maincolor!5}}
\newenvironment{zebratabular}{\mainrowcolors\begin{tabular}}{\end{tabular}}
\newcommand{\setrownumber}[1]{\global\rownum#1\relax}
\newcommand{\headerrow}{\rowcolor{maincolor!50}\setrownumber1}

% add main color to section headers
\addtokomafont{chapter}{\color{maincolor}}
\addtokomafont{section}{\color{maincolor}}
\addtokomafont{subsection}{\color{maincolor}}
\addtokomafont{subsubsection}{\color{maincolor}}
\addtokomafont{paragraph}{\color{maincolor}}

% do not print numbers next to each formula
\usepackage{mathtools}
\mathtoolsset{showonlyrefs}
% left align formulas
\makeatletter
\@fleqntrue\let\mathindent\@mathmargin \@mathmargin=\leftmargini
\makeatother

% Allow page breaks in align environments
\allowdisplaybreaks

% header and footer
\pagestyle{scrheadings}
\setkomafont{pagenumber}{\normalfont\sffamily\color{maincolor}}
\setkomafont{pageheadfoot}{\normalfont\sffamily}
\setkomafont{headsepline}{\color{maincolor}}

% German guillemets quotes
%\usepackage[german=guillemets]{csquotes}

% load TikZ to draw diagrams
\usepackage{tikz}

% load additional libraries for TikZ
\usetikzlibrary{%
  automata,%
  positioning,%
}

% set some default options for TikZ -- in this case for automata
\tikzset{
  every state/.style={
    draw=maincolor,
    thick,
    fill=maincolor!18,
    minimum size=0pt
  }
}

% load listings package to typeset sourcecode
\usepackage{listings}

% set some options for the listings package
\lstset{%
  upquote=true,%
  showstringspaces=false,%
  captionpos=b,%
  basicstyle=\ttfamily,%
  keywordstyle=\color{keywordcolor}\slshape,%
  commentstyle=\color{commentcolor}\itshape,%
  stringstyle=\color{stringcolor}}
\renewcommand{\lstlistingname}{Quelltext}
\renewcommand{\lstlistlistingname}{Quelltextverzeichnis}

% enable german umlauts in listings
\lstset{
  literate={ö}{{\"o}}1
           {Ö}{{\"O}}1
           {ä}{{\"a}}1
           {Ä}{{\"A}}1
           {ü}{{\"u}}1
           {Ü}{{\"U}}1
           {ß}{{\ss}}1
}

% define style for pseudo code
\lstdefinestyle{pseudo}{language={},%
  basicstyle=\normalfont,%
  morecomment=[l]{//},%
  morekeywords={for,to,while,do,if,then,else},%
  mathescape=true,%
  columns=fullflexible}

% load the AMS math library to typeset formulas
\usepackage{amsmath}
\usepackage{amsthm}
\usepackage{thmtools}
\usepackage{amssymb}

% load the paralist library to use compactitem and compactenum environment
\usepackage{paralist}

% load varioref and hyperref to create nicer references using vref
%\usepackage[ngerman]{varioref}
\PassOptionsToPackage{hyphens}{url} % allow line break at hyphens in URLs
\usepackage{hyperref}

% setup hyperref
\hypersetup{breaklinks=true,
            pdfborder={0 0 0},
            %ngerman,
            pdfhighlight={/N},
            pdfdisplaydoctitle=true}

% Fix bugs in some package, e.g. listings and hyperref
\usepackage{scrhack}

% Allow todos
\usepackage{todonotes}

% define german names for referenced elements
% (vref automatically inserts these names in front of the references)
%\labelformat{figure}{Abbildung\ #1}
%\labelformat{table}{Tabelle\ #1}
%\labelformat{appendix}{Anhang\ #1}
%\labelformat{chapter}{Kapitel\ #1}
%\labelformat{section}{Abschnitt\ #1}
%\labelformat{subsection}{Unterabschnitt\ #1}
%\labelformat{subsubsection}{Unterunterabschnitt\ #1}
%\AtBeginDocument{\labelformat{lstlisting}{Quelltext\ #1}}

% define theorem environments
%\declaretheorem[numberwithin=chapter,style=plain]{Theorem}
%\labelformat{Theorem}{Theorem\ #1}

%\declaretheorem[sibling=Theorem,style=plain]{Lemma}
%\labelformat{Lemma}{Lemma\ #1}

%\declaretheorem[sibling=Theorem,style=plain]{Korollar}
%\labelformat{Korollar}{Korollar\ #1}

%\declaretheorem[sibling=Theorem,style=definition]{Definition}
%\labelformat{Definition}{Definition\ #1}

%\declaretheorem[sibling=Theorem,style=definition]{Beispiel}
%\labelformat{Beispiel}{Beispiel\ #1}

%\declaretheorem[sibling=Theorem,style=definition]{Bemerkung}
%\labelformat{Bemerkung}{Bemerkung\ #1}

%!TEX root = thesis.tex

% Use this file to define some macros you need in your thesis. A macro is a short command that inserts some mathematical symbols or texts you do not want to retype each time you need some. I recommend to use as many macros as possible, because you are able to change them later. For example if you use the same macro each time you need to give the formal semantics of an expression you can easily change the appearance of these brackets by updating the macro later on.

% Set of natural numbers
\newcommand{\N}{\mathbb{N}}

% The default epsilon does not look very nice
\let\epsilon\varepsilon

% If you need to use mathematical expressins like an epsilon in the section titles of your thesis you will end up with warnings that these special symbols cannot be included in the PDF favorites. The following macro uses the mathematical symbol during the text of the thesis and the string "Epsilon" in the PDF favorites.
\newcommand{\pdfepsilon}{\texorpdfstring{$\epsilon$}{Epsilon}}


% Set title and author used in the PDF meta data
\hypersetup{
  pdftitle={Monitoring der Autosar Timing Extensions mittesl TeSSLa},
  pdfauthor={Hendrik Streichhahn}
}

% Depending on which of the following two color schemes you import your thesis will be in color or grayscale. I recommend to generate a colored version as a PDF and a grayscale version for printing.

%%!TEX root = thesis.tex

% define color of example university
\xdefinecolor{exampleuniversity}{rgb}{1, 0.5, 0}

\colorlet{maincolor}{exampleuniversity}

\colorlet{stringcolor}{green!60!black}
\colorlet{commentcolor}{black!50}
\colorlet{keywordcolor}{maincolor!80!black}

\newcommand{\imagesuffix}{-color}
%!TEX root = thesis.tex

\colorlet{maincolor}{black}

\colorlet{stringcolor}{black}
\colorlet{commentcolor}{black!50}
\colorlet{keywordcolor}{black}

\newcommand{\imagesuffix}{-gray}

\newcommand{\duedate}{1.1. 1970}

\begin{document}
  \frontmatter
  %!TEX root = thesis.tex

\begin{titlepage}
  \thispagestyle{empty}

  \vskip1cm

  \pgfimage[height=2.5cm]{uni-logo-example\imagesuffix}
  
  \vskip2.5cm
  
  \LARGE
  
  \textbf{\sffamily\color{maincolor}Monitoring der AUTOSAR Timing Extensions mittels TeSSLa}

  \textit{Monitoring of the AUTOSAR Timing Extensions with TeSSLa}

  \normalfont\normalsize

  \vskip2em
  
  \textbf{\sffamily\color{maincolor}Bachelorarbeit}

  im Rahmen des Studiengangs \\
  \textbf{\sffamily\color{maincolor}Informatik} \\
  der Universität zu Lübeck

  \vskip1em

  vorgelegt von \\
  \textbf{\sffamily\color{maincolor}Hendrik Streichhahn}

  \vskip1em
  
  ausgegeben und betreut von \\
  \textbf{\sffamily\color{maincolor}Prof. Dr. Martin Leucker}

  \vskip1em

  mit Unterstützung von\\
  Martin Sachenbacher und\\
  Daniel Thoma

  \vskip1em


  \vfill

  Lübeck, den \duedate
\end{titlepage}

  %!TEX root = thesis.tex

\cleardoublepage
\thispagestyle{plain}
\vspace*{\fill}

\section*{Erklärung}

Ich erkläre hiermit an Eides statt, dass ich diese Arbeit selbständig verfasst und keine
anderen als die angegebenen Quellen und Hilfsmittel benutzt habe.

\vskip2cm

\rule{5cm}{0.4pt}\\
(Hendrik Streichhahn)\\
Lübeck, den \duedate

  %!TEX root = thesis.tex

\cleardoublepage
\thispagestyle{plain}

\pdfbookmark{Abstract}{abstract}
\paragraph{Abstract}
Satisfying given timing requirements is essential for the correct behavior of embedded real-time systems.
In the automotive domain, the AUTOSAR timing extensions are a recent and widely accepted standard for specifying timing requirements.
Previous work, such as the TIMMO-2-USE project, has focused on formalizing the AUTOSAR timing model and timing extensions in a mathematically rigorous way, in order to make them amenable for off-line system analysis tools such as automated model-checking and verification.\\
Because of computational problems, model-checking and offline verification is limited to relatively small-scale systems. Furthermore, not all types of specification violations can be detected at system development time, and sporadic, rare events typically require a capability for long-term observations.
Run-time verification is a more lightweight method that lies at the boundary between formal verification and testing. Run-time verification checks properties, expressed in temporal logic, on-the-fly during the operation of the system using finite-state monitors generated from the logical specifications. 
In this thesis, an analysis of the 18 TADL2 timing constraints defined in the TIMMO-2-USE project is made to decide, whether they can be expressed as finite-state monitors, thus making them monitorable by runtime verification. Further, a monitor for each of the TADL2 timing constraint is implemented in the temporal stream-based specification language TeSSLa.

\cleardoublepage
\thispagestyle{plain}

\foreignlanguage{german}{%
\pdfbookmark{Kurzfassung}{abstract}
\paragraph{Kurzfassung} 
	Die Einhaltung von Zeitschranken ist essentiell wichtig für das korrekte Verhalten von eingebetteten Echtzeitsystemen.
	In der Automobilindustrie werden in breiter Masse die AUTOSAR Timing Extensions (etwa \textit{AUTOSAR Zeiterweiterungen}) verwendet, mit denen das das Zeitverhalten von Hard- und Softwarekomponenten beschrieben werden kann. Andere Arbeiten, etwa das TIMMO-2-USE Projekt, haben daran gearbeitet, die AUTOSAR Timing Extensions zu formalisieren und somit einen Grundbaustein dafür zu legen, die Definitionen vom Zeitverhalten automatisiert zu kontrollieren, etwa durch Model Checking. Ein Problem von Model Checking und ähnlichen Ansätzen ist, dass diese aufgrund der extrem großen Laufzeit auf kleinere Systeme beschränkt sind. Runtime Verification ist eine leichtgewichtigere Methode der Analyse von Systemkomponenten, die einen Mittelweg zwischen formaler Analyse und Testen geht, wobei formal definierte Eigenschaften des Systems während der Laufzeit geprüft werden.\\
	Im Rahmen dieser Arbeit werden die 18 TADL2 Timing Constraints, welche im Rahmen des TIMMO-2-USE Projekt erarbeitet wurden, dahingehend überprüft, ob sie in mittels Runtime Verification überwacht werden können. Darauf aufbauend wird für jeden dieser Constraints ein Monitor in der Sprache TeSSLa, welche für die Überwachung von Zeiteigenschaften auf Strömen entwickelt wurde, implementiert.
}

  \cleardoublepage
  \phantomsection
  \pdfbookmark{Inhaltsverzeichnis}{tableofcontents}
  \markboth{Inhaltsverzeichnis}{}
  \tableofcontents

  % Remove this for the final version of the thesis!
  \cleardoublepage
  \phantomsection
  \pdfbookmark{Liste der Todos}{listoftodos}
  \listoftodos[Liste der Todos]

  \mainmatter
  %!TEX root = thesis.tex

\chapter{Einleitung}

Das Zeitverhalten ist eine der wichtigsten Eigenschaften von vielen Hard- und Software. Insbesondere in sicherheitskritischen Anwendungen kann eine zeitlich falsche Reaktion verheerende Folgen haben, so kann zum Beispiel eine verfrühte oder verspätete Reaktion eines Herzschrittmachers das Leben eines Menschen gefährden. In Cyber-Physical-Systems, wie zum Beispiel der Fahrdynamikregelung in Kraftfahrzeugen (ESP), kann ein fehlerhaftes Zeitverhalten hohe Personen- und Sachschäden hervorbringen, durch die Vernetzung verschiedener Komponenten wird das Erstellen und die Analyse des Zeitverhalten aber erheblich erschwert, da nicht nur die einzelnen Komponenten, sondern auch das Gesamtsystem untersucht werden muss. Auch Umweltschutzaspekte können eine Rolle spielen, da z.B. eine zeitlich fehlerhafte Steuerung eines Verbrennungsmotors zu erhöhten Emissionen führen kann.
Der Wichtigkeit des Zeitverhaltens steht der große Aufwand einer Zeitanalyse und somit auch wirtschaftliche Aspekte, so dass abgewägt werden muss, ob und in welcher Tiefe eine Analyse des Zeitverhaltens nötig ist.\\
Nicht nur in der Entwicklung von Systemen ist die Analyse des Zeitverhaltens ein wichtiger Bestandteil, auch im Betrieb von diesem sollte das Zeitverhalten des Systems und der einzelnen Komponenten geprüft werden, da Schäden an einzelnen Bauteilen nicht ausgeschlossen werden können. Ein frühzeitiges Erkennen dieser Schäden kann Folgeschäden verhindern, außerdem wird die Instandsetzung des Systems erleichtert, da bei der Verwendung geeigneter Monitoringtools die Eingrenzung des Fehlers erleichtert wird.\\
In dieser Arbeit geht es um die Entwicklung eines Monitoringtools, mit dessen Hilfe die online Überwachung von Zeitverhalten, also der Prüfung von Eigenschaften in \emph{nahezu} Echtzeit, ermöglicht wird. Im Fokus der Entwicklung liegt der Automobilbereich, die Ergebnisse sind aber auf andere Bereiche übertragbar.
%Die Einleitung führt zum eigentlichen Thema dieser Arbeit hin. Dabei wird ein großer Bogen gespannt, in dem die Relevanz und der Kontext der untersuchten Thematik deutlich wird. Grundlegende Begriffe aus dem Titel und der Kurzfassung sollten aufgegriffen und definiert werden. Unterstützend können Zitate herangezogen werden, die der Arbeit einen Rahmen geben.

\section{Verwandte Arbeiten}

\textbf{AUTOSAR} (AUTomotive Open System ARchitecture) ist eine Partnerschaft aus Automobilherstellern und dazugehörigen Software, Hardware Unternehmen, deren Zulieferern und weiteren. Ziel dieser Partnerschaft ist die Erstellung offener Standards für Soft- und Hardwarekomponenten im Automobilbereich, sowie deren Entwicklungsprozesse ~\cite{AUTOSAR_History}.\\
Die AUTOSAR Timing Extensions (kurz \textbf{AUTOSAR TIMEX}) spezifieren Constraints, mit denen das Zeitverhalten von Komponenten, die mit Hilfe anderer AUTOSAR Standards definiert wurden, beschrieben werden kann ~\cite{TIMEX}.\\
TODO ~\cite{TIMMO2USE}\\
\textbf{TeSSLa} (Temporal Stream-based Specification Language) ist eine turingfähige Programmiersprache, die zur Analyse und zur Überwachung von Zeitverhalten, insbesondere das von Cyber-Physical Systems. Es nimmt dabei Ströme von Datenpunkten, die mit Zeitstempeln verknüpft sind, entgegen und führt auf diesen Berechnungen durch ~\cite{TeSSLa}.\\
Damit eng verknüpft ist das \textbf{COEMS}-Project, in dem Möglichkeiten von hardwarebasierte, non-intrusive, online Stream Runtime Verification erarbeitet wurden. Hierbei werden vorhandene Debug-Informationen aus einem System mittels einer TeSSLa-Spezifikation, die auf eine FPGA-basierter Hardware übertragen wurde, analysiert.




% Eine wichtiger Abschnitt der Einleitung stellt einen Überblick über verwandte Arbeiten dar. Was wurde bereits in der Literatur untersucht und ist \emph{nicht} Thema dieser Arbeit?

\section{Aufbau der Arbeit}

Neben dieser Einleitung und der Zusammenfassung am Ende gliedert sich diese Arbeit in die folgenden drei Kapitel.
\begin{description}
  \item[\ref{chapter-basics}] beschreibt die für diese Arbeit benötigten Grundlagen. In diesem Kapitel werden \ldots, \ldots und \ldots eingeführt, da diese für die folgenden Kapitel dringend benötigt werden.
  \item[\ref{chapter-konzept}] stellt das eigentliche Konzept vor. Dabei handelt es sich um ein Konzept zur Verbesserung der Welt. Das Kapitel gliedert sich daher in einen globalen und einen lokalen Ansatz, wie die Welt zum Besseren beeinflusst werden kann.
  \item[\ref{chapter-evaluation}] beinhaltet eine Evaluation des Konzeptes aus dem vorherigen Kapitel. Anhand von Simulationen wird in diesem Kapitel untersucht, wie die Welt durch konkrete Maßnahmen deutlich verbessert werden kann.
\end{description}


  %!TEX root = thesis.tex

\chapter{Grundlagen}
\label{chapter-basics}

\section{TeSSLa}

TeSSLa (\textbf{Te}mporal \textbf{S}tream-based \textbf{S}pecification \textbf{La}nguage) ist eine funktionale Programmiersprache, die für die Laufzeitverifikation von Datenströmen konzipiert wurde. In TeSSLa sind Ströme als Folgen von Events definiert, wobei ein Event aus einem Element der jeweiligen Datentypmenge $\mathbb D$ sowie aus einem Zeitwert aus der diskreten Zeitdomäne $\mathbb T$ besteht. In dieser muss es eine totale Ordnung geben und ein Event $b$, welches zeitlich nach einem Event $a$ auftritt, muss einen höheren Zeitwert aufweisen. Innerhalb einer Spezifikation können mehrere Ströme aus unterschiedlichen Datentypmengen $\mathbb D_1, \dots, \mathbb D_n$ verwendet werden, wobei die Zeitdomäne $\mathbb T$ innerhalb einer Spezifikation auf allen Strömen dieselbe sein muss.\\
In TeSSLa wird zwischen synchronen Strömen, in denen alle Ströme einer Spezifikation Events in gemeinsamen Zeitpunkten haben, und asynchronen Strömen unterschieden, bei denen die Zeitpunkte der Events zwar einer globalen Ordnung folgen, die Zeitpunkte aber sonst unanhängig von einander sind.Die Spezifikationen mit synchronen Strömen sind eine echte Teilmenge der Spezifikationen mit asynchronen Strömen, da Ströme mit geordneten, aber ansonsten unabhängigen Zeitpunkten Events mit gleichzeitigen Zeitpunkten auf allen Strömen zulassen. Andersherum gilt diese Relation offensichtlich nicht. Aufgrund dieser Teilmengenrelation werden im Folgenden nur asynchrone Ströme behandelt, wenn von Strömen die Rede ist, sind immer asynchrone gemeint.\\
Die Berechnungen erfolgen, nachdem sie von eintreffenden Events gestartet wurden, wodurch ein Ausgabestrom mit den gleichen Zeitwerten wie Eingabeströme, allerdings kann der \emph{delay}-Operator verwendet werden, um neue Zeitpunkte zu erzeugen, dazu später mehr. Ohne neue Zeitpunkte heißt die Spezifikation \emph{timestamp conservative}. Innerhalb dieser Berechnungen sind Direktzugriffe nur auf die aktuellen Datenwerte der Ströme möglich. Diese Werte bleiben solange bestehen, bis ein neues Event auf diesem Strom eintrifft, der Zeitwert des Events ändert sich hierbei nicht. Mit dem \emph{last}-Operator, welcher auch rekursiv angewendet werden kann, sind Zugriffe auf das jeweils letzte Element möglich. Der \emph{lift}-Operator wendet eine Funktion über Datenwerten auf die Datenwerte jedes eintreffenden Events an. Der \emph{slift}-Operator agiert ähnlich dem \emph{lift}-Operator, allerdings wird die Funktion erst dann angewendet, wenn auf jedem Strom, der dem \emph{slift}-Operator übergeben wurde, bereits ein Event übertragen wurde. (TODO$\rightarrow$ weiter ausführen)\\
In ~\cite{TeSSLa} werden verschiedene Fragmente von TeSSLa beschrieben, die unterschiedliche Mächtigkeiten haben und äquivalent zu verschiedenen Transduktormodellen sind. Im Fragment \emph{TeSSLa$_{bool}$} sind die Datentypmengen der Ströme auf boolesche Werte beschränkt, als Operatoren sind nur der oben genannte \emph{last}-Operator, der \emph{lift}-Operator
  %!TEX root = thesis.tex

\chapter{Monitorbarkeit}
\label{chapter-konzept}

In diesem Kapitel wird die eigentliche Erkenntnis dieser Arbeit beschrieben. Der Aufbau dieses Kapitels hängt stark vom Thema der Arbeit ab. Die in dieser Vorlage vorgeschlagenen Kapitel sind auch nur als Vorschlag und auf keinen Fall als verbindlich zu verstehen.

Die folgenden Abschnitte dieses Kapitels enthalten Beispiele für die diversen Inhaltselemente einer Arbeit.\todo{Die Abschnitte dieses Kapitels sollten natürlich nicht so in die Arbeit übernommen werden.}

\todo[inline]{Notizen an einen selbst oder den Betreuer der Arbeit sind während der Arbeit sehr nützlich. Für die finale Version können diese Todo-Notes dann komplett aufgeblendet werden.}

\section{Quellen}

Quellen sind wichtig für gutes wissenschaftliches Arbeiten. Eine Quelle kann dabei zum Beispiel
\begin{compactitem}
  \item ein Beitrag in einer Zeitschrift \cite{MopOverview},
  \item ein Beitrag in einem Sammlungsband \cite{moore},
  \item ein Buch \cite{scala},
  \item ein Beitrag im Berichtsband einer Konferenz \cite{rltl},
  \item ein technischer Bericht \cite{bitkom},
  \item eine Dissertation \cite{Leucker02},
  \item eine Abschlussarbeit \cite{RltlConv},
  \item ein (noch) nicht veröffentlichter Artikel \cite{ptLTL} oder
  \item ein Artikel auf einer Website \cite{codecommit} sein.
\end{compactitem}

Dabei ist zu beachten, dass nicht veröffentlichte Artikel und insbesondere Webseiten nur in Ausnahmefällen gute Quellen sind, da diese nicht durch Fachleute begutachtet wurden.

Im Bereich der Informatik können Quellenangaben im Bib\TeX-Format direkt der dblp\footnote{zum Beispiel \url{http://dblp.uni-trier.de}} entnommen werden.

\section{Tabellen}

In \vref{tbl-prozessoren} sehen wir ein Beispiel für eine Tabelle.

\begin{table}
  \centering
  \begin{zebratabular}{llr}
    \headerrow Jahr & Prozessor & MHz \\
    1975 & 6502 (C64) & 1 \\
    1985 & 80386 & 16 \\
    2005 & Pentium 4 & 2\,800 \\
    2030 & Phoenix 3 & 7\,320\,000 \\
    \hiderowcolors
    2050 & \ldots \\
    2070 & \ldots
  \end{zebratabular}
  \caption[Rechengeschwindigkeit von Computern]{Rechengeschwindigkeit von Computern. Inhaltlich vollkommen egal, ist dies doch ein sehr schönes Beispiel für eine Tabelle.}
  \label{tbl-prozessoren}
\end{table}

\section{Abbildungen und Diagramme}

In \vref{fig-flower} sehen wir ein Beispiel für eine Abbildung, die aus einer externen Grafik geladen wurde. In \vref{fig-buechi} sehen wir ein Beispiel für eine Abbildung, die in \LaTeX\ generiert wurde.

\begin{figure}
  \centering
  \pgfimage[width=.5\textwidth]{flower}
  \caption[Kurzfassung der Beschreibung für das Abbildungsverzeichnis]{Lange Version der Beschreibung, die direkt unter der Abbildung gesetzt wird. Es ist wichtig, für jede Abbildung eine umfangreiche Beschreibung anzugeben, da der Leser beim ersten Durchblättern der Arbeit vor allem an den Abbildungen hängen bleibt.}
  \label{fig-flower}
\end{figure}

\begin{figure}
  \centering
  \begin{tikzpicture}[
      node distance=15ex,
      auto,
      on grid,
      shorten >=1pt
    ]
    \node [state, initial] (q0) {$q_0$};
    \node [state, accepting, right=of q0] (q1) {$q_1$};
    \path[->]
      (q0) edge node {$a$} (q1);
  \end{tikzpicture}
  \caption[Graph des Büchi-Automaten $\hat A$.]{Graph des Büchi-Automaten $\hat A$. Der Zustand $q_1$ hat dabei keine ausgehende Kante. Der Zustand ist trotzdem akzeptierend, da beide enthaltenen Zustände von $\acute A$ akzeptierend sind. Die naive Anwendung des Leerheitstests auf alternierenden Büchi-Automaten liefert in diesem Fall also zu viele akzeptierende Zustände.}
  \label{fig-buechi}
\end{figure}

\section{Quelltext}

Quelltext sollte in Abschlussarbeiten nur äußerst sparsam eingesetzt werden. Wichtig ist insbesondere, dass Quelltextauszüge sorgsam ausgewählt und gut erklärt werden.

\begin{lstlisting}[language=Java,gobble=2]
  public class Main {
    // Hello Word in Java
    public static void main(String[] args) {
      System.out.println("Hello World");
    }
  }
\end{lstlisting}

\subsection{Quelltext mit automatischer Nummerierung}

Manchmal möchte man Quelltext eher als Abbildung und nicht als Fließtext behandeln. In diesem Fall soll der Quelltext eine Bildunterschrift und eine automatische Nummerierung erhalten. Die automatische Nummerierung kann dann natürlich auch in Referenzen verwendet werden: In \vref{lst:java} haben wir Java-Quelltext.

\begin{lstlisting}[language=Java,gobble=2,caption={Ich bin die Bildunterschrift des Quelltextes},label=lst:java]
  public class AnotherClass {
    private int number = 0;
    public void add() {
      this.number++;
    }
  }
\end{lstlisting}

Wenn man Quelltext mit Bildunterschrift setzt, muss man darauf achten, dass der Quelltext nach wie vor nicht als Fließumgebung behandelt wird. Entsprechend kann es passieren, dass der Quelltext über zwei Seiten hinweg gesetzt wird. Während das normalerweise nicht stört, kann dieser Umstand in Zusammenhang mit einer Bildunterschrift den Leser irritieren.

\section{Pseudocode}

Um Algorithmen zu erklären ist Pseudocode viel besser geeignet als Quelltext. Im Pseudocode kann man alles unwichtige weglassen und sich auf die mathematische Modellierung des Algorithmus konzentrieren. So kann die Struktur des Verfahrens unabhängig von Implementierungsdetails des jeweiligen Frameworks erklärt werden.

\begin{lstlisting}[style=pseudo,gobble=2]
  // Schleife von 1 bis 5
    for $i \gets 1$ to $5$ do
      while $S[i] \neq S[S[i]]$ do
        $S[i] \gets S[S[i]]$
\end{lstlisting}

\section{Formeln mit \pdfepsilon}

Das $\epsilon$ in der Überschrift dieses Abschnitts ist ein Beispiel für ein mathematisches Symbol, dass in den PDF-Lesezeichen als reiner Text gesetzt wird. Siehe \texttt{macros.tex}. In dieser Datei wird auch $n \in \N$ definiert.

Wir wissen aus der Analysis, dass
\begin{align}
  f(x) &= x^2 + px + q
\end{align}
Nullstellen bei
\begin{align}
  x_1 &= -\frac p2 + \sqrt{\frac{p^2}4 - q} \text{ und}\\
  x_2 &= -\frac p2 - \sqrt{\frac{p^2}4 - q}
\end{align}
hat.

\section{Theoreme}

\begin{Definition}[Definition]
  Eine \emph{Definition} ist die Bestimmung eines Begriffs oder eines mathematischen Zusammenhangs.
\end{Definition}

\begin{Lemma}[Unwichtiger Hilfssatz]
  \label{lem:hilfssatz}
  Der Satz \ldots
\end{Lemma}

\begin{proof}
  \ldots und sein Beweis.
\end{proof}

\begin{Theorem}[Wichtiger Satz]
  \label{thm:wichtig}
  Ein wichtiger Satz.
\end{Theorem}

\begin{proof}
  Der Beweis folgt unter Verwendung der Erkenntnisse aus \vref{lem:hilfssatz}.
\end{proof}

\begin{Korollar}[Ein Geschenk]
  Eine unmittelbare Folgerung.
\end{Korollar}

\begin{proof}
  Die Folgerung folgt unmittelbar aus \vref{thm:wichtig}.
\end{proof}

\begin{Beispiel}
  Ein Beispiel.
\end{Beispiel}

\begin{Bemerkung}
  Beispiele und Bemerkungen werden nicht in das Verzeichnis der Theoreme und Definitionen aufgenommen.
\end{Bemerkung}

\section{Anführungszeichen}

%Malte sagt: \enquote{Fritz hat gesagt: \enquote{Man kann wörtliche Rede verschachteln.}}

Anführungszeichen werden nur für wörtliche Rede oder direkt übernommene kurze Zitate verwendet. Für die Hervorhebung von neu eingeführten Fachbegriffen eignet sich \emph{kursiver Satz} besser.

  %!TEX root = thesis.tex
\chapter{AUTOSAR TIMEX Constraints}
\label{chapter-autosar}
  %!TEX root = thesis.tex

\chapter{Implementierungen}
\label{chapter-evaluation}

In der Evaluierung wird das Ergebnis dieser Arbeit bewertet. Eine praktische Evaluation eines neuen Algorithmus kann zum Beispiel durch eine Implementierung geschehen. Je nach Thema der Arbeit kann sich natürlich auch die gesamte Arbeit eher im praktischen Bereich mit einer Implementierung beschäftigen. In diesem Fall gilt es am Ende der Arbeit insbesondere die Implementierung selber zu evaluieren. Wesentliche Fragen dabei können sein:
\begin{compactitem}[--]
  \item Was funktioniert jetzt besser als vor meiner Arbeit?
  \item Wie kann das praktisch eingesetzt werden?
  \item Was sagen potenzielle Anwender zu meiner Lösung?
\end{compactitem}

\section{Implementierungen}

Wenn Implementierungen umfangreich beschrieben werden, ist darauf zu achten, den richtigen Mittelweg zwischen einer zu detaillierten und zu oberflächlichen Beschreibung zu finden. Eine Beschreibung aller Details der Implementierung ist in der Regel zu detailliert, da die primäre Zielgruppe einer Abschlussarbeit sich nicht im Detail in den geschriebenen Quelltext einarbeiten will. Die Beschreibung sollte aber durchaus alle wesentlichen Konzepte der Implementierung enthalten. Gerade bei einer Abschlussarbeit am Institut für Softwaretechnik und Programmiersprachen lohnt es sich, auf die eingesetzten Techniken und Programmiersprachen einzugehen. Ich würde in einer solchen Beschreibung auch einige unterstützende Diagramme erwarten.


  %!TEX root = thesis.tex

\chapter{Zusammenfassung und Ausblick}
\label{chapter-fazit}

Die Zusammenfassung greift die in der Einleitung angerissenen Bereiche wieder auf und erläutert, zu welchen Ergebnissen diese Arbeit kommt. Dabei wird insbesondere auf die neuen Erkenntnisse und den Nutzen der Arbeit eingegangen.

Im anschließenden Ausblick werden mögliche nächste Schritte aufgezählt, um die Forschung an diesem Thema weiter voranzubringen. Hier darf man sich nicht scheuen, klar zu benennen, was im Rahmen dieser Arbeit nicht bearbeitet werden konnte und wo noch weitere Arbeit notwendig ist.

  \appendix

  %!TEX root = thesis.tex

\chapter{Anhang}

Dieser Anhang enthält tiefergehende Informationen, die nicht zur eigentlichen Arbeit gehören.

\section{Abschnitt des Anhangs}

In den meisten Fällen wird kein Anhang benötigt, da sich selten Informationen ansammeln, die nicht zum eigentlichen Inhalt der Arbeit gehören. Vollständige Quelltextlisting haben in ausgedruckter Form keinen Wort und gehören daher weder in die Arbeit noch in den Anhang. Darüber hinaus gehören Abbildungen bzw. Diagramme, auf die im Text der Arbeit verwiesen wird, auf keinen Fall in den Anhang.

\section{Event Feeder}
\label{sec:Evet_Feeder}

  \backmatter

  \cleardoublepage
  \phantomsection
  \pdfbookmark{Abbildungsverzeichnis}{listoffigures}
  \listoffigures

  \cleardoublepage
  \phantomsection
  \pdfbookmark{Tabellenverzeichnis}{listoftables}
  \listoftables

  \cleardoublepage
  \phantomsection
  \pdfbookmark{Definitions- und Theoremverzeichnis}{listoftheorems}
  \renewcommand{\listtheoremname}{Definitions- und Theoremverzeichnis}
  \listoftheorems[ignoreall,show={Lemma,Theorem,Korollar,Definition}]

  \cleardoublepage
  \phantomsection
  \pdfbookmark{Quelltextverzeichnis}{listoflistings}
  \lstlistoflistings

  %!TEX root = thesis.tex

\cleardoublepage
\phantomsection
\pdfbookmark{Abkürzungsverzeichnis}{abbreviations}
\chapter*{Abbreviations}
\label{section-abbrevs}

\begin{tabularx}{\textwidth}{lX}
	AUTOSAR & Automotive Open System Architecture\\
	EAST-ADL  & Electronics Architecture and Software Technology-Architecture Description Language\\
	TIMMO	& Timing Model (EAST-ADL)\\
	TIMMO-2-USE & Updated version of TIMMO\\
	TADL2 & Timing Augmented Description Language V2\\
  	DFST & Deterministic Finite State Transducer\\
  	TDFST & Timed Deterministic Finite State Transducer\\
  	LTL & Linear Temporal Logic\\
  	RV	& Runtime Verification

\end{tabularx}


  \cleardoublepage
  \phantomsection
  \pdfbookmark{Literaturverzeichnis}{bibliography}
  \bibliography{literature}
\end{document}
