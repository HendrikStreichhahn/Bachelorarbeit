%!TEX root = thesis.tex

\chapter{Summary and Outlook}
\label{chapter-fazit}

In this chapter, a summary of the presented work is given. Additionally, some ideas for future work will be presented.
\section{Summary}
	In this thesis, it has been shown that implementing a monitor for the AUTOSAR Timing Extensions is problematic due to informal definitions. Because of this, the timing constraints defined in the Timing Augmented Description Language v2 (TADL2) were considered for the implementation of a monitoring tool. The TADL2 Timing Constraints were presented and the relations between them and the AUTOSAR Timing Constraints were described.\\
	After the relations between the AUTOSAR Timing Extensions and the timing constraints defined in TADL2 were explained, the term \textit{simple monitorable}, which ensures that a property on a possibly infinite trace can be monitored with finite resources were introduced. This term was expended by the possibility of creating new timestamps, which ensures that a violation of a constraint can be detected as early as possible. These terms were applied to the TADL2 timing constraints, with the result that eight of the constraints are simple monitorable with or without delay and nine constraints are not simple monitorable. Seven of the not simple monitorable constraints have memory requirements, which can be infinite in worst-cases. Two of them always have infinite memory requirements when monitoring infinite streams. The term \textit{simple monitorable} is not applicable to one constraint because it is not defined on event streams.\\
	After the theoretical part, an implementation for all of the TADL2 timing constraints in TeSSLa was given, except for the \textit{ComparionConstraints}, whose functionality was already implemented in TeSSLa. The worst-case runtime per timestamp with input events and the memory usage were analyzed. In the end, the runtime of the implementations was measured on large, generated traces.\\
	The timing constraints defined in TADL2 are not equal to the AUTOSAR Timing Extensions. However, some constraints have equal semantics and the implemented monitors can be used for the TADL2 constraints and the AUTOSAR Timing Extensions. Other TADL2 constraints have a less strict semantic than their AUTOSAR counterparts. For these constraints, the implemented monitors can check the correctness of a trace in parts, but not entirely.\\
\section{Future Work}
	For a real-world use of the monitors, more work on this topic is required. It is possible to map TeSSLa-specifications to reconfigurable hardware like FPGAs~\cite{8342124}. Because of memory and recursion restrictions, this is not possible for all specifications. %TODO Quelle
	The possibility of mapping the implementations on reconfigurable hardware would significantly increase the performance and opens the gate for real-world usage in embedded systems in the automotive industry. 
	
	Some constraints were classified as \textit{not simple monitorable} but could be restricted so that they are \textit{simple monitorable}. For example, the \textit{InputSynchronization-} and \textit{OutputSynchronizationConstraint} are classified as \textit{always not simple monitorable} because they need to store every occurring color and therefore have a continuously growing memory usage. If all events have a minimal distance and the color attribute is defined as integers, which occur strictly ordered, a monitor with a fixed memory limit could be built. Restrictions of this kind may be possible to many of the timing constraints classified as \textit{not simple monitorable}. Further work on these possible restrictions is needed because it must be ensured that the monitored system also fulfills the restrictions.


% TODO manche (patternConstraint) finite monitorable benutzen map/list und sind somit nicht auf FGPA kompilierbar
% TODO Optimierungspotential, z.B. ArbitraryConstraint, OutputSynchronizationConstraint
% TODO einschränkung von Color?
% TODO LOLA_eff = finite monitorable

%TODO infinite monitorable ergänzen, um Bufferung etc

% TODO Laufzeitevalution auf sehr künstlichem Setup

%TODO Implementierung der Synchronization Constraints könnten bei 'Überfüllung' früher false geben