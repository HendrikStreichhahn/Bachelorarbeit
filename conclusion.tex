%!TEX root = thesis.tex

\chapter{Zusammenfassung und Ausblick}
\label{chapter-fazit}

Die Zusammenfassung greift die in der Einleitung angerissenen Bereiche wieder auf und erläutert, zu welchen Ergebnissen diese Arbeit kommt. Dabei wird insbesondere auf die neuen Erkenntnisse und den Nutzen der Arbeit eingegangen.

Im anschließenden Ausblick werden mögliche nächste Schritte aufgezählt, um die Forschung an diesem Thema weiter voranzubringen. Hier darf man sich nicht scheuen, klar zu benennen, was im Rahmen dieser Arbeit nicht bearbeitet werden konnte und wo noch weitere Arbeit notwendig ist.

% TODO variablenüberläufe, z.B. bei orderConstraint
% TODO manche (patternConstraint) finite monitorable benutzen map/list und sind somit nicht auf FGPA kompilierbar
% TODO Optimierungspotential, z.B. ArbitraryConstraint, OutputSynchronizationConstraint
% TODO Sinnhaftigkeit der Überwachung der eventChains in Reaction-, Age, OutputSynch, InputSynchronizationConstraint

% TODO Laufzeitevalution mit 'passenderem' setting (Raspberry PI, events kommen nicht aus gespeichertem Trace, sondern von extern)